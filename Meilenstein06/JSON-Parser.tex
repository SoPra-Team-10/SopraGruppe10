\subsection{Klassen-Diagramm}
	\begin{figure}[H]
        \centering
        \includegraphics[scale=1]{images/JSON-Parser.pdf}
    \end{figure}

\subsection{Beschreibung}
	\subsubsection{JSON-Parser (Klasse)}
	
		\begin{description}
                
        	\item[deserialize (Methode)]
        	Die deserialize Methode dient dazu einen dem JSON-Format konformen String in ein JSON-Object umzuwandeln. Dabei wird bei der Umsetzung dieser Funktionalität auf Methoden und Strukturen aus einer Software-Bibliothek eines Drittanbieters zurückgegriffen. 
        	
        	\item[serialize (Methode)]
        	Die serialize Methode dient dazu aus einem JSON-Object ein dem JSON-Format genügenden String zu erzeugen. Wie auch schon bei der deserialize Methode wird dabei auf Methoden und Strukturen aus einer Software-Bibliothek eines Drittanbieters zurückgegriffen. 

    	\end{description}
    	
    \subsubsection{basic\_json (Externe Bibliothek)}
		Bei dieser Klasse Handelt es sich um eine externe Bibliothek, die verschiedenste Methoden und Strukturen zur Verfügung stellt um in einem C++ Projekt mit Objekten mit JSON-Struktur einfach zu arbeiten. Vermutlich wird die von \textit{Niels Lohmann} entwickelte freie JSON Bibliothek namens \textit{JSON for Modern C++} verwendet werden. Genauere Informationen zu dieser Bibliothek sind unter folgendem Link zu finden: \\ \url{https://github.com/nlohmann/json}

\subsection{Zuordnung der Funktionalen Anforderungen}

Die funktionalen Anforderungen gemäß dem Pflichtenheft werden den Komponenten folgendermaßen zugeteilt:


\begin{table}[h]
	\centering
	\begin{tabular}{|l|l|}
    	\hline
    	\textbf{Methode} & \textbf{Abgedeckte funktionale Anforderungen}\\ \hline
    	deserialize & keine explizite FA vorhanden  \\ \hline
    	serialize & keine explizite FA vorhanden \\ \hline


	\end{tabular}
\end{table}