Der Server und die KI wurden in C++17 geschrieben.
Als Compiler wurde dafür der GNU-C++-Compiler (GCC) (Version 8.3, Lizenziert unter GPL 3) verwendet,
als Buildsystem wurde CMake (Version 3.10, Lizenziert unter BSD-3) genutzt.
Zur statischen Codeanalyse wird Clang-Tidy (Version 6.0, Lizenziert unter Apache 2) verwendet,
Unit Tests werden mit Google-Test bzw. Google-Mock (Version 2.56, Lizenziert unter BSD-3) implementiert.
Um undefiniertes Verhalten zu vermeiden wird das Programm während der Ausführung mit Adress Sanitizer (Version 5.0, Lizenziert unter Apache 2) überprüft.

Alle Teile des Quellcodes wurden mithilfe von Doxygen-Kommentaren (Version 1.8, Lizenziert unter GPL) dokumentiert.

Es wurde sich während des Entwicklungsprozess an den Linux-Kernel-Styleguide gehalten werden. Zwei wesentliche Abweichungen von diesem Styleguide sind eine Einrückungstiefe von 4 Zeichen, sowie die Platzierung der öffnenden Klammer bei Funktionsaufrufen in der selben Zeile (die Unterscheidung bei der platizierung von öffnenden geschweiften Klammern ist wohl ursprünglich K\&R zuzuschreiben, die Begründung hierfür ist, dass Funktionen sich nicht verschachteln lassen, dies
ist in C++ mit Lambdas nicht mehr so eindeutig der Fall).

Als Entwicklungsumgebung wurden die IDE CLion (Version 2019.1, proprietär) sowie der Editor vim (Version 8.0, Lizenziert als Careware).

Als externe Librarys wurde libwebsockets (Version 3.1.0, LGPL 2, \url{https://github.com/warmcat/libwebsockets}), 
nlohmann::json (Version 3.6.0 bzw. 3.6.1, MIT, \url{https://github.com/nlohmann/json}) sowie 
MLP (GPL 3, \url{https://github.com/aul12/MLP}) verwendet.
