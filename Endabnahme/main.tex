\NeedsTeXFormat{LaTeX2e}
\documentclass[a4paper,12pt,
headsepline,           % Linie zw. Kopfzeile und Text
oneside,               % einseitig
pointlessnumbers,      % keine Punkte nach den letzten Ziffern in Überschriften
bibtotoc,              % LV im IV
%DIV=15,               % Satzspiegel auf 15er Raster, schmalere Ränder   
%BCOR15mm              % Bindekorrektur
%,draft
]{scrartcl}

\usepackage{amsmath}
\usepackage{amsfonts}
\usepackage{amssymb}
\usepackage{enumitem}
\usepackage[utf8]{inputenc} % this is needed for umlauts
\usepackage[ngerman]{babel} % this is needed for umlauts
\usepackage[T1]{fontenc} 
\usepackage{commath}
\usepackage{xcolor}
\usepackage{booktabs}
\usepackage{float}
\usepackage{tikz-timing}
\usepackage{tikz}
\usepackage{multirow}
\usepackage[final]{pdfpages}
\usepackage{blindtext}
\usepackage[scaled]{helvet}
\usepackage{hyperref}
\usepackage{comment}
\usepackage{mathtools}
\DeclarePairedDelimiter{\ceil}{\lceil}{\rceil}

\usetikzlibrary{calc,shapes.multipart,chains,arrows}

\KOMAoptions{DIV=last} % Neuberechnung Satzspiegel nach Laden von Paket helvet

\usepackage{scrpage2}
\pagestyle{useheadings}

\renewcommand{\familydefault}{\sfdefault} 

\setlength{\parindent}{0pt}   % kein linker Einzug der ersten Absatzzeile
\setlength{\parskip}{1.4ex plus 0.35ex minus 0.3ex} % Absatzabstand, leicht variabel

\newcommand{\fullname}{Gruppe 10}
\newcommand{\titel}{Endabnahmedokument}
\newcommand{\jahr}{2018/2019}
\newcommand{\dozent}{Florian Ege}
\newcommand{\betreuer}{Stefanos Mytilineos}
\newcommand{\fakultaet}{Ingenieurwissenschaften, Informatik und\\Psychologie}
\newcommand{\institut}{Institut für Softwaretechnik und Programmiersprachen}

\pdfinfo{
    /Author (\fullname)
    /Title (\titel)
    /Producer     (pdfeTex 3.14159-1.30.6-2.2)
    /Keywords ()
}

\hypersetup{
    pdftitle=\titel,
    pdfauthor=\fullname,
    pdfsubject={Softwaregrundprojekt-Abgabe},
    pdfproducer={pdfeTex 3.14159-1.30.6-2.2},
    colorlinks=false,
    pdfborder=0 0 0	% keine Box um die Links!
}

% Trennungsregeln
\hyphenation{Sil-ben-trenn-ung}


\newcommand{\begriff}[7] {
	\begin{table}[H]
		\centering
		\begin{tabular}{|p{2,5cm}|p{12cm}|}
			%\hline
			%\toprule \\
			\hline
			\textbf{Begriff} & \textbf{#1} \\ \hline
			%\midrule \\
			\textbf{Beschreibung} & #2 \\ \hline
			%\midrule
			\textbf{Ist-ein} & #3 \\ \hline
			%\midrule
			\textbf{Kann-sein} & #4 \\ \hline
			%\midrule
			\textbf{Aspekt} & #5 \\ \hline
			%\midrule
			\textbf{Bemerkung} & #6 \\ \hline
			%\midrule
			\textbf{Beispiel} & #7 \\ %\hline
			%\bottomrule
			\hline
		\end{tabular}
	\end{table}
}

\newcommand{\anf}[7] {
    \begin{table}[H]
        \centering
        \begin{tabular}{|p{3.2cm}|p{11.3cm}|}
        	\hline
            \textbf{ID:} & \textbf{#1} \\ \hline
            \textbf{Titel:} & #2 \\ \hline
            \textbf{Beschreibung:} & #3 \\ \hline
            \textbf{Begründung:} & #4 \\ \hline
            \textbf{Abhängigkeiten:} & #5 \\ \hline
            \textbf{Priorität:} & #6 \\ \hline
            \textbf{Akteure:} & #7 \\ \hline
        \end{tabular}
    \end{table}
}

\newcounter{fanfCount}
\newcommand{\fanf}[6] {
    \stepcounter{fanfCount}
    \anf{FA\thefanfCount}{#1}{#2}{#3}{#4}{#5}{#6}
}
\newcounter{qanfCount}
\newcommand{\qanf}[6] {
    \stepcounter{qanfCount}
    \anf{QA\theqanfCount}{#1}{#2}{#3}{#4}{#5}{#6}
}

\newcommand{\akt}[4] {
    \begin{table}[H]
        \centering
        \begin{tabular}{|p{3cm}|p{11.5cm}|}
        	\hline
            \textbf{ID:} & \textbf{#1} \\ \hline
            \textbf{Titel:} & #2 \\ \hline
            \textbf{Beschreibung:} & #3 \\ \hline
            \textbf{Rolle:} & #4 \\ \hline
        \end{tabular}
    \end{table}
}

\newcounter{faktCount}
\newcommand{\fakt}[3] {
    \stepcounter{faktCount}
    \akt{AKT\thefaktCount}{#1}{#2}{#3}
}

\begin{document}
    \thispagestyle{empty}
    \begin{addmargin*}[4mm]{-10mm}

        \includegraphics[height=1.8cm]{images/unilogo_bild}
        \hfill
        \includegraphics[height=1.8cm]{images/unilogo_wort}\\[1em]

        {\footnotesize
        %{\bfseries Universität Ulm} \textbar ~89069 Ulm \textbar ~Germany
        \hspace*{115mm}\parbox[t]{35mm}{\bfseries Fakultät für\\
        \fakultaet\\
        \mdseries \institut}\\[2cm]

        \parbox{140mm}{\bfseries \LARGE \titel}\\[2.5em]
        {\footnotesize Softwaregrundprojekt an der Universität Ulm}\\[3em]

        {\footnotesize \bfseries Vorgelegt von:}\\
        {\footnotesize \fullname\\}\\ [1em]
        {\footnotesize \bfseries Dozent:}\\
        {\footnotesize \dozent\\}\\[1em]
        {\footnotesize \bfseries Betreuer:}\\
        {\footnotesize \betreuer}\\ [1em]
        {\footnotesize \jahr}
        }
    \end{addmargin*}
    \pagebreak
    \tableofcontents
    \pagebreak

    \section{Verwendete Technologien}
    Der Server und die KI wurden in C++17 geschrieben.
Als Compiler wurde dafür der GNU-C++-Compiler (GCC) (Version 8.3, Lizenziert unter GPL 3) verwendet,
als Buildsystem wurde CMake (Version 3.10, Lizenziert unter BSD-3) genutzt.
Zur statischen Codeanalyse wird Clang-Tidy (Version 6.0, Lizenziert unter Apache 2) verwendet,
Unit Tests werden mit Google-Test bzw. Google-Mock (Version 2.56, Lizenziert unter BSD-3) implementiert.
Um undefiniertes Verhalten zu vermeiden wird das Programm während der Ausführung mit Adress Sanitizer (Version 5.0, Lizenziert unter Apache 2) überprüft.

Alle Teile des Quellcodes wurden mithilfe von Doxygen-Kommentaren (Version 1.8, Lizenziert unter GPL) dokumentiert.

Es wurde sich während des Entwicklungsprozess an den Linux-Kernel-Styleguide gehalten werden. Zwei wesentliche Abweichungen von diesem Styleguide sind eine Einrückungstiefe von 4 Zeichen, sowie die Platzierung der öffnenden Klammer bei Funktionsaufrufen in der selben Zeile (die Unterscheidung bei der platizierung von öffnenden geschweiften Klammern ist wohl ursprünglich K\&R zuzuschreiben, die Begründung hierfür ist, dass Funktionen sich nicht verschachteln lassen, dies
ist in C++ mit Lambdas nicht mehr so eindeutig der Fall).

Als Entwicklungsumgebung wurden die IDE CLion (Version 2019.1, proprietär) sowie der Editor vim (Version 8.0, Lizenziert als Careware).

Als externe Librarys wurde libwebsockets (Version 3.1.0, LGPL 2, \url{https://github.com/warmcat/libwebsockets}), 
nlohmann::json (Version 3.6.0 bzw. 3.6.1, MIT, \url{https://github.com/nlohmann/json}) sowie 
MLP (GPL 3, \url{https://github.com/aul12/MLP}) verwendet.

    
    \section{Projekttagebuch}
    Das Projekttagebuch wurde mit der Hilfe eines Google-Formular verwaltet. Die dazugehörige Eingabemaske ist unter folgendem Link zu finden: \url{https://docs.google.com/forms/d/e/1FAIpQLSdlc-WKKGjKdk1e3qPVxXkzdcqY6LYrXj0sgBuA4_gtZcW2Hw/viewform}\\
Die einzelnen Eintrage sind unter dem nachfolgenden Link einsehbar: \url{https://drive.google.com/open?id=1AJPGgD-9MZZR8e6xCiLpxgbUAKqJjAv-QcIBXes6I5M}\\
Aus diesen Einträgen wurden die folgenden Diagramme automatisch generiert:

\begin{figure}[H]
    \centering
    \includegraphics[width=\textwidth]{../Endabnahme/images/Verantwortlicher_Chart.PNG}
\end{figure}

\begin{figure}[H]
    \centering
    \includegraphics[width=\textwidth]{../Endabnahme/images/Typ_Chart.PNG}
\end{figure}

\begin{figure}[H]
    \centering
    \includegraphics[width=\textwidth]{../Endabnahme/images/Anwesend_Chart.PNG}
\end{figure}
    
    \section{UML - Komponentendiagramm}
    \subsection{Komponentendiagramm zum Server}
\begin{figure}[H]
    \centering
    \includegraphics[scale=0.85]{../Endabnahme/images/ServerDiagramme.pdf}
\end{figure}

\subsection{Komponentendiagramm zur KI}
\begin{figure}[H]
    \centering
    \includegraphics[scale=0.85]{../Endabnahme/images/KiDiagramme.pdf}
\end{figure}

\subsection{Komponentendiagramm zum Benutzer-Client}
\begin{figure}[H]
    \centering
    \includegraphics[scale=0.13]{../Meilenstein05/images/komponentendiagramm_benutzer-client.png}
\end{figure}

\subsection{Komponentendiagramm zum Level-Editor}
\begin{figure}[H]
    \centering
    \includegraphics[scale=0.6]{../Meilenstein05/images/Konfigurator.pdf}
\end{figure}
    
\end{document}
