\NeedsTeXFormat{LaTeX2e}
\documentclass[a4paper,12pt,
headsepline,           % Linie zw. Kopfzeile und Text
oneside,               % einseitig
pointlessnumbers,      % keine Punkte nach den letzten Ziffern in Überschriften
bibtotoc,              % LV im IV
%DIV=15,               % Satzspiegel auf 15er Raster, schmalere Ränder   
%BCOR15mm               % Bindekorrektur
%,draft
]{scrartcl}

\usepackage{amsmath}
\usepackage{amsfonts}
\usepackage{amssymb}
\usepackage{enumitem}
\usepackage[utf8]{inputenc} % this is needed for umlauts
\usepackage[ngerman]{babel} % this is needed for umlauts
\usepackage[T1]{fontenc} 
\usepackage{commath}
\usepackage{xcolor}
\usepackage{booktabs}
\usepackage{float}
\usepackage{tikz-timing}
\usepackage{tikz}
\usepackage{multirow}
\usepackage[final]{pdfpages}
\usepackage{blindtext}
\usepackage[scaled]{helvet}
\usepackage{hyperref}

\usetikzlibrary{calc,shapes.multipart,chains,arrows}

\KOMAoptions{DIV=last} % Neuberechnung Satzspiegel nach Laden von Paket helvet

\usepackage{scrpage2}
\pagestyle{useheadings}

\renewcommand{\familydefault}{\sfdefault} 

\setlength{\parindent}{0pt}   % kein linker Einzug der ersten Absatzzeile
\setlength{\parskip}{1.4ex plus 0.35ex minus 0.3ex} % Absatzabstand, leicht variabel

\newcommand{\fullname}{Gruppe 10}
\newcommand{\titel}{Softwaregrundprojekt Meilenstein 1}
\newcommand{\jahr}{2018}
\newcommand{\dozent}{Florian Ege}
\newcommand{\betreuer}{Stefanos Mytilineos}
\newcommand{\fakultaet}{Ingenieurwissenschaften, Informatik und\\Psychologie}
\newcommand{\institut}{Institut für Softwaretechnik und Programmiersprachen}

\pdfinfo{
    /Author (\fullname)
    /Title (\titel)
    /Producer     (pdfeTex 3.14159-1.30.6-2.2)
    /Keywords ()
}

\hypersetup{
    pdftitle=\titel,
    pdfauthor=\fullname,
    pdfsubject={Softwaregrundprojekt-Abgabe},
    pdfproducer={pdfeTex 3.14159-1.30.6-2.2},
    colorlinks=false,
    pdfborder=0 0 0	% keine Box um die Links!
}

% Trennungsregeln
\hyphenation{Sil-ben-trenn-ung}


\newcommand{\begriff}[7] {
    \begin{table}[H]
        \centering
        \begin{tabular}{|p{4.5cm}|p{10cm}|}
            %\hline
            %\toprule \\
            \hline
            \textbf{Begriff} & \textbf{#1} \\ \hline
            %\midrule \\
            \textbf{Beschreibung} & #2 \\ \hline
            %\midrule
            \textbf{Ist-ein} & #3 \\ \hline
            %\midrule
            \textbf{Kann-sein} & #4 \\ \hline
            %\midrule
            \textbf{Aspekt} & #5 \\ \hline
            %\midrule
            \textbf{Bemerkung} & #6 \\ \hline
            %\midrule
            \textbf{Beispiel} & #7 \\ %\hline
            %\bottomrule
            \hline
        \end{tabular}
    \end{table}
}

%\title{Softwaregrundprojekt}
%\author{Gruppe 10}

\newcommand{\anf}[7] {
    \begin{table}[H]
        \centering
        \begin{tabular}{|p{3.2cm}|p{9.8cm}|}
        	\hline
            \textbf{ID:} & \textbf{#1} \\ \hline
            \textbf{Titel:} & #2 \\ \hline
            \textbf{Beschreibung:} & #3 \\ \hline
            \textbf{Begründung:} & #4 \\ \hline
            \textbf{Abhängigkeiten:} & #5 \\ \hline
            \textbf{Priorität:} & #6 \\ \hline
            \textbf{Akteure:} & #7 \\ \hline
        \end{tabular}
    \end{table}
}

\newcounter{fanfCount}
\newcommand{\fanf}[6] {
    \stepcounter{fanfCount}
    \anf{FA\thefanfCount}{#1}{#2}{#3}{#4}{#5}{#6}
}
\newcounter{qanfCount}
\newcommand{\qanf}[6] {
    \stepcounter{qanfCount}
    \anf{QA\theqanfCount}{#1}{#2}{#3}{#4}{#5}{#6}
}

\newcommand{\akt}[4] {
    \begin{table}[H]
        \centering
        \begin{tabular}{|p{3cm}|p{10cm}|}
        	\hline
            \textbf{ID:} & \textbf{#1} \\ \hline
            \textbf{Titel:} & #2 \\ \hline
            \textbf{Beschreibung:} & #3 \\ \hline
            \textbf{Rolle:} & #4 \\ \hline
        \end{tabular}
    \end{table}
}

\newcounter{faktCount}
\newcommand{\fakt}[3] {
    \stepcounter{faktCount}
    \akt{AKT\thefaktCount}{#1}{#2}{#3}
}

\begin{document}
    \thispagestyle{empty}
    \begin{addmargin*}[4mm]{-10mm}

        \includegraphics[height=1.8cm]{images/unilogo_bild}
        \hfill
        \includegraphics[height=1.8cm]{images/unilogo_wort}\\[1em]

        {\footnotesize
        %{\bfseries Universität Ulm} \textbar ~89069 Ulm \textbar ~Germany
        \hspace*{115mm}\parbox[t]{35mm}{\bfseries Fakultät für\\
        \fakultaet\\
        \mdseries \institut}\\[2cm]

        \parbox{140mm}{\bfseries \LARGE \titel}\\[2.5em]
        {\footnotesize Softwaregrundprojekt an der Universität Ulm}\\[3em]

        {\footnotesize \bfseries Vorgelegt von:}\\
        {\footnotesize \fullname\\}\\ [1em]
        {\footnotesize \bfseries Dozent:}\\
        {\footnotesize \dozent\\}\\[1em]
        {\footnotesize \bfseries Betreuer:}\\
        {\footnotesize \betreuer}\\ [1em]
        {\footnotesize \jahr}
        }
    \end{addmargin*}
    \pagebreak
    \tableofcontents
    \pagebreak

    \section{Kontextanalyse}
    %TODO
    
    \section{Fachwissen}
    \begriff{Nutzer}
{Ein Mensch, der einen Rechner bedient und entweder den Client zum Spielen des Spiels oder zur Beobachtung einer Partie benutzt, oder den Team-Editor bedient. Jeder Benutzer hat einen Nutzernamen, mittels dem er von anderen Nutzern erkannt werden kann.}
{-}
{Spieler, Gast}
{Zur Beschreibung des Programmverlaufs}
{-}
{JägerMaister69}

\begriff{Spieler}
{Ein Nutzer, der das Computerspiel \glqq Fatastic Feasts\grqq spielt.}
{Nutzer}
{-}
{Zur Beschreibung des Programmverlaufs}
{-}
{-}

\begriff{Gast}
{Ein Nutzer, der eine laufende Partie beobachtet}
{Nutzer}
{-}
{Zur Beschreibung des Programmverlaufs}
{-}
{-}

\begriff{Client}
{Das Computerprogramm, das mit einer grafischen Oberfläche ausgestattet ist und einem Nutzer erlaubt, eine Verbindung mit einem Server herzustellen und damit zu Kommunizieren}
{-}
{-}
{Zum spielen des Spiels \glqq Fantastic Feasts\grqq}
{Der Begriff bezieht sich nicht auf den Menschen, der das Programm bedient}
{-}

\begriff{Server}
{Die zentrale Komponente, in dem die Spiellogik implementiert ist und die Programmbefehle abwickelt und mit dem sich Clients verbinden können, um eine Partie zu spielen oder zu beobachten. Die Kommunikation erfolgt mit JSON}
{-}
{-}
{Ist für die Kommunikation von Clients, für das Verwalten des Spielgeschehens, Ressourcenverwaltung und die Spiellogik verantwortlich.}
{-}
{-}

\begriff{Team-Editor}
{Ermöglicht einem Nutzer mit einer grafischen Oberfläche, ein eigenes Team zu erstellen und zu bearbeiten. Die Einstellungen werden danach als JSON-Datei gespeichert.}
{-}
{-}
{Zur Erstellung von Nutzereigenen Teams.}
{-}
{-}

\begriff{KI-Client}
{Meldet sich beim Server wie ein normaler Client an und simuliert mit einer KI einen menschlichen Spieler. Hat keine grafische Oberfläche. Meldet sich mit dem Nutzernamen \glqq KI\grqq ein.}
{-}
{-}
{Zum Spielen gegen einen Computergegner}
{-}
{-}

\begriff{KI}
{Definiert die Regeln, nach denen der KI-Client auf die durch den Server vermittelten Geschehen im Spiel reagiert.}
{-}
{-}
{Zum Spielen gegen einen Computergegner}
{Die KI ist die Logik, nach der der Computer das Spiel spielt und kein Programm}
{-}

\begriff{Spielfeld}
{Ein grafisch darstellbares Raster, auf dem sich die Spielfiguren bewegen}
{-}
{-}
{Dient als virtuelles Spielbrett mit klar definierten Abgrenzungen}
{Wird nicht Spielumgebung genannt um Verwechslung mit dem Client zu vermeiden}
{-}

\begriff{Zelle}
{Die kleinste Einheit des Spielfeldes, also ein Quadrat davon}
{-}
{Zentrumszelle, Torring, Hüterzonenzelle}
{Mögliche Standorte der Spielfiguren}
{Wird nicht Feld genannt, da das ein eher vager Begriff ist}
{-}

\begriff{Zentrum}
{Der 3x3 Zellen große Abschnitt in der Mitte des Spielfeldes}
{-}
{-}
{Summe aller Zentrumszellen}
{Ist das Mittelfeld im Lastenheft}
{-}

\begriff{Hüterzone}
{Die Bereiche am linken und rechten Rand des Spielfeldes, in dem sich die Torringe befinden}
{-}
{-}
{Summe aller kritischen Zellen und Torring}
{-}
{-}

\begriff{Torring}
{Die Zellen in die beide Teams die Payload bewegen wollen. Es wird zwischen eigenen und gegnerischen Torringen unterschieden.}
{Zelle}
{Eigener Torring, Gegnerischer Torring}
{Hauptquelle von Punkten}
{Torring im Lastenheft}
{}

\begriff{Zentrumszelle}
{Eine Zelle im Zentrum des Spielfeldes (siehe Zentrum)}
{Zelle}
{-}
{Startpunkt für Payload und Quälgeister}
{-}
{-}

\begriff{Hüterzonenzelle}
{Eine Zelle in einem kritischen Bereich des Spielfeldes}
{Zelle}
{-}
{limitierendes Element für das Abliefern der Payload}
{-}
{-}

\begriff{Spielobjekt}
{Jedes Objekt, das sich auf dem Spielfeld befindet und darauf bewegt werden kann}
{-}
{Ball, Spielfigur}
{-}
{Nicht zu verwechseln mit Spielfigur}
{-}

\begriff{Ball}
{Ein Spielobjekt, das nicht direkt, nur indirekt von einem Spieler beeinflusst werden kann}
{Spielobjekt}
{Payload, Quälgeist, Schatz}
{Festpunkte zur Steuerung des Spielverlaufs}
{-}
{-}

\begriff{Spielfigur}
{Ein Spielobjekt, das von einem Spieler direkt gesteuert wird. Jede Spielfigur hat einen Namen, einen Rang und eine Rolle. Man unterscheidet außerdem zwischen eigenen und gegnerischen Spielfiguren.}
{Spielobjekt}
{Hüter, Sucher, Angreifer, Treiber}
{Mitglieder eines Teams}
{Spieler im Lastenheft}
{Luke Skywalker, eigener Hüter, Rang 5}

\begriff{Payload}
{Passives Objekt, mit dem Angreifer und Hüter interagieren können und von ihnen nach Möglichkeit in ein gegnerisches Zielfeld befördert werden soll.}
{Ball}
{-}
{Zentrales Spielobjekt}
{\glqq Quaffel\grqq im Lastenheft}
{-}

\begriff{Quälgeist}
{Ball, der sich von selbst auf Spielfiguren zubewegt, die keine Treiber sind und diese betäuben können und von Treibern bewegt werden können.}
{Ball}
{-}
{Zusätzliches taktisches Spielelement}
{\glqq Klatscher\grqq im Lastenheft}
{-}

\begriff{Schatz}
{Ball, der von den Suchern gejagt wird und deren Fund die Partie beendet}
{Ball}
{-}
{Definiert Spielende}
{\glqq Schnatz\grqq im Lastenheft}
{-}

\begriff{Partie}
{Ein einzelnes Spiel. Beginnt beim Platzieren der Figuren und endet mit dem Bestimmen des Gewinners.}
{-}
{-}
{Beschreibung des Spielablaufs}
{-}
{VodkaVodka98 spielt gegen LongEiländ}

\begriff{Hüter}
{Spielfigur, deren Aufgabe es ist, die Payload von den eigenen Zielfeldern fernzuhalten}
{Spielfigur}
{Eigener Hüter, Gegnerischer Hüter}
{Letzte Verteidigungslinie}
{-}
{Siehe \glqqSpielfigur\grqq}

\begriff{Sucher}
{Spielfigur, die den Schatz jagt}
{Spielfigur}
{Eigener Sucher, Gegnerischer Sucher}
{Beendet die Partie}
{\glqqJäger\grqq wäre ein besserer Begriff, könnte aber mit den Begriffen im Lastenheft zu Verwirrungen führen.}
{Darth Vader, gegnerischer Sucher, Rang 2}

\begriff{Angreifer}
{Spielfigur, die die Payload in ein einen gegnerischen Torring befördern soll}
{Spielfigur}
{Eigener Angreifer, Gegnerischer Angreifer}
{Holt Punkte für das eigene Team}
{\glqqJäger\grqq im Lastenheft. Angreifer beschreibt die Rolle der Spielfigur aber besser.}
{Han Solo, eigener Angreifer, Rang 3}

\begriff{Treiber}
{Spielfigur, mit der der Spieler eigene Spielfiguren vor Quälgeistern schützt und gegnerische damit abschießen kann}
{Spielfigur}
{Eigener Treiber, Gegnerischer Treiber}
{Interagiert mit Quälgeistern}
{\glqqTreiber\grqq im Lastenheft}
{Boba Fett, gegnerischer Treiber, Rang 4}

\begriff{Punkte}
{Der Spieler mit mehr Punkten am Ende der Partie gewinnt. Werden durch das Platzieren der Payload in einem gegnerischen Torring oder das Finden des Schatzes erhalten.}
{-}
{-}
{Bestimmung des Gewinners}
{-}
{SchnapsNase hat 20 Punkte}

\begriff{Besetzen}
{Eine Spielfigur besetzt das Feld, auf dem sie sich befindet}
{-}
{-}
{Beschreibung des Spielgeschehens}
{Zwei Spielfiguren können sich nicht auf derselben Zelle befinden}
{Chewbacca besetzt Zelle 5:3}

\begriff{Rang}
{Jede Spielfigur hat einen Rang von 1 bis 5, der die Wahrscheinlichkeit bestimmt, dass sie noch einmal ziehen kann. Rang 1 ist der beste.}
{-}
{-}
{Unterscheidet Qualität der Spielfiguren.}
{Ersetzt die \glqqBesen\grqq aus dem Lastenheft.}
{Yoda hat Rang 1.}

\begriff{Aktion}
{Jede durch einen Spieler hervorgerufene Änderung der Spielsituation}
{-}
{Ziehen, Schießen, Schlagen, Einmischung, Übernahme}
{Weiterführung der Partie}
{-}
{Obi-Wan Kenobi zieht von Zelle 8:7 auf Zelle 9:7}

\begriff{Ziehen}
{Die Bewegung einer Spielfigur von einer Zelle auf eine andere durch direkten Befehl des Spielers}
{Aktion}
{-}
{Beschreibung des Spielverlaufs}
{Bezieht sich nicht auf erzwungene Bewegungen einer Spielfigur.}
{Obiwan Kenobi zieht von Zelle 8:7 auf Zelle 9:7}

\begriff{Befördern}
{Bewegen den Payload mittels einer Spielfigur}
{-}
{-}
{Bewegen der Payload, allgemeiner Begriff}
{Keine Aktion, da eventuell eine passive Folge, z.B. durch Ziehen}
{-}

\begriff{Schießen}
{Die Bewegung der Payload durch einen Hüter oder Angreifer auf eine andere, entfernte Zelle ohne Bewegung der Spielfigur}
{Aktion}
{-}
{Bewegung der Payload um mehrere Felder}
{\glqqWerfen\grqq im Lastenheft. Analog zum Schussvektor benannt.}
{Mace Windu schießt die Payload auf Zelle 10:4}

\begriff{Schlagen}
{Die erzwungene Bewegung eines Quälgeistes durch einen Treiber}
{Aktion}
{-}
{Interaktion mit Quälgeistern}
{\glqqKloppen\grqq im Lastenheft}
{R2-D2 schlägt einen Quälgeist auf Zelle 5:10}

\begriff{Einmischung}
{Hilfsfähigkeiten, die nicht von Spielobjekten ausgehen. Werden von einem Spieler gesteuert. Bei jeder Benutzung besteht eine Chance, dass die verwendete Einmischung bis zum Ende der Partie für den jeweiligen Spieler von Schutzmaßnahmen deaktiviert werden.}
{Aktion}
{Teleportation, Fernangriff, Impuls, Schatzjagd}
{Zusätzliche taktische Element}
{Ersetzt die \glqqFans\grqq aus dem Lastenheft}
{Lando Calrissian wird auf Zelle 6:6 teleportiert}

\begriff{Teleportation}
{Einmischung, die eine Spielfigur auf eine zufällige Zelle teleportiert}
{Einmischung}
{-}
{-}
{Ersetzt \glqqElfen\grqq aus Lastenheft}
{Siehe \glqqEinmischungen\grqq}

\begriff{Fernangriff}
{Trifft eine gegnerische Spielfigur. Ziel verliert gegebenenfalls die Payload und wird auf eine zufällige benachbarte, freie Zelle bewegt.}
{Einmischung}
{-}
{-}
{Statt \glqqKobolde\grqq im Lastenheft}
{Jango Fett wird von Fernangriff auf Zelle 5:6 gestoßen}

\begriff{Impuls}
{Wenn eine Spielfigur die Payload hält, wird sie bei Benutzung verloren}
{Einmischung}
{-}
{-}
{Statt \glqqTrolle\grqq im Lastenheft}
{C-3PO verliert wegen eines Impuls die Payload}

\begriff{Schatzstoß}
{Bewegt den Schatz zufällig um ein Feld}
{Einmischung}
{-}
{-}
{\glqqSchnatzschnappen\grqq im Lastenheft}
{Ein Schatzstoß treibt den Schatz auf Zelle 4:12}

\begriff{Entfernung}
{Eine Entfernung zwischen zwei Zellen ist die minimale Anzahl von Zügen, in denen eine Spielfigur von der einen auf die andere ziehen kann.}
{-}
{-}
{Spielfeldgeometrie}
{-}
{-}

\begriff{Schussvektor}
{Pfeil vom Mittelpunkt einer Zelle zum Mittelpunkt einer anderen}
{Torschussvektor}
{-}
{Spielfeldgeometrie}
{-}
{-}

\begriff{Torschussvektor}
{Schussvektor zu einem Schuss, der möglicherweise in einem Torschuss resultiert.}
{Schussvektor}
{-}
{Punkte sammeln}
{-}
{-}

\begriff{Torschuss}
{Ein Angreifer schießt die Payload in einen Torring und holt damit Punkte für sei Team}
{-}
{-}
{Punkte sammeln}
{Nur erfolgreiche Schüsse auf das Tor werden als Torschüsse bezeichnet.}
{Darth Sidious schießt die Payload in ein eigenes Tor}

\begriff{Rundenphase}
{Phase, in der eine Spielfigur Aktionen durchführt}
{-}
{-}
{Zeiteinteilung}
{}
{Leia Organa ist dran}

\begriff{Zug}
{Von der ersten Aktion eines Spielers bis zur ersten Aktion des Gegners}
{-}
{-}
{Zeiteinteilung}
{Nicht die Rundenphase einer Spielfigur}
{Bierdurst69 ist am Zug}

\begriff{Verlieren}
{Die Payload wird auf eine zufällige Zelle bewegt, die an die Zelle angrenzt, auf der sich die Spielfigur, die sie derzeit hält befindet.}
{-}
{-}
{Spielablauf}
{\glqqVertändeln\grqq im Lastenheft}
{Jar Jar verliert die Payload}

\begriff{Halten}
{Ein Angreifer oder Hüter kann die Payload halten. Ist das der Fall, bewegt sich die Payload auf die Zelle, auf die die Spielfigur zieht.}
{-}
{-}
{Beschreibung des Spielgeschehens}
{-}
{-}

\begriff{Übernahme}
{Ein Angreifer neben einer gegnerischen Spielfigur, die die Payload hält, kann diesen mit einer bestimmten Wahrscheinlichkeit übernehmen und hält sie anschließend selbst.}
{Aktion}
{-}
{Aggressives Spielmanöver}
{-}
{Darth Vader übernimmt die Payload von Anakin Skywalker}

\begriff{Betäubt}
{Eine betäubte Spielfigur kann in seiner nächsten Rundenphase keine Aktion durchführen}
{-}
{-}
{Wirkung der Quälgeister}
{\glqqAusgeknockt\grqq im Lastenheft}
{Jango Fett ist betäubt}

\begriff{Riskante Strategie}
{Handlung, wegen der eine Spielfigur vorübergehend vom Spielfeld entfernt werden kann.}
{-}
{Torring Blockieren, Stürmen, Großoffensive, Rammen, Schatz Blockieren}
{Taktische Elemente}
{\glqqFaul\grqq im Lastenheft}
{Qui-Gon Jinn blockiert den Schatz}

\begriff{Torring Blockieren}
{Eine eigene Spielfigur besetzt einen eigenen Torring, was verhindert, dass die Payload dort abgeliefert wird.}
{Riskante Strategie}
{-}
{Taktik}
{\glqqFlackern\grqq im Pflichtenheft}
{-}

\begriff{Stürmen}
{Ein Angreifer, der die Payload hält, zieht auf einen gegnerischen Torring, was das Abliefern garantiert.}
{Riskante Strategie}
{-}
{Taktik}
{\glqqNachtarocken\grqq im Lastenheft}
{Han Solo stürmt mittleren gegnerischen Torring}

\begriff{Großoffensive}
{Ein eigener Angreifer betritt eine gegnerische Hüterzonenzelle während ein anderer eigener Angreifer sich auf einer anderen befindet.}
{Riskante Strategie}
{-}
{Taktik}
{\glqqStutschen\grqq im Lastenheft}
{Lando Calrissia schließt sich Chewbacca in einer Großoffensive an}

\begriff{Rammen}
{Eine eigene Spielfigur zieht auf eine Zelle, die von einer gegnerischen Spielfigur besetzt wird. Dadurch wird die gegnerische Spielfigur auf eine benachbarte Zelle bewegt und verliert die Payload}
{Riskante Strategie}
{-}
{Taktik}
{\glqqKeilen\grqq im Lastenheft}
{Boba Fett rammt Jar Jar}

\begriff{Schatz blockieren}
{Eine Spielfigur, die kein Sucher ist, besetzt die Zelle, auf der sich der Schatz befindet.}
{Riskante Strategie}
{-}
{Taktik}
{\glqqSchnatzeln\grqq im Lastenheft}
{Darth Maul blockiert den Schatz}

\begriff{Schutzmaßnahmen}
{Entfernt mit bestimmter Wahrscheinlichkeit eine Spielfigur, die eine riskante Strategie ausführt vom Spielfeld bis eine Payload abgeliefert wird und deaktiviert permanent eine Einmischung für den Rest der Partie.}
{-}
{-}
{Taktik}
{\glqqSchiedsrichter\grqq im Lastenheft}
{Sheev Palpatine wurde von Schutzmaßnahmen vom Spielfeld entfernt}

\begriff{Disqualifikation}
{Tritt ein wenn fünf Spielfiguren eines Spielers gleichzeitig durch Schutzmaßnahmen aus dem Spiel entfernt sind. Führt zur Niederlage des Spielers.}
{-}
{-}
{Erhöhtes Risiko}
{-}
{CubaLibre wurde disqualifiziert. CaptainCola gewinnt die Partie.}
    
    \includepdf[pagecommand={\section{Domänenmodell}}]{domaenenmodell.pdf}
    \section{Anforderungsdefinition}    
    
    \subsection{Akteure}
    
    \fakt	{Nutzer}
			{Menschlicher Nutzer, der eine Anwendungen bedient.}
			{Ein Mensch, welcher entweder als Spieler aktiv an einem Spiel teilnimmt oder als Gas passiv einem Spiel zusieht oder den Quidditchteam-Editor benutzt um ein Team zu erstellen oder an zu passen.}
			
	\fakt	{Spieler}
			{Nimmt aktiv Einfluss auf das Spielgeschehen.}
			{Entweder Nutzer oder KI. Nimmt aktiv Einfluss auf das Spielgeschehen indem er Züge vorgibt, wenn er an der Reihe ist. Muss vor Spielbeginn eine Konfiguration für sein Team angeben.}
			
	\fakt	{Gast}
			{Beobachtet eine Partie als Außenstehender.}
			{Nutzer, der mit der Client-Anwendung ein laufendes Spiel beobachtet, jedoch keinen Einfluss auf das Spielgeschehen nehmen kann.}
		
	\fakt	{Systemadministrator}
			{Anwendungsnutzer mit Zusatzqualifikation um Server zu verwalten.}
			{Der Systemadministrator hat zugriff auf die Server-Anwendung. Er ist dafür verantwortlich, eine Instanz der Server Anwendung zu starten und zu betreuen. Zudem hat er Zugriff auf die Partie-Konfiguration und kann diese bei Bedarf verändern.}
	
	\fakt	{KI}
			{Vom Computer gesteuerter Spieler.}
			{Spieler, dessen Entscheidungen und Züge auf einem Computer von der KI-Client Software getroffen werden.}
	
	\fakt	{Kunde}
			{SoPra-Tutor}
			{Der Kunde kann Anforderungen und zusätzliche Wünsche an das Produkt stellen. Er entscheidet schlussendlich auch darüber, ob das Endprodukt den Anforderungen genügt.}
	
	\fakt	{Client}
			{Software eines Nutzers}
			{Mit Hilfe der Client Software kann der Nutzer entweder als Spieler oder als Gast einem von einem Server zur Verfügung gestellten Spiel beitreten.}
	
	\fakt	{KI-Client}
			{Anwendung, welche die KI steuert.}
			{Der KI-Client ist die Anwendung, welche eine KI steuert. Der Nutzer kann von Außen keinen Einfluss auf die Züge und Entscheidungen des KI-Clients nehmen.}
	
	\fakt	{Server}
			{Zentraler Computer mit spezieller Software zu dem alle am Spiel beteiligten Clients ein Verbindung aufbauen.}
			{Auf dem Server läuft die eigentliche Spiellogik. Er fungiert dabei als Bindeglied zwischen den am Spiel beteiligten Clients und stellt für diese alle benötigten Informationen, wie etwas die Spielfeldkonfiguration oder die Züge des Gegners bereit.}
			
	\fakt	{Quidditchteam-Editor}
			{Editor für die Team-Konfiguration.}
			{Mit dieser Anwendung kann ein Nutzer sein für eine Partie gewünschtes Quiddichteam im Rahmen bestimmter Grenzen beliebig konfigurieren.}
	
	\fakt 	{Team}
			{Einheit aus Spielfiguren die einem spielenden Nutzer zugeordnet ist.}
			{Im Spiel ist jedem Spieler ein Team aus 7 Spielfiguren zugeordnet. Die Zusammensetzung des Team lässt sich mit der Quidditchteam-Editor den persönlichen Wünschen entsprechend anpassen.}
	
	\fakt 	{Schiedsrichter}
			{Ahndet Fouls.}
			{Der Schiedsrichter wird durch eine Software Repräsentiert die mit Hilfe eines Zufallsgenerator entscheidet, ob ein Foul geahndet wird oder nicht.}
	
	\fakt	{Spiel}
			{Eine Partie Fantastic Feasts.}
			{Ein Spiel besteht aus genau 2 Teams, Fans, dem Schiedsrichter und einem Spielfeld. Das Steuern der Spiellogik übernimmt der Server. Die Spieler der beiden Teams spielen gegeneinander um den Sieg.}	
	
	\subsection{Allgemeine funktionale Anforderungen}
	
	\fanf	{Quidditch-Spielfeld}
			{Das Quidditch-Spielfeld ist eine Ovale Form, welche in ein Raster von 17x13 quadratischen Feldern eingepasst ist. Auf diesem Feld finden alle Spielhandlungen statt, welche während dem Spiel getätigt werden können. Auf dem Spielfeld gibt es noch ein Mittelkreis und an den jeweils gegenüberliegenden Enden noch Hüterzonen.}
			{Diese Anforderung geht aus den im Lastenheft zu Verfügung gestellten Spielregeln für das Spiel \textit{Fantastic Feasts} hervor.}
			{<Abhängigkeiten>}
			{++}
			{\textbf{ALLE?}}
	
	\fanf	{Mittelkreis}
			{Der Mittelkreis ist ein Bereich auf dem Quidditch-Spielfeld, welcher in der Mitte angeordnet ist und aus 3x3 quadratischen Kacheln besteht. In dem Mittelkreis befindet sich das Mittelfeld, welches die mittlere Kachel des Mittelkreises ist und durch ein \textit{M} gekennzeichnet ist.}
			{Diese Anforderung geht aus den im Lastenheft zu Verfügung gestellten Spielregeln für das Spiel \textit{Fantastic Feasts} hervor.}
			{FA1}
			{++}
			{<Akteure>}
	
	\fanf	{Mittelfeld}
			{Das Mittelfeld stellt den mittleren Punkt des Mittelkreises dar, welcher im Zentrum des Quidditch-Spielfeldes ist.}
			{Diese Anforderung geht aus den im Lastenheft zu Verfügung gestellten Spielregeln für das Spiel \textit{Fantastic Feasts} hervor.}
			{FA2}
			{++}
			{<Akteure>}
	
	\fanf	{Hüterzone}
			{Die Hüterzonen sind jeweils an den jeweils gegenüberliegenden Seiten des Quidditch-Spielfeldes. Die Hüterzonen beinhalten jeweils 3 Torringe und sind somit die Zonen in den die Teams Punkten können. Die Hüterzonen sind in einer Ovalen Form, welche 11 quadratische Felder hoch ist und 5 quadratische Felder breit ist an den jeweilgen Maximalen Punkten.}
			{Diese Anforderung geht aus den im Lastenheft zu Verfügung gestellten Spielregeln für das Spiel \textit{Fantastic Feasts} hervor. Außerdem beinhaltet diese Zone die Torringe, durch welche die Teams Punkte machen können.}
			{FA1}
			{++}
			{<Akteure>}
	
	\fanf	{Torring}
			{Die Torringe sind in der Hüterzone angebracht und dazu da, dass die Teams jeweils Punkten können. Die 3 torringe werden jeweils von einem Torhüter bewacht, welcher verhindern kann, dass das gegnerische Team ein Tor schiessen kann.}
			{Diese Anforderung geht aus den im Lastenheft zu Verfügung gestellten Spielregeln für das Spiel \textit{Fantastic Feasts} hervor.}
			{FA4}
			{++}
			{<Akteure>}
			
	\fanf	{Schussvektor}
			{Als Schussvektor wird ein möglicher Wurf bezeichnet, um einen Torring zu treffen. Der Schussvektor geht dabei vom Mittelpunkt des Rasterfeldes, von dem geworfen wird zu dem Feld, zu dem geworfen wird gerade. Alle Rasterfelder, die dabei geschnitten werden, sind so genannte überstrichene Felder. Dies gilt aber nur für Felder, die wirklich geschnitten wurden und nicht nur an einer Ecke tangiert wurden.
			Um einen möglichen Torschuss zu erzielen, muss der Torschussvektor mit einem Winkel unter 90 Grad auf das Feld eintreffen, andernfalls ist der Torschuss nicht möglich.}
			{Diese Anforderung geht aus den im Lastenheft zu Verfügung gestellten Spielregeln für das Spiel \textit{Fantastic Feasts} hervor.}
			{FA4}
			{++}
			{<Akteure>}	
	
	\fanf	{Entfernung}
			{Die Entferunung ist die kleinst mögliche Anzahl an Zügen, die man braucht, um Feld A zu Feld B zu kommen, wobei man sich in alle Richtengen bewegen darf, also Vertikal, Horizontal und Diagonal. Dies entspricht allerdings nicht immer den überstrichenen Feldern -1!.}
			{Die Entfernung ist wichtig für das Spiel, um angeben zu können, welcher Wurf am besten geeignet ist und welcher der ganzen möglichen Schussvektoren am idealsten ist von der Entfernung.}
			{<Abhängigkeiten>}
			{++}
			{<Akteure>}
	
	\fanf	{Bälle}
			{Es gibt 3 verschiedene Arten von Bällen. So existieren der Quaffel, der Klatscher und der Schnatz. Jeder Ball hat seine eigenen spezifischen Fähigkeiten und eigene Funktion, die er für das Spiel erfordert. In einem Quidditch Spiel kommen 4 Bälle vor. Davon ist einer ein Quaffel, einer ein Schnatz und zwei sind Klatscher.}
			{Diese Anforderung geht aus den im Lastenheft zu Verfügung gestellten Spielregeln für das Spiel \textit{Fantastic Feasts} hervor.}
			{FA1}
			{++}
			{<Akteure>}
	
	\fanf	{Quaffel [Ball]}
			{Der Quaffel ist ein roter Lederball, welcher dazu dient, dass das TEam punkten kann. Dies geschieht in dem der Quaffel von den Jägern durch einen der 3 Torringe geschmießen wird. Dies gibt dann 10 Punkte für das Team. Da der Quaffel selber keine Flugfähigkeiten oder andere magische Fähigkeiten hat, muss er immer von einem Jäger mitgetragen werden oder geworfen werden oder von einem Hüter gefangen werden. ein Jäger oder Hüter kann den Quaffel aufnehmen, in dem der Ball er auf das Feld zieht, auf dem der Quaffel liegt, oder wenn der Quaffel auf einem Feld zum liegen kommt, auf dem ein Jäger oder Hüter steht. Zieht ein Treiber oder Sucher auf das Feld des Quaffels oder konmmt der Quaffel auf einem der Felder zum liegen, so hopst der Quaffel auf ein zufälliges freies Nachbarfeld.}
			{Diese Anforderung geht aus den im Lastenheft zu Verfügung gestellten Spielregeln für das Spiel \textit{Fantastic Feasts} hervor.
			Der Quaffel wird benötigt, dass das Team Punkte machen kann.}
			{FA8}
			{++}
			{<Akteure>}
	
	\fanf	{Klatscher [Ball]}
			{Bei dem Spiel Quidditch gibt es 2 Klatscher, dies sind kleine schwarze Bälle, welcher magische Eigenschaften haben, was in dem Fall bedeutet, dass sie selber fliegen können. Das Ziel von den Klatschern ist es, die Spieler beider Teams von den Besen zu schmeißen. Dies geschieht, in dem der Klatscher sich jedes mal wenn er in der Rundenphase dran ist, einen zufälligen Spieler aussucht, welcher kein Treiber ist. Dann bewegt sich der Klatscher ein Feld auf den Spieler zu. Schafft der Klatscher es auf das selbe Feld wie der SPieler zu kommen, so wird dieser Spieler mit einer gewissen Wahrscheinkichkeit ausgenockt.}
			{Diese Anforderung geht aus den im Lastenheft zu Verfügung gestellten Spielregeln für das Spiel \textit{Fantastic Feasts} hervor.
			DIe Klatscher sind dazu da um dem Spiel einen Witz zu geben, dass manche Spieler zeitweise nicht verfügbar sind.}
			{FA8}
			{++}
			{<Akteure>}
	
	\fanf	{Goldener Schnatz [Ball]}
			{Der goldene Schnatz ist ein kleiner goldener Ball, welcher von dem Sucher gesucht werden muss. Der goldene Schnatz ist ebenfalls ein Ball, welcher eigenständig herumfliegt, wodurch er versucht zu vermeiden von den Suchern gefangen zu werden. Wird der Schnatz jedoch von einem der beiden Suchern gefangen, so gibt dieser 30 Punkte für das Team des Suchers, welcher den Schnatz gefangen hat und das Spiel ist beendet. In jeder Rundenphase sucht der Schnatz sich ein neues Feld aus, auf dass er sich dann eine Position weiter bewegt. Dies geschieht, in dem er den Sucher, der die geringste Entfernung hat nur berücksichtigt und dann unter allen Feldern, die eine größere Entfernung als seine aktuelle Position haben, ein zufälliges Feld weitergeht, so dass er sich von dem Sucher entfernt.}
			{Diese Anforderung geht aus den im Lastenheft zu Verfügung gestellten Spielregeln für das Spiel \textit{Fantastic Feasts} hervor.
			Der Schnatz verleiht dem Spiel eine besonderheit, so dass ein Team durch das fangen des Schnatzes die Chance hat, evtl doch noch zu gewinnen. Außerdem ist somit das Spiel zu ende.}
			{FA8}
			{++}
			{<Akteure>}
	
	\fanf	{Besen}
			{Die Besen sind essentiell für das Spiel, da alle 7 Spieler auf den Besen fliegen müssen. Es gibt im Fall von \textit{Fantastic Feasts}  verschiedene Besenmodelle, welche jeweils unterschiedliche Eigenschaften haben. Dies zeichnet sich dadurch aus, dass die Besen unterschiedliche Wahrscheinlichkeiten haben, mit denen der Spieler des Besens noch ein zweites Feld sich weiter bewegen darf. Dies bedeutet, je besser der Besen ist, desto wahrscheinlicher ist es, dass der Spieler ein zweites Feld weiter darf. Zusätzlich gilt die Besen-Repräsentanz-Regel, welche besagt, dass in jedem Team jeder Typ eines Besens einmal vorkommen muss.}
			{Diese Anforderung geht aus den im Lastenheft zu Verfügung gestellten Spielregeln für das Spiel \textit{Fantastic Feasts} hervor.
			Die Besen sind wichtig, da die Spieler sich damit fort bewegen und die Besen auch noch Eigenschaften mitbringen, welche für das Spiel wichtig sind.}
			{FA1}
			{++}
			{<Akteure>}

	\fanf	{Zauberfauch [Besen]}
			{Der Zauberfauch ist der schlechteste Besen den ein Spieler auswählen kann. Die Wahrscheinlichkeit, dass der Spieler ein zweites Feld in einem Zug vorrücken kann beträgt bei dem Besen 10\%.}
			{Der Besen ist ein wichtiges Feature, welches Einfluss auf das Spiel hat, da je nach Besen der Spieler einen zweiten Zug machen darf.}
			{<Abhängigkeiten>}
			{<Prio>}
			{<Akteure>}

	\fanf	{Sauberwisch 11 [Besen]}
			{Der Sauberwisch 11 ist der zweit schlechteste Besen den ein Spieler auswählen kann. Die Wahrscheinlichkeit, dass der Spieler ein zweites Feld in einem Zug vorrücken kann beträgt bei dem Besen 30\%.}
			{Der Besen ist ein wichtiges Feature, welches Einfluss auf das Spiel hat, da je nach Besen der Spieler einen zweiten Zug machen darf.}
			{<Abhängigkeiten>}
			{<Prio>}
			{<Akteure>}

	\fanf	{Komet 2-60 [Besen]}
			{Der Komet 2-60 ist der mittel beste Besen den ein Spieler auswählen kann. Die Wahrscheinlichkeit, dass der Spieler ein zweites Feld in einem Zug vorrücken kann beträgt bei dem Besen 50\%.}
			{Der Besen ist ein wichtiges Feature, welches Einfluss auf das Spiel hat, da je nach Besen der Spieler einen zweiten Zug machen darf.}
			{<Abhängigkeiten>}
			{<Prio>}
			{<Akteure>}

	\fanf	{Nimbus 2001  [Besen]}
			{DerNimbus 2001 ist der zweit beste Besen den ein Spieler auswählen kann. Die Wahrscheinlichkeit, dass der Spieler ein zweites Feld in einem Zug vorrücken kann beträgt bei dem Besen 70\%.}
			{Der Besen ist ein wichtiges Feature, welches Einfluss auf das Spiel hat, da je nach Besen der Spieler einen zweiten Zug machen darf.}
			{<Abhängigkeiten>}
			{<Prio>}
			{<Akteure>}

	\fanf	{Feuerblitz [Besen]}
			{Der Feuerblitz ist der beste Besen den ein Spieler auswählen kann. Die Wahrscheinlichkeit, dass der Spieler ein zweites Feld in einem Zug vorrücken kann beträgt bei dem Besen 90\%.}
			{Der Besen ist ein wichtiges Feature, welches Einfluss auf das Spiel hat, da je nach Besen der Spieler einen zweiten Zug machen darf.}
			{<Abhängigkeiten>}
			{<Prio>}
			{<Akteure>}
			
	\fanf	{Teams}
			{An einem Spiel sind 2 Teams beteiligt, die gegeneinander spielen. Diese beiden Teams haben einen Namen, ein Team-Motto, ein Logo und eine Hauptfarbe für das Trickot, aber auch eine Ersatzfarbe, falls das andere Team eine ähnliche Farbe hat, gegen das gespielt wird. Jedes Team besteht zudem aus 7 Spieler, wovon einer ein Sucher ist, einer ein Hüter, 2 sind sogenannte Treiber und die restlichen 3 sind Jäger.}
			{Die Teams sind notwendig, da sie der Hauptbestandteil sind, in dem sich alles vereint.}
			{FA1, FA19}
			{++}
			{<Akteure>}
	
	\fanf	{Spieler}
			{Es gibt 4 Arten von Spielern, welche in einem Quidditch Spiel unterschiedliche Aufgaben haben. Ein Team besteht insgesamt aus 7 Spielern, wovon es einen Sucher gibt, einen Hüter, 2 Treiber und 3 Jäger. Jeder dieser Spieler besitzt einen Namen und ein Geschlecht. Jedes Team muss dabei ausgeglichen sein, was das Geschlecht angeht, womit immer nur 4 Spieler dem selben Geschlecht angehören dürfen. In der Rundenphase, machen die Spieler zuerst ihre Bewegung, und dann ihre Aktion. Dies bedeutet, dass sie sich erst ein Feld weiterbewegen, und anschließend je nach Wahrscheinlichkeit des Besens ein zweites Feld weiterbewegen, und danach dann erst ihre Aktion wie Quaffel werfen oder Klatscher klopfen durchführen.}
			{Spieler sind absolut notwendig für das Spiel, da sie die Aktionen durchführen, welche von dem Menschen gesteuert werden.}
			{FA1}
			{++}
			{<Akteure>}
	
	\fanf	{Jäger [Spielertyp]}
			{Die Jäger sind die Spieler auf dem Spielfeld, welche versuchen die Punkte für das Team mithilfe des Quaffels zu erzielen. Dabei darf jedoch immer nur ein angreifender Jäger in der gegnerischen Hüterzone sein. Trifft der Jäger auf ein Feld auf dem der Quaffel ist, hebt er ihn auf. Um zu werfen, bewegt sich der Jäger auf das gewünschte Feld und nach seiner Bewegung wirft er den Ball auf einen der 3 Torringe.}
			{Jäger sind wichtig für das Spiel, da diese die Punkte machen, die das Team sammelt.}
			{Spielfeld, Quaffel, Torring, Quaffel Wurf}
			{++}
			{<Akteure>}
	
	\fanf	{Treiber [Spielertyp]}
			{Die Treiber sind ausgestattet mit Kloppern, welcher dazu dient, die Klatscher von den eigenen Teamkameraden fern zu halten und sie auf die gegnerischen Spieler zu lenken. Kommt ein Treiber auf das Feld eines Klatschers oder falls er bereits schon dort ist, so kann der Treiber in dieser Rundenphase den Klatscher kloppen. Er macht also einen Wurf mit dem Klatscher auf ein Zielfeld.}
			{Treiber haben die Funktion die eigenen Spieler vor den Klatschern zu beschützen, so dass nicht alle Spieler des Teams vom Besen fliegen.}
			{Klatscher, Spielfeld, Spieler, Kloppen}
			{++}
			{<Akteure>}
	
	\fanf	{Hüter [Spielertyp]}
			{Hüter sind die Beschützer der jeweils eigenen Torringe. Hüter können allerdings nur einen Quaffel aufnehmen und dann wieder werfen, zum Beispiel zu einem eigenen Jäger, die Hüter können aber keine Tore erzielen, dies ist nur durch Jäger möglich.}
			{Diese Anforderung geht aus den im Lastenheft zu Verfügung gestellten Spielregeln für das Spiel \textit{Fantastic Feasts} hervor.}
			{Spielfeld, Torringe, Hüterzone, Quaffel, Quaffel werfen,Spieler}
			{++}
			{<Akteure>}
	
	\fanf	{Sucher [Spielertyp]}
			{Der Sucher hat in dem gesamten Spiel nur eine einzige Aufgabe. Seine Aufgabe ist es, den goldenen Schnatz zu suchen und ihn genau dann zu fangen, wenn sein Team durch die zusätzlichen 30 Punkte das Spiel gewinnt.}
			{Der Sucher ist wichtig für das Spiel, da er die Möglichkeit hat, dass Spiel am ende noch zu drehen und somit dem Spiel einen zusätzlichen Kick zu verleihen, außerdem ist das Spiel sofort zu Ende, sobald der Schnatz gefangen wurde.}
			{Spielfeld,Spieler, goldener Schnatz}
			{++}
			{<Akteure>}
	
	\fanf	{Fans]}
			{<Beschreibung>}
			{Diese Anforderung geht aus den im Lastenheft zu Verfügung gestellten Spielregeln für das Spiel \textit{Fantastic Feasts} hervor.}
			{<Abhängigkeiten>}
			{<Prio>}
			{<Akteure>}	
	
	\fanf	{Elfen [Fantyp]}
			{<Beschreibung>}
			{Diese Anforderung geht aus den im Lastenheft zu Verfügung gestellten Spielregeln für das Spiel \textit{Fantastic Feasts} hervor.}
			{<Abhängigkeiten>}
			{<Prio>}
			{<Akteure>}
	
	\fanf	{Kobolde [Fantyp]}
			{<Beschreibung>}
			{Diese Anforderung geht aus den im Lastenheft zu Verfügung gestellten Spielregeln für das Spiel \textit{Fantastic Feasts} hervor.}
			{<Abhängigkeiten>}
			{<Prio>}
			{<Akteure>}
	
	\fanf	{Trolle [Fantyp]}
			{<Beschreibung>}
			{Diese Anforderung geht aus den im Lastenheft zu Verfügung gestellten Spielregeln für das Spiel \textit{Fantastic Feasts} hervor.}
			{<Abhängigkeiten>}
			{<Prio>}
			{<Akteure>}
	
	\fanf	{Niffler [Fantyp]}
			{<Beschreibung>}
			{Diese Anforderung geht aus den im Lastenheft zu Verfügung gestellten Spielregeln für das Spiel \textit{Fantastic Feasts} hervor.}
			{<Abhängigkeiten>}
			{<Prio>}
			{<Akteure>}
	
	\fanf	{Foul}
			{<Beschreibung>}
			{Diese Anforderung geht aus den im Lastenheft zu Verfügung gestellten Spielregeln für das Spiel \textit{Fantastic Feasts} hervor.}
			{<Abhängigkeiten>}
			{<Prio>}
			{<Akteure>}
	
	\fanf	{Flacken [Foul]}
			{<Beschreibung>}
			{Diese Anforderung geht aus den im Lastenheft zu Verfügung gestellten Spielregeln für das Spiel \textit{Fantastic Feasts} hervor.}
			{<Abhängigkeiten>}
			{<Prio>}
			{<Akteure>}
	
	\fanf	{Nachtarocken [Foul]}
			{<Beschreibung>}
			{Diese Anforderung geht aus den im Lastenheft zu Verfügung gestellten Spielregeln für das Spiel \textit{Fantastic Feasts} hervor.}
			{<Abhängigkeiten>}
			{<Prio>}
			{<Akteure>}
	
	\fanf	{Stutschen [Foul]}
			{<Beschreibung>}
			{Diese Anforderung geht aus den im Lastenheft zu Verfügung gestellten Spielregeln für das Spiel \textit{Fantastic Feasts} hervor.}
			{<Abhängigkeiten>}
			{<Prio>}
			{<Akteure>}
	
	\fanf	{Keilen [Foul]}
			{<Beschreibung>}
			{Diese Anforderung geht aus den im Lastenheft zu Verfügung gestellten Spielregeln für das Spiel \textit{Fantastic Feasts} hervor.}
			{<Abhängigkeiten>}
			{<Prio>}
			{<Akteure>}
	
	\fanf	{Schnaltzeln [Foul]}
			{<Beschreibung>}
			{Diese Anforderung geht aus den im Lastenheft zu Verfügung gestellten Spielregeln für das Spiel \textit{Fantastic Feasts} hervor.}
			{<Abhängigkeiten>}
			{<Prio>}
			{<Akteure>}
	
	\fanf	{Wurf mit dem Quaffel}
			{<Beschreibung>}
			{Diese Anforderung geht aus den im Lastenheft zu Verfügung gestellten Spielregeln für das Spiel \textit{Fantastic Feasts} hervor.}
			{<Abhängigkeiten>}
			{<Prio>}
			{<Akteure>}
	
	\fanf	{Quaffel Abfangen}
			{<Beschreibung>}
			{Diese Anforderung geht aus den im Lastenheft zu Verfügung gestellten Spielregeln für das Spiel \textit{Fantastic Feasts} hervor.}
			{<Abhängigkeiten>}
			{<Prio>}
			{<Akteure>}
	
	\fanf	{Torschuss}
			{<Beschreibung>}
			{Diese Anforderung geht aus den im Lastenheft zu Verfügung gestellten Spielregeln für das Spiel \textit{Fantastic Feasts} hervor.}
			{<Abhängigkeiten>}
			{<Prio>}
			{<Akteure>}
	
	\fanf	{Klatscher kloppen}
			{<Beschreibung>}
			{Diese Anforderung geht aus den im Lastenheft zu Verfügung gestellten Spielregeln für das Spiel \textit{Fantastic Feasts} hervor.}
			{<Abhängigkeiten>}
			{<Prio>}
			{<Akteure>}
			
	\fanf	{Partie-Konfiguration}
			{<Beschreibung>}
			{<Begründung>}
			{<Abhängigkeiten>}
			{<Prio>}
			{<Akteure>}
	
	\fanf	{Quidditchteam-Konfiguration}
			{<Beschreibung>}
			{<Begründung>}
			{<Abhängigkeiten>}
			{<Prio>}
			{<Akteure>}	
			
	\fanf	{Netzwerkinterface}
			{<Beschreibung>}
			{<Begründung>}
			{<Abhängigkeiten>}
			{<Prio>}
			{<Akteure>}
	
	\fanf	{Runde}
			{<Beschreibung>}
			{Diese Anforderung geht aus den im Lastenheft zu Verfügung gestellten Spielregeln für das Spiel \textit{Fantastic Feasts} hervor.}
			{<Abhängigkeiten>}
			{<Prio>}
			{<Akteure>}
	
	\fanf	{Ballphase}
			{<Beschreibung>}
			{Diese Anforderung geht aus den im Lastenheft zu Verfügung gestellten Spielregeln für das Spiel \textit{Fantastic Feasts} hervor.}
			{<Abhängigkeiten>}
			{<Prio>}
			{<Akteure>}
			
	\fanf	{Spielerphase}
			{<Beschreibung>}
			{Diese Anforderung geht aus den im Lastenheft zu Verfügung gestellten Spielregeln für das Spiel \textit{Fantastic Feasts} hervor.}
			{<Abhängigkeiten>}
			{<Prio>}
			{<Akteure>}
			
	\fanf	{Fanphase}
			{<Beschreibung>}
			{Diese Anforderung geht aus den im Lastenheft zu Verfügung gestellten Spielregeln für das Spiel \textit{Fantastic Feasts} hervor.}
			{<Abhängigkeiten>}
			{<Prio>}
			{<Akteure>}
			
	\fanf	{Zufallsgenerator}
			{<Beschreibung>}
			{<Begründung>}
			{<Abhängigkeiten>}
			{<Prio>}
			{<Akteure>}
			
	\fanf	{Spielende}
			{<Beschreibung>}
			{<Begründung>}
			{<Abhängigkeiten>}
			{<Prio>}
			{<Akteure>}
			
	\fanf	{Überlängenbehnadlung}
			{<Beschreibung>}
			{<Begründung>}
			{<Abhängigkeiten>}
			{<Prio>}
			{<Akteure>}
			
	\fanf	{Spiellogik}
			{<Beschreibung>}
			{<Begründung>}
			{<Abhängigkeiten>}
			{<Prio>}
			{<Akteure>}
	
	\fanf	{Log-Datei}
			{<Beschreibung>}
			{<Begründung>}
			{<Abhängigkeiten>}
			{<Prio>}
			{<Akteure>}
	
	\fanf	{<Titel>}
			{<Beschreibung>}
			{<Begründung>}
			{<Abhängigkeiten>}
			{<Prio>}
			{<Akteure>}
	
	\fanf	{<Titel>}
			{<Beschreibung>}
			{<Begründung>}
			{<Abhängigkeiten>}
			{<Prio>}
			{<Akteure>}
	
	\fanf	{<Titel>}
			{<Beschreibung>}
			{<Begründung>}
			{<Abhängigkeiten>}
			{<Prio>}
			{<Akteure>}
	
	\fanf	{<Titel>}
			{<Beschreibung>}
			{<Begründung>}
			{<Abhängigkeiten>}
			{<Prio>}
			{<Akteure>}
	
	\fanf	{<Titel>}
			{<Beschreibung>}
			{<Begründung>}
			{<Abhängigkeiten>}
			{<Prio>}
			{<Akteure>}
			
	
	\subsection{Server spezifische funktionale Anforderungen}
	
	\subsection{Client spezifische Funktionale Anforderungen}
	
	\fanf	{Hauptmenü [Ansicht]}
			{Erste grafische Oberfläche die dem Nutzer angezeigt wird, wenn die Anwendung gestartet wurde.}
			{Das Hauptmenü soll den Zentralen Punkt darstellen von dem aus alle Funktionen der Software zu erreichen. Es soll also unter anderem ein Spiel gestartet werden können, die Hilfe aufgerufen werden können, die Einstellungen der Anwendung angepasst werden können und eventuell vorhandene Statistiken aufgerufen werden können.}
			{<Abhängigkeiten>}
			{+}
			{<Akteure>}
	
	\fanf	{Connect to Game [Ansicht]}
			{Grafische Oberfläche um sich mit einem Server auf dem ein Spiel bereit gestellt wird zu verbinden. Dabei soll man außerdem die Möglichkeit haben seine Team-Konfiguration an zu geben, die man für das neue Spiel verwenden möchte.}
			{Der Nutzer muss die Möglichkeit haben sich komfortabel mit einem Server verbinden zu können.}
			{Abb}
			{+}
			{<Akteure>}
	
	\fanf	{End of Game [Ansicht]}
			{Grafische Oberfläche die, die Spieler sehen nachdem eine Partie zu Ende ist. Der Nutzer sollte hier auch die Möglichkeit haben die Anwendung zu verlassen oder wieder ins Hauptmenü zurück kehren. Optional ist hier auch Platz für etwaige Statistiken über den Spielverlauf.}
			{Nach dem Ende einer Partie muss dem Nutzer mitgeteilt werden ob er gewonnen hat oder nicht und wie es von da an weiter geht. }
			{<Abhängigkeiten>}
			{-+}
			{<Akteure>}
			
	\fanf	{Import Team Config [Ansicht]}
			{Grafische Oberfläche zum importieren bzw. öffen einer Team-Konfiguration für ein Spiel.}
			{Es muss für den Benutzer eine einfachen Weg geben eine Team Konfiguration im Dateisystem zu suchen und an die Anwendung zu übergeben.}
			{<Abhängigkeiten>}
			{-+}
			{<Akteure>}
			
	\fanf	{Spiel [Ansicht]}
			{Grafische Oberfläche die der Spieler sieht.}
			{Es handelt sich um eine Anwendung mit grafischer Benutzeroberfläche. Es ist also zwingend von Nöten, dass such das aktuelle Spielgeschehen angezeigt werden kann.}
			{<Abhängigkeiten>}
			{++}
			{<Akteure>}	
			
	\fanf	{Hilfe [Ansicht]}
			{Grafische Oberfläche, in der zum einen das Spielprinzip erklärt wird und zum anderen gezeigt wird wie genau man die Clientsoftware bedient wird.}
			{Um unerfahren Benutzer die Bedienung der Software zu erleichtern.}
			{<Abhängigkeiten>}
			{-+}
			{<Akteure>}
			
	\fanf	{Beobachter [Ansicht]}
			{Grafische Oberfläche, die ein Gast sieht.}
			{Wenn ein Nutzer einem Spiel nur als Gast zuschaut muss ihm das Spiel trotzdem in eine grafische Oberfläche aufbereitet werden.}
			{<Abhängigkeiten>}
			{+}
			{<Akteure>}
	
	\fanf	{Game Rendering Engine}
			{Die Game Rendering Engine bereitet die Ansicht des Spielfelds grafisch auf.}
			{Die Benutzeroberfläche muss während des Spiels mit Inhalt gefüllt werden. Dieser Inhalt muss je nach Spielgeschehen automatisch generiert werden.}
			{<Abhängigkeiten>}
			{++}
			{<Akteure>}
	
	\fanf	{Input Handler}
			{Diese Einheit ist für die Verarbeitung von Benutzereingaben verantwortlich.}
			{Jede Benutzereingabe muss ausgewertet werden.Nach der Validierung muss dann eine Entscheidung getroffen werden was als Reaktion auf diese Eingaben passieren muss.}
			{<Abhängigkeiten>}
			{++}
			{<Akteure>}
			
	\fanf	{Input Validierung}
			{Einheit die Benutzereingaben auf Korrektheit prüft.}
			{Um etwaige falsche Benutzereingaben zu erkennen und den Nuzer darauf hin weisen zu können ist es von Nöten, dass alle Benutzereingaben geprüft werden.}
			{<Abhängigkeiten>}
			{<Prio>}
			{<Akteure>}
	
	\fanf	{Hotkey}
			{Oft benötigte Funktionen auf Bestimmte (besondere) Testen (-Kombinationen) mappen.}
			{Hotkeys sind optionale Features, die im Lastenheft aufgeführt sind und zu einer einfacheren Spielsteuerung und höherem Spielkomfort betragen können.}
			{<Abhängigkeiten>}
			{--}
			{<Akteure>}
			
	\fanf	{Pausieren}
			{Das aktuelle Spiel pausieren.}
			{Pausieren ist ein optionales Feature, das im Lastenheft aufgeführt ist und einem menschliche Spieler im Client zur Verfügung stehen sollte um den Spielkomfort zu erhöhen.}
			{<Abhängigkeiten>}
			{--}
			{<Akteure>}
			
	
	\subsection{Quidditchteam-Editor spezifische funktionale Anforderungen}
	
	\fanf	{Edit Team View}
			{Grafische Oberfläche in der der Nutzer sein Team nach seinen Wünschen entsprechend anpassen kann.}
			{Für das Bearbeiten des Teams soll dem Nutzer eine grafische Oberfläche bereit gestellt werden.}
			{<Abhängigkeiten>}
			{<Prio>}
			{<Akteure>}
			
	\fanf	{Open Team View}
			{'Datei Öffnen' Dialog um die Team-Konfiguration von einem beliebigen Ort im Dateisystem zu öffen}
			{}
			{<Abhängigkeiten>}
			{<Prio>}
			{<Akteure>}
	
	\fanf	{Save Team View}
			{'Datei Speichern' Dialog um die Team-Konfiguration an einem beliebigen Ort im Dateisystem abzulegen.}
			{}
			{<Abhängigkeiten>}
			{<Prio>}
			{<Akteure>}
			
	
	\subsection{Nicht funktionale Anforderungen}
	
	\qanf 	{Plattformunabhängigkeit}
			{Der Client und der Team-Konfigurator soll auf mindestens einer gängigen Computerbetriebssystem-Plattform (z.B. Linux, Windows) uneingeschränkt benutzbar sein. Des weiteren soll die Serveranwendung und der KI-Client auf mindestens zwei gängigen Computerbetriebssystem-Plattformen (z.B. Linux, Windows) uneingeschränkt benutzbar sein.}
			{Die Plattformunabhängigkeit ist im Lastenheft gefordert.}
			{<Abhängigkeiten>}
			{++}
			{<Akteure>}
	
	\qanf 	{Version-Controlling}
			{Beim Verwalten des Quellcodes soll ein Git basiertes Version-Controlling Werkzeug (\textit{GitHub / GitLab}) verwendet werden.}
			{Durch das Verwenden eines Versionierungswerkzeuges wird das zusammenarbeiten unterschiedlicher Entwickler erleichtert, da das zusammenführen des Codes größtenteils automatisiert abläuft.}
			{-}
			{++}
			{-}
	
	\qanf 	{Continuous Integration}
			{In die Version-Controlling Lösung mit Hilfe einer CI automatisch jeder gepushte Commit Unit-Tests und der Statischen Codeanalyse unterzogen werden. Zudem soll eine automatisierte Code Dokumentation angestoßen werden. Bei erfolgreichem Abschließen alle Test soll zum Schluss der aktuelle Stand deployed werden.}
			{Die CI nimmt den Entwicklern Arbeit ab und kann dazu beitragen, dass Fehler frühzeitig erkannt und behoben werden können.}
			{-}
			{-+}
			{<Akteure>}
	
	\qanf 	{Statische Codeanalyse}
			{Mit Hilfe des Tools 'SonarQube' bzw. 'SonarCloud' soll der gesamt Quellcode einer statischen Analyse unterzogen werden. Dabei darf die technische Codequalität von diesen Tool nicht schlechter als 'B' bewertet werden.}
			{Quellcode mit einer hohen Codequalität ist weniger problemanfällig.}
			{-}
			{+}
			{<Akteure>}
	
	\qanf 	{Automatisierte Unit-Tests}
			{50 Prozent des Quellcodes der einzelnen Komponenten soll durch automatische Unit-Test auf Korrektheit und Funktion geprüft werden}
			{Da alle Komponenten möglichst fehlerfrei funktionieren müssen ist es unerlässlich die einzelnen Teil der Software ständig auf ihre Funktionalität zu prüfen, die ist nur effizient möglich wenn automatisiert Test durchgeführt werden, damit Fehler frühzeitig erkannt werden.}
			{-}
			{-+}
			{<Akteure>}
	
	\qanf 	{Docker Container}
			{Um die Plattformunabhängigkeit zu gewährleisten soll sowohl die Server Komponente, als auch die KI-Komponenten mit Hilfe eines Docker Container veröffentlicht werden.}
			{Docker Container bieten den Vorteil, dass die Software nicht auf jedem Zielsystem neu compiliert werden muss sondern, sobald sie auf einem System in einem Docker-Container lauffähig gemacht wurde lässt sich dieser Container in der Regel auf diversen anderen Zielsystemen ausführen.}
			{-}
			{+}
			{<Akteure>}
	
	\qanf 	{Dokumentation}
			{Alle Klassen und Methoden der Software müssen dokumentiert werden. Dabei sollen mindestens alle Übergabeparameter und Rückgabewerte genau spezifiziert werden. Zudem ist sind komplexe Algorithmen detailliert zu dokumentieren.}
			{Gut dokumentierte Software vereinfacht die Fehlersuche, die Wartung und das hinzufügen von neuen Features.}
			{-}
			{+}
			{<Akteure>}
	
	\qanf 	{Benutzerhandbuch}
			{Zu jeder Komponente des Projektes muss eine Benutzerhandbuchh existieren, in welchem alle Features unmissverständlich erklärt sind, sodass ein neuer Benutzer auf Basis des Benutzerhandbuches die Software bedienen kann.}
			{Das Benutzerhandbuch wird im Lastenheft gefordert.}
			{-}
			{-+}
			{<Akteure>}
	
	\qanf 	{Anwendungssprache}
			{Das User-Interface der Anwendungen soll in deutscher Sprache gestaltet werden.}
			{Die Anwendungssprache ist im Lastenheft vorgegeben.}
			{-}
			{-+}
			{<Akteure>}
	
	\qanf 	{Implementierungssprache}
			{Die Implementierung der Anwendungen soll in englischer Sprache gehalten sein.}
			{Die Implementierungssprache ist im Lastenheft vorgegeben.}
			{-}
			{-+}
			{<Akteure>}
	
	\qanf 	{Dokumentationssprache}
			{Die Dokumentation der Software kann in englischer oder deutscher Sprache gestaltet sein.}
			{Die Dokumentationssprache ist im Lastenheft vorgegeben.}
			{-}
			{-+}
			{<Akteure>}
	
	\qanf 	{Programmiersprache}
			{Die Software soll in einer der folgenden Programmiersprachen geschrieben sein: Java, C++, C\# Die entgültig verwendete Sprache kann jedoch von Komponente zu Komponente variieren, müss aber mit dem Kunden abgesprochen werden.}
			{Es soll eine Programmiersprache mit großem Funktionsumfang verwendet werden, welche von allen Teammitgliedern beherrscht wird.}
			{-}
			{++}
			{<Akteure>}	
	
	\qanf 	{Format für Konfigurationsdateien}
			{Alle Konfigurationsdateien müssen dem \textit{JSON} Standard genügen. Des Weiteren sind alle vom Komitee festgelegten weiteren Standards einzuhalten.}
			{Das Konfigurationsformat ist im Lastenheft vorgegeben.}
			{-}
			{+}
			{<Akteure>}
	
	\qanf 	{Netzwerkkommunikation}
			{Die Netzwerkkommunikation zwischen Client und Server soll über sogenannte \textit{Web-Socket-Sessions} realisiert werden, sodass Client und Server ortsunabhängig von einander betrieben werden können.}
			{Die Netzwerkkommunikation muss gewissen Standards genügen, damit Client- und Serveranwendungen von unterschiedlichen Entwicklerteams mit einander kompatibel sind und Client und Server ortsunabhängig von einander betrieben werden können.}
			{-}
			{<Prio>}
			{<Akteure>}
	
	\qanf 	{Log-Dateien}
			{Log-Dateien anlegen um unter anderem den Spielverlauf zu Speichern und eventuelle Fehlfunktionen der Software fest zu halten.}
			{Log-Dateien können unter anderen die Wartung der Software erleichtern und für zusätzliche Features, wie eine Statistik über den Spielverlauf verwendet werden.}
			{-}
			{-}
			{<Akteure>}
	
	\qanf 	{Funktionalität}
			{Die Anwendungen müssen alle im Lastenheft als Minimalanforderungen aufgeführten Anforderungen erfüllen.}
			{Um die Abnahmen zu bestehen müssen die Minimalanforderungen erfüllt werden.}
			{-}
			{++}
			{<Akteure>}
	
	\qanf 	{Zuverlässigkeit}
			{Die Anwendungen dürfen niemals komplett abstürzen.}
			{Durch zu häufiges Abstürzen der Software ist das Benutzererlebnis massiv beeinträchtigt.}
			{-}
			{+}
			{<Akteure>}
	
	\qanf 	{Robustheit}
			{Die Anwendungen dürfen nicht aufgrund einer Falschen Benutzereingabe abstürzen, sondern müssen den Benutzer auf seinen Fehler hinweisen.}
			{Um das Benutzererlebnis nicht zu beeinträchtigen und keine Sicherheitslücken zu verursachen ist es notwendig, dass die Funktion der Software nicht durch fehlerhafte Benutzereingaben beeinträchtigt wird.}
			{-}
			{++}
			{<Akteure>}
	
	\qanf 	{Benutzbarkeit}
			{Dem Endnutzer muss es möglich sein alle Komponenten des Projektes nur auf Basis des Mitgelieferten Benutzerhandbuches und den Hilfeseiten die Software ohne Einschränkungen bedienen zu können.}
			{Wenn es für die Endnutzer der Software zu kompliziert ist die Software zu Benutzen, dann ist das Benutzererlebnis erheblich gestört und die Software wird nicht Benutzt werden, da die Endbenutzer unzufrieden sind.}
			{-}
			{+}
			{<Akteure>}
			
	\qanf 	{Wartbarkeit}
			{Die Software muss so aufgebaut sein, dass einzelne Teilstücke bei Bedarf ohne Umbauten der übrigen Software ersetzbar sind.}
			{Im Falle einer Fehlfunktion in einem Teilstück der Software muss diese einfach austauschbar sein um den Fehler schnellst möglich beheben zu können. Zudem sollte das Hinzufügen weiterer Features möglich sein um das Produkt stetig weiter entwickeln zu können.}
			{-}
			{-+}
			{<Akteure>}
	
    
\end{document}
