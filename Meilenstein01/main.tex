\documentclass[DIN, pagenumber=false, fontsize=11pt, parskip=half]{scrartcl}

\usepackage{amsmath}
\usepackage{amsfonts}
\usepackage{amssymb}
\usepackage{enumitem}
\usepackage[utf8]{inputenc} % this is needed for umlauts
\usepackage[ngerman]{babel} % this is needed for umlauts
\usepackage[T1]{fontenc} 
\usepackage{commath}
\usepackage{xcolor}
\usepackage{booktabs}
\usepackage{float}
\usepackage{tikz-timing}
\usepackage{tikz}
\usepackage{multirow}
\usepackage[final]{pdfpages}

\usetikzlibrary{calc,shapes.multipart,chains,arrows}

\title{Softwaregrundprojekt}
\author{Gruppe 10}

\newcommand{\anf}[7] {
    \begin{table}[H]
        \centering
        \begin{tabular}{p{3.2cm}|p{9.8cm}}
            \textbf{ID:} & \textbf{#1} \\ \hline
            \textbf{Titel:} & #2 \\ \hline
            \textbf{Beschreibung:} & #3 \\ \hline
            \textbf{Begründung:} & #4 \\ \hline
            \textbf{Abhängigkeiten:} & #5 \\ \hline
            \textbf{Priorität:} & #6 \\ \hline
            \textbf{Akteure:} & #7 \\ \hline
        \end{tabular}
    \end{table}
}

\newcounter{fanfCount}
\newcommand{\fanf}[6] {
    \stepcounter{fanfCount}
    \anf{FA\thefanfCount}{#1}{#2}{#3}{#4}{#5}{#6}
}
\newcounter{qanfCount}
\newcommand{\qanf}[6] {
    \stepcounter{qanfCount}
    \anf{QA\theqanfCount}{#1}{#2}{#3}{#4}{#5}{#6}
}

\newcommand{\akt}[4] {
    \begin{table}[H]
        \centering
        \begin{tabular}{p{3cm}|p{10cm}}
            \textbf{ID:} & \textbf{#1} \\ \hline
            \textbf{Titel:} & #2 \\ \hline
            \textbf{Beschreibung:} & #3 \\ \hline
            \textbf{Rolle:} & #4 \\ \hline
        \end{tabular}
    \end{table}
}

\newcounter{faktCount}
\newcommand{\fakt}[3] {
    \stepcounter{faktCount}
    \akt{AKT\thefaktCount}{#1}{#2}{#3}
}

\begin{document}
    \maketitle
    \section{Kontextanalyse}
    %TODO
    \section{Fachwissen}
    %TODO
    \includepdf[pagecommand={\section{Domänenmodell}}]{domaenenmodell.pdf}
    \section{Anforderungsdefinition}
    %TODO
    
    \subsection{Akteure}
    
    \fakt	{Anwendungsbenutzer}
			{Menschlicher Benutzer, der die Anwendungen bedient.}
			{<Rolle>}
			
	\fakt	{Spielender Benutzer}
			{Anwendungsbenutzer, der mit der Client-Anwendung am tatsächlichen Spielgeschehen teilnimmt.}
			{<Rolle>}
			
	\fakt	{Beobachtender Benutzer}
			{<Beschreibung>}
			{<Rolle>}
			
	\fakt	{Systemadministrator}
			{Anwendungsnutzer mit Zusatzqualifikation um Server zu verwalten.}
			{Der Systemadministrator hat zugriff auf die Server Anwendung. Er ist dafür verantwortlich eine Instanz der Server Anwendung zu starten und zu betreuen. Zudem hat er Zugriff auf die Partie-Konfiguration und kann diese bei bedarf verändern.}
	
	\fakt	{KI}
			{<Beschreibung>}
			{<Rolle>}
	
	\fakt	{Kunde}
			{SoPra-Tutor}
			{<Rolle>}
	
	\fakt	{Client [Komponente]}
			{<Beschreibung>}
			{<Rolle>}
	
	\fakt	{KI-Client [Komponente]}
			{<Beschreibung>}
			{<Rolle>}
	
	\fakt	{Server [Komponente]}
			{<Beschreibung>}
			{<Rolle>}
			
	\fakt	{Quidditchteam-Editor [Komponente]}
			{<Beschreibung>}
			{<Rolle>}
	
	\fakt 	{Team}
			{<Beschreibung>}
			{<Rolle>}
	
	\fakt 	{Schiedsrichter}
			{<Beschreibung>}
			{<Rolle>}
	
	
	\subsection{Funktionale Anforderungen (allgemein)}
	
	\fanf	{Quidditch-Spielfeld}
			{<Beschreibung>}
			{Diese Anforderung geht aus den im Lastenheft zu Verfügung gestellten Spielregeln für das Spiel \textit{Fantastic Feasts} hervor.}
			{<Abhängigkeiten>}
			{<Prio>}
			{<Akteure>}
	
	\fanf	{Mittelkreis}
			{<Beschreibung>}
			{Diese Anforderung geht aus den im Lastenheft zu Verfügung gestellten Spielregeln für das Spiel \textit{Fantastic Feasts} hervor.}
			{<Abhängigkeiten>}
			{<Prio>}
			{<Akteure>}
	
	\fanf	{Mittelfeld}
			{<Beschreibung>}
			{Diese Anforderung geht aus den im Lastenheft zu Verfügung gestellten Spielregeln für das Spiel \textit{Fantastic Feasts} hervor.}
			{<Abhängigkeiten>}
			{<Prio>}
			{<Akteure>}
	
	\fanf	{Hüterzone}
			{<Beschreibung>}
			{<Begründung>}
			{<Abhängigkeiten>}
			{<Prio>}
			{<Akteure>}
	
	\fanf	{Torring}
			{<Beschreibung>}
			{Diese Anforderung geht aus den im Lastenheft zu Verfügung gestellten Spielregeln für das Spiel \textit{Fantastic Feasts} hervor.}
			{<Abhängigkeiten>}
			{<Prio>}
			{<Akteure>}
	
	\fanf	{Schussvektor}
			{<Beschreibung>}
			{Diese Anforderung geht aus den im Lastenheft zu Verfügung gestellten Spielregeln für das Spiel \textit{Fantastic Feasts} hervor.}
			{<Abhängigkeiten>}
			{<Prio>}
			{<Akteure>}
	
	\fanf	{Bälle}
			{<Beschreibung>}
			{Diese Anforderung geht aus den im Lastenheft zu Verfügung gestellten Spielregeln für das Spiel \textit{Fantastic Feasts} hervor.}
			{<Abhängigkeiten>}
			{<Prio>}
			{<Akteure>}
	
	\fanf	{Quaffel [Ball]}
			{<Beschreibung>}
			{Diese Anforderung geht aus den im Lastenheft zu Verfügung gestellten Spielregeln für das Spiel \textit{Fantastic Feasts} hervor.}
			{<Abhängigkeiten>}
			{<Prio>}
			{<Akteure>}
	
	\fanf	{Klatscher [Ball]}
			{<Beschreibung>}
			{Diese Anforderung geht aus den im Lastenheft zu Verfügung gestellten Spielregeln für das Spiel \textit{Fantastic Feasts} hervor.}
			{<Abhängigkeiten>}
			{<Prio>}
			{<Akteure>}
	
	\fanf	{Goldener Schnatz [Ball]}
			{<Beschreibung>}
			{Diese Anforderung geht aus den im Lastenheft zu Verfügung gestellten Spielregeln für das Spiel \textit{Fantastic Feasts} hervor.}
			{<Abhängigkeiten>}
			{<Prio>}
			{<Akteure>}
	
	\fanf	{Besen}
			{<Beschreibung>}
			{Diese Anforderung geht aus den im Lastenheft zu Verfügung gestellten Spielregeln für das Spiel \textit{Fantastic Feasts} hervor.}
			{<Abhängigkeiten>}
			{<Prio>}
			{<Akteure>}
	
	\fanf	{Zauberfauch [Besen]}
			{<Beschreibung>}
			{Diese Anforderung geht aus den im Lastenheft zu Verfügung gestellten Spielregeln für das Spiel \textit{Fantastic Feasts} hervor.}
			{<Abhängigkeiten>}
			{<Prio>}
			{<Akteure>}
	
	\fanf	{Sauberwisch 11 [Besen]}
			{<Beschreibung>}
			{Diese Anforderung geht aus den im Lastenheft zu Verfügung gestellten Spielregeln für das Spiel \textit{Fantastic Feasts} hervor.}
			{<Abhängigkeiten>}
			{<Prio>}
			{<Akteure>}
	
	\fanf	{Komet 2-60 [Besen]}
			{<Beschreibung>}
			{Diese Anforderung geht aus den im Lastenheft zu Verfügung gestellten Spielregeln für das Spiel \textit{Fantastic Feasts} hervor.}
			{<Abhängigkeiten>}
			{<Prio>}
			{<Akteure>}
	
	\fanf	{Nimbus 2001  [Besen]}
			{<Beschreibung>}
			{Diese Anforderung geht aus den im Lastenheft zu Verfügung gestellten Spielregeln für das Spiel \textit{Fantastic Feasts} hervor.}
			{<Abhängigkeiten>}
			{<Prio>}
			{<Akteure>}
	
	\fanf	{Feuerblitz [Besen]}
			{<Beschreibung>}
			{Diese Anforderung geht aus den im Lastenheft zu Verfügung gestellten Spielregeln für das Spiel \textit{Fantastic Feasts} hervor.}
			{<Abhängigkeiten>}
			{<Prio>}
			{<Akteure>}
	
	\fanf	{Spieler}
			{<Beschreibung>}
			{Diese Anforderung geht aus den im Lastenheft zu Verfügung gestellten Spielregeln für das Spiel \textit{Fantastic Feasts} hervor.}
			{<Abhängigkeiten>}
			{<Prio>}
			{<Akteure>}	
	
	\fanf	{Jäger [Spielertyp]}
			{<Beschreibung>}
			{Diese Anforderung geht aus den im Lastenheft zu Verfügung gestellten Spielregeln für das Spiel \textit{Fantastic Feasts} hervor.}
			{<Abhängigkeiten>}
			{<Prio>}
			{<Akteure>}
	
	\fanf	{Treiber [Spielertyp]}
			{<Beschreibung>}
			{Diese Anforderung geht aus den im Lastenheft zu Verfügung gestellten Spielregeln für das Spiel \textit{Fantastic Feasts} hervor.}
			{<Abhängigkeiten>}
			{<Prio>}
			{<Akteure>}
	
	\fanf	{Hüter [Spielertyp]}
			{<Beschreibung>}
			{Diese Anforderung geht aus den im Lastenheft zu Verfügung gestellten Spielregeln für das Spiel \textit{Fantastic Feasts} hervor.}
			{<Abhängigkeiten>}
			{<Prio>}
			{<Akteure>}
	
	\fanf	{[Spielertyp] Sucher}
			{<Beschreibung>}
			{Diese Anforderung geht aus den im Lastenheft zu Verfügung gestellten Spielregeln für das Spiel \textit{Fantastic Feasts} hervor.}
			{<Abhängigkeiten>}
			{<Prio>}
			{<Akteure>}
	
	\fanf	{Fans]}
			{<Beschreibung>}
			{Diese Anforderung geht aus den im Lastenheft zu Verfügung gestellten Spielregeln für das Spiel \textit{Fantastic Feasts} hervor.}
			{<Abhängigkeiten>}
			{<Prio>}
			{<Akteure>}	
	
	\fanf	{Elfen [Fantyp]}
			{<Beschreibung>}
			{Diese Anforderung geht aus den im Lastenheft zu Verfügung gestellten Spielregeln für das Spiel \textit{Fantastic Feasts} hervor.}
			{<Abhängigkeiten>}
			{<Prio>}
			{<Akteure>}
	
	\fanf	{Kobolde [Fantyp]}
			{<Beschreibung>}
			{Diese Anforderung geht aus den im Lastenheft zu Verfügung gestellten Spielregeln für das Spiel \textit{Fantastic Feasts} hervor.}
			{<Abhängigkeiten>}
			{<Prio>}
			{<Akteure>}
	
	\fanf	{Trolle [Fantyp]}
			{<Beschreibung>}
			{Diese Anforderung geht aus den im Lastenheft zu Verfügung gestellten Spielregeln für das Spiel \textit{Fantastic Feasts} hervor.}
			{<Abhängigkeiten>}
			{<Prio>}
			{<Akteure>}
	
	\fanf	{Niffler [Fantyp]}
			{<Beschreibung>}
			{Diese Anforderung geht aus den im Lastenheft zu Verfügung gestellten Spielregeln für das Spiel \textit{Fantastic Feasts} hervor.}
			{<Abhängigkeiten>}
			{<Prio>}
			{<Akteure>}
	
	\fanf	{Foul}
			{<Beschreibung>}
			{Diese Anforderung geht aus den im Lastenheft zu Verfügung gestellten Spielregeln für das Spiel \textit{Fantastic Feasts} hervor.}
			{<Abhängigkeiten>}
			{<Prio>}
			{<Akteure>}
	
	\fanf	{Flacken [Foul]}
			{<Beschreibung>}
			{Diese Anforderung geht aus den im Lastenheft zu Verfügung gestellten Spielregeln für das Spiel \textit{Fantastic Feasts} hervor.}
			{<Abhängigkeiten>}
			{<Prio>}
			{<Akteure>}
	
	\fanf	{Nachtarocken [Foul]}
			{<Beschreibung>}
			{Diese Anforderung geht aus den im Lastenheft zu Verfügung gestellten Spielregeln für das Spiel \textit{Fantastic Feasts} hervor.}
			{<Abhängigkeiten>}
			{<Prio>}
			{<Akteure>}
	
	\fanf	{Stutschen [Foul]}
			{<Beschreibung>}
			{Diese Anforderung geht aus den im Lastenheft zu Verfügung gestellten Spielregeln für das Spiel \textit{Fantastic Feasts} hervor.}
			{<Abhängigkeiten>}
			{<Prio>}
			{<Akteure>}
	
	\fanf	{Keilen [Foul]}
			{<Beschreibung>}
			{Diese Anforderung geht aus den im Lastenheft zu Verfügung gestellten Spielregeln für das Spiel \textit{Fantastic Feasts} hervor.}
			{<Abhängigkeiten>}
			{<Prio>}
			{<Akteure>}
	
	\fanf	{Schnaltzeln [Foul]}
			{<Beschreibung>}
			{Diese Anforderung geht aus den im Lastenheft zu Verfügung gestellten Spielregeln für das Spiel \textit{Fantastic Feasts} hervor.}
			{<Abhängigkeiten>}
			{<Prio>}
			{<Akteure>}
	
	\fanf	{Wurf mit dem Quaffel}
			{<Beschreibung>}
			{Diese Anforderung geht aus den im Lastenheft zu Verfügung gestellten Spielregeln für das Spiel \textit{Fantastic Feasts} hervor.}
			{<Abhängigkeiten>}
			{<Prio>}
			{<Akteure>}
	
	\fanf	{Quaffel Abfangen}
			{<Beschreibung>}
			{Diese Anforderung geht aus den im Lastenheft zu Verfügung gestellten Spielregeln für das Spiel \textit{Fantastic Feasts} hervor.}
			{<Abhängigkeiten>}
			{<Prio>}
			{<Akteure>}
	
	\fanf	{Torschuss}
			{<Beschreibung>}
			{Diese Anforderung geht aus den im Lastenheft zu Verfügung gestellten Spielregeln für das Spiel \textit{Fantastic Feasts} hervor.}
			{<Abhängigkeiten>}
			{<Prio>}
			{<Akteure>}
	
	\fanf	{Klatscher kloppen}
			{<Beschreibung>}
			{Diese Anforderung geht aus den im Lastenheft zu Verfügung gestellten Spielregeln für das Spiel \textit{Fantastic Feasts} hervor.}
			{<Abhängigkeiten>}
			{<Prio>}
			{<Akteure>}
			
	\fanf	{Partie-Konfiguration}
			{<Beschreibung>}
			{<Begründung>}
			{<Abhängigkeiten>}
			{<Prio>}
			{<Akteure>}
	
	\fanf	{Quidditchteam-Konfiguration}
			{<Beschreibung>}
			{<Begründung>}
			{<Abhängigkeiten>}
			{<Prio>}
			{<Akteure>}	
			
	\fanf	{Netzwerkinterface}
			{<Beschreibung>}
			{<Begründung>}
			{<Abhängigkeiten>}
			{<Prio>}
			{<Akteure>}
	
	\fanf	{Runde}
			{<Beschreibung>}
			{Diese Anforderung geht aus den im Lastenheft zu Verfügung gestellten Spielregeln für das Spiel \textit{Fantastic Feasts} hervor.}
			{<Abhängigkeiten>}
			{<Prio>}
			{<Akteure>}
	
	\fanf	{Ballphase}
			{<Beschreibung>}
			{Diese Anforderung geht aus den im Lastenheft zu Verfügung gestellten Spielregeln für das Spiel \textit{Fantastic Feasts} hervor.}
			{<Abhängigkeiten>}
			{<Prio>}
			{<Akteure>}
			
	\fanf	{Spielerphase}
			{<Beschreibung>}
			{Diese Anforderung geht aus den im Lastenheft zu Verfügung gestellten Spielregeln für das Spiel \textit{Fantastic Feasts} hervor.}
			{<Abhängigkeiten>}
			{<Prio>}
			{<Akteure>}
			
	\fanf	{Fanphase}
			{<Beschreibung>}
			{Diese Anforderung geht aus den im Lastenheft zu Verfügung gestellten Spielregeln für das Spiel \textit{Fantastic Feasts} hervor.}
			{<Abhängigkeiten>}
			{<Prio>}
			{<Akteure>}
			
	\fanf	{Zufallsgenerator}
			{<Beschreibung>}
			{<Begründung>}
			{<Abhängigkeiten>}
			{<Prio>}
			{<Akteure>}
			
	\fanf	{Spielende}
			{<Beschreibung>}
			{<Begründung>}
			{<Abhängigkeiten>}
			{<Prio>}
			{<Akteure>}
			
	\fanf	{Überlängenbehnadlung}
			{<Beschreibung>}
			{<Begründung>}
			{<Abhängigkeiten>}
			{<Prio>}
			{<Akteure>}
			
	\fanf	{Spiellogik}
			{<Beschreibung>}
			{<Begründung>}
			{<Abhängigkeiten>}
			{<Prio>}
			{<Akteure>}
	
	\fanf	{Log-Datei}
			{<Beschreibung>}
			{<Begründung>}
			{<Abhängigkeiten>}
			{<Prio>}
			{<Akteure>}
	
	\fanf	{<Titel>}
			{<Beschreibung>}
			{<Begründung>}
			{<Abhängigkeiten>}
			{<Prio>}
			{<Akteure>}
	
	\fanf	{<Titel>}
			{<Beschreibung>}
			{<Begründung>}
			{<Abhängigkeiten>}
			{<Prio>}
			{<Akteure>}
	
	\fanf	{<Titel>}
			{<Beschreibung>}
			{<Begründung>}
			{<Abhängigkeiten>}
			{<Prio>}
			{<Akteure>}
	
	\fanf	{<Titel>}
			{<Beschreibung>}
			{<Begründung>}
			{<Abhängigkeiten>}
			{<Prio>}
			{<Akteure>}
	
	\fanf	{<Titel>}
			{<Beschreibung>}
			{<Begründung>}
			{<Abhängigkeiten>}
			{<Prio>}
			{<Akteure>}
			
	
	\subsection{Funktionale Anforderungen (Server)}
	
	\subsection{Funktionale Anforderungen (Client)}
	
	\fanf	{Hauptmenü [Client-View]}
			{<Beschreibung>}
			{<Begründung>}
			{<Abhängigkeiten>}
			{<Prio>}
			{<Akteure>}
	
	\fanf	{Connect to Game [Client-View]}
			{<Beschreibung>}
			{<Begründung>}
			{<Abhängigkeiten>}
			{<Prio>}
			{<Akteure>}
	
	\fanf	{End of Game [Client-View]}
			{<Beschreibung>}
			{<Begründung>}
			{<Abhängigkeiten>}
			{<Prio>}
			{<Akteure>}
			
	\fanf	{Import Team Config [Client-View]}
			{<Beschreibung>}
			{<Begründung>}
			{<Abhängigkeiten>}
			{<Prio>}
			{<Akteure>}
			
	\fanf	{Game Play [Client-View]}
			{<Beschreibung>}
			{<Begründung>}
			{<Abhängigkeiten>}
			{<Prio>}
			{<Akteure>}	
			
	\fanf	{Hilfe [Client-View]}
			{<Beschreibung>}
			{<Begründung>}
			{<Abhängigkeiten>}
			{<Prio>}
			{<Akteure>}
			
	\fanf	{Beobachter [Client-View]}
			{<Beschreibung>}
			{<Begründung>}
			{<Abhängigkeiten>}
			{<Prio>}
			{<Akteure>}
	
	\fanf	{Game Rendering Engine}
			{<Beschreibung>}
			{<Begründung>}
			{<Abhängigkeiten>}
			{<Prio>}
			{<Akteure>}
	
	\fanf	{Input Handler [Client]}
			{<Beschreibung>}
			{<Begründung>}
			{<Abhängigkeiten>}
			{<Prio>}
			{<Akteure>}
			
	\fanf	{Input Validierung [Client]}
			{<Beschreibung>}
			{<Begründung>}
			{<Abhängigkeiten>}
			{<Prio>}
			{<Akteure>}
	
	\fanf	{Hotkey [Client]}
			{<Beschreibung>}
			{<Begründung>}
			{<Abhängigkeiten>}
			{--}
			{<Akteure>}
			
	\fanf	{Pausieren [Client]}
			{<Beschreibung>}
			{<Begründung>}
			{<Abhängigkeiten>}
			{--}
			{<Akteure>}
			
	
	\subsection{Funktionale Anforderungen (Quidditchteam-Editor)}
	
	\fanf	{Edit Team [Team-Editor-View]}
			{<Beschreibung>}
			{<Begründung>}
			{<Abhängigkeiten>}
			{<Prio>}
			{<Akteure>}
	
	\fanf	{Save Team [Team-Editor-View]}
			{<Beschreibung>}
			{<Begründung>}
			{<Abhängigkeiten>}
			{<Prio>}
			{<Akteure>}
			
	
	\subsection{Nicht funktionale Anforderungen}
	
	\qanf 	{Plattformunabhängigkeit}
			{Der Spielclient soll auf mindestens einer gängigen Computerbetriebssystem-Plattform (z.B. Linux, Windows) uneingeschränkt benutzbar sein. Des weiteren soll die Server- und die KI-Komponente auf mindestens zwei gängigen Computerbetriebssystem-Plattformen (z.B. Linux, Windows) uneingeschränkt benutzbar sein.}
			{<Begründung>}
			{<Abhängigkeiten>}
			{<Prio>}
			{<Akteure>}
	
	\qanf 	{Version-Controlling}
			{Verwaltung und Version-Controlling des Quellcodes mit Hilfe eines Git basierten Version-Controlling Tool.}
			{<Begründung>}
			{-}
			{<Prio>}
			{<Akteure>}
	
	\qanf 	{Continuous Integration}
			{<Beschreibung>}
			{<Begründung>}
			{-}
			{<Prio>}
			{<Akteure>}
	
	\qanf 	{Statische Codeanalyse}
			{Mit Hilfe des Tools 'SonarQube' bzw. 'SonarCloud' soll der gesamt Quellcode einer statischen Analyse unterzogen werden. Dabei darf die technische Codequalität von diesen Tool nicht schlechter als 'B' bewertet werden.}
			{<Begründung>}
			{-}
			{<Prio>}
			{<Akteure>}
	
	\qanf 	{Qualitätssicherung}
			{80 Prozent des Quellcodes der einzelnen Komponenten soll durch automatische Test auf Korrektheit geprüft werden}
			{<Begründung>}
			{-}
			{<Prio>}
			{<Akteure>}
	
	\qanf 	{Docker Container}
			{<Beschreibung>}
			{<Begründung>}
			{-}
			{<Prio>}
			{<Akteure>}
	
	\qanf 	{Dokumentation}
			{<Beschreibung>}
			{<Begründung>}
			{-}
			{<Prio>}
			{<Akteure>}
	
	\qanf 	{Benutzerhandbuch}
			{<Beschreibung>}
			{<Begründung>}
			{-}
			{<Prio>}
			{<Akteure>}
	
	\qanf 	{Anwendungssprache}
			{<Beschreibung>}
			{<Begründung>}
			{-}
			{<Prio>}
			{<Akteure>}
	
	\qanf 	{Programmiersprache}
			{Implementierungssprache}
			{<Begründung>}
			{-}
			{<Prio>}
			{<Akteure>}
	
	\qanf 	{Dokumentationssprache}
			{<Beschreibung>}
			{<Begründung>}
			{-}
			{<Prio>}
			{<Akteure>}
	
	\qanf 	{Format für Konfigurationsdateien}
			{<Beschreibung>}
			{<Begründung>}
			{-}
			{<Prio>}
			{<Akteure>}
	
	\qanf 	{Netzwerkkommunikation}
			{<Beschreibung>}
			{<Begründung>}
			{-}
			{<Prio>}
			{<Akteure>}
	
	\qanf 	{Rundenspiel}
			{<Beschreibung>}
			{<Begründung>}
			{-}
			{<Prio>}
			{<Akteure>}
	
	\qanf 	{<Titel>}
			{<Beschreibung>}
			{<Begründung>}
			{-}
			{<Prio>}
			{<Akteure>}
	
	\qanf 	{<Titel>}
			{<Beschreibung>}
			{<Begründung>}
			{-}
			{<Prio>}
			{<Akteure>}
	
	\qanf 	{<Titel>}
			{<Beschreibung>}
			{<Begründung>}
			{-}
			{<Prio>}
			{<Akteure>}
	
	\qanf 	{<Titel>}
			{<Beschreibung>}
			{<Begründung>}
			{-}
			{<Prio>}
			{<Akteure>}
	
	\qanf 	{<Titel>}
			{<Beschreibung>}
			{<Begründung>}
			{-}
			{<Prio>}
			{<Akteure>}
	
	\qanf 	{<Titel>}
			{<Beschreibung>}
			{<Begründung>}
			{-}
			{<Prio>}
			{<Akteure>}
	
	
\end{document}
