\documentclass[DIN, pagenumber=false, fontsize=11pt, parskip=half]{scrartcl}

\usepackage{amsmath}
\usepackage{amsfonts}
\usepackage{amssymb}
\usepackage{enumitem}
\usepackage[utf8]{inputenc} % this is needed for umlauts
\usepackage[ngerman]{babel} % this is needed for umlauts
\usepackage[T1]{fontenc} 
\usepackage{commath}
\usepackage{xcolor}
\usepackage{booktabs}
\usepackage{float}
\usepackage{tikz-timing}
\usepackage{tikz}
\usepackage{multirow}
\usepackage[final]{pdfpages}

\usetikzlibrary{calc,shapes.multipart,chains,arrows}

\newcommand{\begriff}[7] {
	\begin{table}[H]
		\centering
		\begin{tabular}{|p{4.5cm}|p{10cm}|}
			%\hline
			%\toprule \\
			\hline
			\textbf{Begriff} & \textbf{#1} \\ \hline
			%\midrule \\
			\textbf{Beschreibung} & #2 \\ \hline
			%\midrule
			\textbf{Ist-ein} & #3 \\ \hline
			%\midrule
			\textbf{Kann-sein} & #4 \\ \hline
			%\midrule
			\textbf{Aspekt} & #5 \\ \hline
			%\midrule
			\textbf{Bemerkung} & #6 \\ \hline
			%\midrule
			\textbf{Beispiel} & #7 \\ %\hline
			%\bottomrule
			\hline
		\end{tabular}
	\end{table}
}

\title{Softwaregrundprojekt}
\author{Gruppe 10}

\newcommand{\anf}[7] {
    \begin{table}[H]
        \centering
        \begin{tabular}{p{3.2cm}|p{9.8cm}}
            \textbf{ID:} & \textbf{#1} \\ \hline
            \textbf{Titel:} & #2 \\ \hline
            \textbf{Beschreibung:} & #3 \\ \hline
            \textbf{Begründung:} & #4 \\ \hline
            \textbf{Abhängigkeiten:} & #5 \\ \hline
            \textbf{Priorität:} & #6 \\ \hline
            \textbf{Akteure:} & #7 \\ \hline
        \end{tabular}
    \end{table}
}

\newcounter{fanfCount}
\newcommand{\fanf}[6] {
    \stepcounter{fanfCount}
    \anf{FA\thefanfCount}{#1}{#2}{#3}{#4}{#5}{#6}
}
\newcounter{qanfCount}
\newcommand{\qanf}[6] {
    \stepcounter{qanfCount}
    \anf{QA\theqanfCount}{#1}{#2}{#3}{#4}{#5}{#6}
}

\newcommand{\akt}[4] {
    \begin{table}[H]
        \centering
        \begin{tabular}{p{3cm}|p{10cm}}
            \textbf{ID:} & \textbf{#1} \\ \hline
            \textbf{Titel:} & #2 \\ \hline
            \textbf{Beschreibung:} & #3 \\ \hline
            \textbf{Rolle:} & #4 \\ \hline
        \end{tabular}
    \end{table}
}

\newcounter{faktCount}
\newcommand{\fakt}[3] {
    \stepcounter{faktCount}
    \akt{AKT\thefaktCount}{#1}{#2}{#3}
}

\begin{document}
    \maketitle
    \section{Kontextanalyse}
    %TODO
    \section{Fachwissen}
    \begriff{Nutzer}
{Ein Mensch, der einen Rechner bedient und entweder den Client zum Spielen des Spiels oder zur Beobachtung einer Partie benutzt, oder den Team-Editor bedient. Jeder Benutzer hat einen Nutzernamen, mittels dem er von anderen Nutzern erkannt werden kann.}
{-}
{Spieler, Gast}
{Zur Beschreibung des Programmverlaufs}
{-}
{JägerMaister69}

\begriff{Spieler}
{Ein Nutzer, der das Computerspiel \glqq{}Fantastic Feasts\grqq{}  spielt.}
{Nutzer}
{-}
{Zur Beschreibung des Programmverlaufs}
{-}
{-}

\begriff{Gast}
{Ein Nutzer, der eine laufende Partie beobachtet}
{Nutzer}
{-}
{Zur Beschreibung des Programmverlaufs}
{-}
{-}

\begriff{Client}
{Das Computerprogramm, das mit einer grafischen Oberfläche ausgestattet ist und einem Nutzer erlaubt, eine Verbindung mit einem Server herzustellen und damit zu Kommunizieren}
{-}
{-}
{Zum Spielen des Spiels \glqq{}Fantastic Feasts\grqq{} }
{Der Begriff bezieht sich nicht auf den Menschen, der das Programm bedient}
{-}

\begriff{Server}
{Die zentrale Komponente, in dem die Spiellogik implementiert ist und die Programmbefehle abwickelt und mit dem sich Clients verbinden können, um eine Partie zu spielen oder zu beobachten. Die Kommunikation erfolgt mit JSON}
{-}
{-}
{Ist für die Kommunikation von Clients, für das Verwalten des Spielgeschehens, Ressourcenverwaltung und die Spiellogik verantwortlich.}
{-}
{-}

\begriff{Team-Editor}
{Ermöglicht einem Nutzer mit einer grafischen Oberfläche, ein eigenes Team zu erstellen und zu bearbeiten. Die Einstellungen werden danach als JSON-Datei gespeichert.}
{-}
{-}
{Zur Erstellung von nutzereigenen Teams.}
{-}
{-}

\begriff{KI-Client}
{Meldet sich beim Server wie ein normaler Client an und simuliert mit einer KI einen menschlichen Spieler. Hat keine grafische Oberfläche. Meldet sich mit dem Nutzernamen \glqq{}KI\grqq{}  ein.}
{-}
{-}
{Zum Spielen gegen einen Computergegner}
{-}
{-}

\begriff{KI}
{Definiert die Regeln, nach denen der KI-Client auf die durch den Server vermittelten Geschehen im Spiel reagiert.}
{-}
{-}
{Zum Spielen gegen einen Computergegner}
{Die KI ist die Logik, nach der der Computer das Spiel spielt und kein Programm}
{-}

\begriff{Spielfeld}
{Ein grafisch darstellbares Raster, auf dem sich die Spielfiguren bewegen}
{-}
{-}
{Dient als virtuelles Spielbrett mit klar definierten Abgrenzungen}
{Wird nicht Spielumgebung genannt um Verwechslung mit dem Client zu vermeiden}
{-}

\begriff{Zelle}
{Die kleinste Einheit des Spielfeldes, also ein Quadrat davon}
{-}
{Zentrumszelle, Torring, Hüterzonenzelle}
{Mögliche Standorte der Spielfiguren}
{Wird nicht Feld genannt, da das ein eher vager Begriff ist}
{-}

\begriff{Zentrum}
{Der 3x3 Zellen große Abschnitt in der Mitte des Spielfeldes}
{-}
{-}
{Summe aller Zentrumszellen}
{Ist das Mittelfeld im Lastenheft}
{-}

\begriff{Hüterzone}
{Die Bereiche am linken und rechten Rand des Spielfeldes, in dem sich die Torringe befinden}
{-}
{-}
{Summe aller kritischen Zellen und Torring}
{-}
{-}

\begriff{Torring}
{Die Zellen in die beide Teams den Quaffel bewegen wollen. Es wird zwischen eigenen und gegnerischen Torringen unterschieden.}
{Zelle}
{Eigener Torring, Gegnerischer Torring}
{Hauptquelle von Punkten}
{Torring im Lastenheft}
{}

\begriff{Zentrumszelle}
{Eine Zelle im Zentrum des Spielfeldes (siehe Zentrum)}
{Zelle}
{-}
{Startpunkt für Quaffel und Klatscher}
{-}
{-}

\begriff{Hüterzonenzelle}
{Eine Zelle in einem kritischen Bereich des Spielfeldes}
{Zelle}
{-}
{limitierendes Element für das Abliefern der Quaffel}
{-}
{-}

\begriff{Spielobjekt}
{Jedes Objekt, das sich auf dem Spielfeld befindet und darauf bewegt werden kann}
{-}
{Ball, Spielfigur}
{-}
{Nicht zu verwechseln mit Spielfigur}
{-}

\begriff{Ball}
{Ein Spielobjekt, das nicht direkt, nur indirekt von einem Spieler beeinflusst werden kann}
{Spielobjekt}
{Quaffel, Klatscher, Schnatz}
{Festpunkte zur Steuerung des Spielverlaufs}
{-}
{-}

\begriff{Spielfigur}
{Ein Spielobjekt, das von einem Spieler direkt gesteuert wird. Jede Spielfigur hat einen Namen, einen Besenrang, ein Geschlecht, ein Team und eine Rolle. Man unterscheidet außerdem zwischen eigenen und gegnerischen Spielfiguren.}
{Spielobjekt}
{Hüter, Sucher, Jäger, Treiber}
{Mitglieder eines Teams}
{Spieler im Lastenheft}
{Name: Luke Skywalker, Rolle: Hüter des Teams \glqq{}Jedi\grqq{}, Geschlecht: männlich, Besenrang: 5}

\begriff{Quaffel}
{Passives Objekt, mit dem Jäger und Hüter interagieren können und von ihnen nach Möglichkeit in ein gegnerisches Zielfeld befördert werden soll.}
{Ball}
{-}
{Zentrales Spielobjekt}
{-}
{-}

\begriff{Klatscher}
{Ball, der sich von selbst auf Spielfiguren zubewegt, die keine Treiber sind und diese betäuben können und von Treibern bewegt werden können.}
{Ball}
{-}
{Zusätzliches taktisches Spielelement}
{-}
{-}

\begriff{Schnatz}
{Ball, der von den Suchern gejagt wird und deren Fund die Partie beendet}
{Ball}
{-}
{Definiert Spielende}
{\glqq{}Schnatz\grqq{}  im Lastenheft}
{-}

\begriff{Partie}
{Ein einzelnes Spiel. Beginnt beim Platzieren der Figuren und endet mit dem Bestimmen des Gewinners.}
{-}
{-}
{Beschreibung des Spielablaufs}
{-}
{Spieler VodkaVodka98 spielt gegen Spieler LongEiländ}

\begriff{Hüter}
{Spielfigur, deren Aufgabe es ist, den Quaffel von den eigenen Zielfeldern fernzuhalten}
{Spielfigur}
{Eigener Hüter, Gegnerischer Hüter}
{Letzte Verteidigungslinie}
{-}
{Siehe \glqq{}Spielfigur\grqq{} }

\begriff{Sucher}
{Spielfigur, die den Schnatz jagt}
{Spielfigur}
{Eigener Sucher, Gegnerischer Sucher}
{Beendet die Partie}
{-}
{Darth Vader, gegnerischer Sucher, Besenrang 2}

\begriff{Jäger}
{Spielfigur, die den Quaffel in ein einen gegnerischen Torring befördern soll}
{Spielfigur}
{Eigener Jäger, Gegnerischer Jäger}
{Holt Punkte für das eigene Team}
{\glqq{}Jäger\grqq{}  im Lastenheft. Jäger beschreibt die Rolle der Spielfigur aber besser.}
{Han Solo, eigener Jäger, Besenrang 3}

\begriff{Treiber}
{Spielfigur, mit der der Spieler eigene Spielfiguren vor Klatschern schützt und gegnerische damit abschießen kann}
{Spielfigur}
{Eigener Treiber, Gegnerischer Treiber}
{Interagiert mit Klatschern}
{-}
{Boba Fett, gegnerischer Treiber, Besenrang 4}

\begriff{Geschlecht}
{Jede Spielfigur ist entweder männlich oder weiblich.}
{-}
{-}
{Team-Editierung}
{-}
{-}

\begriff{Team}
{Die Menge aller Spielfiguren auf dem Spielfeld, die von einem einzigen Spieler kontrolliert wird. Ein Team hat einen Namen, ein Motto, ein Logo, eine Hauptfarbe und eine Ersatzfarbe.}
{eigenes Team, gegnerisches Team}
{-}
{Beschreibung einer Partie}
{-}
{Galaktisches Imperium, Motto: \glqq{}Unbegrenzte MAAACHT!\grqq{}, [Todesstern als Logo], Hauptfarbe: Schwarz, Ersatzfarbe: Rot }

\begriff{Punkte}
{Der Spieler mit mehr Punkten am Ende der Partie gewinnt. Werden durch das Platzieren des Quaffel in einem gegnerischen Torring oder das Finden des Schnatzes erhalten.}
{-}
{-}
{Bestimmung des Gewinners}
{-}
{SchnapsNase hat 20 Punkte}

\begriff{Besetzen}
{Eine Spielfigur besetzt das Feld, auf dem sie sich befindet}
{-}
{-}
{Beschreibung des Spielgeschehens}
{Zwei Spielfiguren können sich nicht auf derselben Zelle befinden}
{Chewbacca besetzt Zelle 5:3}

\begriff{Besenrang}
{Jede Spielfigur hat einen Besenrang von 1 bis 5, der die Wahrscheinlichkeit bestimmt, dass sie noch einmal ziehen kann. Besenrang 1 ist der beste.}
{-}
{-}
{Unterscheidet Qualität der Spielfiguren.}
{Ersetzt die verschiedenen \glqq{}Besen\grqq{} aus dem Lastenheft mit einer Skala von 1 bis 5 zur besseren Übersicht.}
{Yoda hat Besenrang 1.}

\begriff{Aktion}
{Jede durch einen Spieler hervorgerufene Änderung der Spielsituation}
{-}
{Ziehen, Schießen, Schlagen, Einmischung, Übernahme}
{Weiterführung der Partie}
{-}
{Obi-Wan Kenobi zieht von Zelle 8:7 auf Zelle 9:7}

\begriff{Ziehen}
{Die Bewegung einer Spielfigur von einer Zelle auf eine andere durch direkten Befehl des Spielers}
{Aktion}
{-}
{Beschreibung des Spielverlaufs}
{Bezieht sich nicht auf erzwungene Bewegungen einer Spielfigur.}
{Obiwan Kenobi zieht von Zelle 8:7 auf Zelle 9:7}

\begriff{Befördern}
{Bewegen des Quaffel mittels einer Spielfigur}
{-}
{-}
{Bewegen des Quaffel, allgemeiner Begriff}
{Keine Aktion, da eventuell eine passive Folge, z.B. durch Ziehen}
{-}

\begriff{Schießen}
{Die Bewegung des Quaffel durch einen Hüter oder Jäger auf eine andere, entfernte Zelle ohne Bewegung der Spielfigur}
{Aktion}
{-}
{Bewegung des Quaffel um mehrere Felder}
{\glqq{}Werfen\grqq{}  im Lastenheft. Analog zum Schussvektor benannt.}
{Mace Windu schießt den Quaffel auf Zelle 10:4}

\begriff{Schlagen}
{Die erzwungene Bewegung eines Klatschers durch einen Treiber}
{Aktion}
{-}
{Interaktion mit Klatschern}
{\glqq{}Kloppen\grqq{}  im Lastenheft}
{R2-D2 schlägt einen Klatscher auf Zelle 5:10}

\begriff{Einmischung}
{Hilfsfähigkeiten, die nicht von Spielobjekten ausgehen. Werden von einem Spieler gesteuert. Bei jeder Benutzung besteht eine Chance, dass die verwendete Einmischung bis zum Ende der Partie für den jeweiligen Spieler vom Schiedsrichter deaktiviert werden.}
{Aktion}
{Teleportation, Fernangriff, Impuls, Schnatzjagd}
{Zusätzliche taktische Element}
{Ersetzt die \glqq{}Fans\grqq{}  aus dem Lastenheft}
{Lando Calrissian wird auf Zelle 6:6 teleportiert}

\begriff{Teleportation}
{Einmischung, die eine Spielfigur auf eine zufällige Zelle teleportiert}
{Einmischung}
{-}
{-}
{Ersetzt \glqq{}Elfen\grqq{}  aus Lastenheft}
{Siehe \glqq{}Einmischungen\grqq{} }

\begriff{Fernangriff}
{Trifft eine gegnerische Spielfigur. Ziel verliert gegebenenfalls den Quaffel und wird auf eine zufällige benachbarte, freie Zelle bewegt.}
{Einmischung}
{-}
{-}
{Statt \glqq{}Kobolde\grqq{}  im Lastenheft}
{Jango Fett wird von Fernangriff auf Zelle 5:6 gestoßen}

\begriff{Impuls}
{Wenn eine Spielfigur den Quaffel hält, wird sie bei Benutzung verloren}
{Einmischung}
{-}
{-}
{Statt \glqq{}Trolle\grqq{}  im Lastenheft}
{C-3PO verliert wegen eines Impuls den Quaffel}

\begriff{Schnatzstoß}
{Bewegt den Schnatz in eine zufällige Richtung um ein Feld}
{Einmischung}
{-}
{-}
{\glqq{}Schnatzschnappen\grqq{}  im Lastenheft}
{Ein Schnatzstoß treibt den Schnatz auf Zelle 4:12}

\begriff{Entfernung}
{Eine Entfernung zwischen zwei Zellen ist die minimale Anzahl von Zügen, in denen eine Spielfigur von der einen auf die andere ziehen kann.}
{-}
{-}
{Spielfeldgeometrie}
{-}
{-}

\begriff{Schussvektor}
{Pfeil vom Mittelpunkt einer Zelle zum Mittelpunkt einer anderen}
{-}
{Torschussvektor}
{Spielfeldgeometrie}
{-}
{-}

\begriff{Torschussvektor}
{Schussvektor zu einem Schuss, der möglicherweise in einem Torschuss resultiert. (Ein Schussvektor, der die linke oder rechte Seite eines Torrings schneidet.)}
{Schussvektor}
{-}
{Punkte sammeln}
{-}
{-}

\begriff{Torschuss}
{Ein Jäger schießt den Quaffel in einen Torring und holt damit Punkte für sei Team}
{-}
{-}
{Punkte sammeln}
{Nur erfolgreiche Schüsse auf das Tor werden als Torschüsse bezeichnet.}
{Darth Sidious schießt den Quaffel in ein eigenes Tor}

\begriff{Zugphase}
{Phase, in der eine Spielfigur Aktionen durchführt. Beginnt, sobald der Spieler die Möglichkeit hat, die jeweilige Spielfigur zu steuern und endet, sobald er ihr den letzten Befehl für diesen Zug gegeben hat. Ein Zug enthält mehrere Zugphasen.}
{-}
{-}
{Zeiteinteilung}
{}
{Leia Organa ist dran.}

\begriff{Zug}
{Von der ersten Aktion eines Spielers bis zur ersten Aktion des Gegners}
{-}
{-}
{Zeiteinteilung}
{Nicht die Zugphase einer Spielfigur}
{Bierdurst69 ist am Zug}

\begriff{Endphase}
{Letzter Teil eines Zuges. Der Spieler kann darin Einmischungen vornehmen.}
{-}
{-}
{Zeiteinteilung}
{-}
{-}

\begriff{Verlieren}
{Der Quaffel wird auf eine zufällige Zelle bewegt, die an die Zelle angrenzt, auf der sich die Spielfigur, die sie derzeit hält befindet.}
{-}
{-}
{Spielablauf}
{\glqq{}Vertändeln\grqq{}  im Lastenheft}
{Jar Jar verliert den Quaffel}

\begriff{Halten}
{Ein Jäger oder Hüter kann den Quaffel halten. Ist das der Fall, bewegt sich der Quaffel auf die Zelle, auf die die Spielfigur zieht.}
{-}
{-}
{Beschreibung des Spielgeschehens}
{-}
{-}

\begriff{Übernahme}
{Ein Jäger neben einer gegnerischen Spielfigur, die den Quaffel hält, kann diesen mit einer bestimmten Wahrscheinlichkeit übernehmen und hält sie anschließend selbst.}
{Aktion}
{-}
{Aggressives Spielmanöver}
{-}
{Darth Vader übernimmt den Quaffel von Anakin Skywalker}

\begriff{Betäubt}
{Eine betäubte Spielfigur kann in seiner nächsten Zugphase keine Aktion durchführen}
{-}
{-}
{Wirkung der Klatscher}
{\glqq{}Ausgeknockt\grqq{}  im Lastenheft}
{Jango Fett ist betäubt}

\begriff{Foul}
{Handlung, wegen der eine Spielfigur vorübergehend vom Spielfeld entfernt werden kann.}
{-}
{Torring Blockieren, Stürmen, Großoffensive, Rammen, Schnatz Blockieren}
{Taktische Elemente}
{-}
{Qui-Gon Jinn blockiert den Schnatz}

\begriff{Torring Blockieren}
{Eine eigene Spielfigur besetzt einen eigenen Torring, was eine Torschuss verhindert.}
{Foul}
{-}
{Taktik}
{\glqq{}Flackern\grqq{}  im Pflichtenheft}
{-}

\begriff{Stürmen}
{Ein Jäger, der den Quaffel hält, zieht auf einen gegnerischen Torring, was das Abliefern garantiert.}
{Foul}
{-}
{Taktik}
{\glqq{}Nachtarocken\grqq{}  im Lastenheft}
{Han Solo stürmt mittleren gegnerischen Torring}

\begriff{Großoffensive}
{Ein eigener Jäger betritt eine gegnerische Hüterzonenzelle während ein anderer eigener Jäger sich auf einer anderen befindet.}
{Foul}
{-}
{Taktik}
{\glqq{}Stutschen\grqq{}  im Lastenheft}
{Lando Calrissia schließt sich Chewbacca in einer Großoffensive an}

\begriff{Rammen}
{Eine eigene Spielfigur zieht auf eine Zelle, die von einer gegnerischen Spielfigur besetzt wird. Dadurch wird die gegnerische Spielfigur auf eine benachbarte Zelle bewegt und verliert den Quaffel}
{Foul}
{-}
{Taktik}
{\glqq{}Keilen\grqq{}  im Lastenheft}
{Boba Fett rammt Jar Jar}

\begriff{Schnatz blockieren}
{Eine Spielfigur, die kein Sucher ist, besetzt die Zelle, auf der sich der Schnatz befindet.}
{Foul}
{-}
{Taktik}
{\glqq{}Schnatzeln\grqq{}  im Lastenheft}
{Darth Maul blockiert den Schnatz}

\begriff{Schiedsrichter}
{Entfernt mit bestimmter Wahrscheinlichkeit eine Spielfigur, die ein Foul ausführt vom Spielfeld bis ein Torschuss erfolgt und deaktiviert permanent eine Einmischung für den Rest der Partie.}
{-}
{-}
{Taktik}
{\glqq{}Schiedsrichter\grqq{}  im Lastenheft}
{Sheev Palpatine wurde vom Schiedsrichter vom Spielfeld entfernt}

\begriff{Disqualifikation}
{Tritt ein wenn fünf Spielfiguren eines Spielers gleichzeitig durch den Schiedsrichter aus dem Spiel entfernt sind. Führt zur Niederlage des Spielers.}
{-}
{-}
{Erhöhtes Risiko}
{-}
{CubaLibre wurde disqualifiziert. CaptainCola gewinnt die Partie.}

    %\includepdf[pagecommand={\section{Domänenmodell}}]{domaenenmodell.pdf}
    \section{Anforderungsdefinition}
    %TODO
    
    \subsection{Akteure}
    
    \fakt	{Anwendungsbenutzer}
			{Menschlicher Benutzer, der eine Anwendungen bedient.}
			{<Rolle>}
			
	\fakt	{Spielender Benutzer}
			{Anwendungsbenutzer, der mit der Client-Anwendung am tatsächlichen Spielgeschehen teilnimmt.}
			{<Rolle>}
			
	\fakt	{Anwendungsbenutzer, der mit der Client-Anwendung ein laufendes Spiel beobachtet.}
			{<Beschreibung>}
			{<Rolle>}
			
	\fakt	{Systemadministrator}
			{Anwendungsnutzer mit Zusatzqualifikation um Server zu verwalten.}
			{Der Systemadministrator hat zugriff auf die Server Anwendung. Er ist dafür verantwortlich eine Instanz der Server Anwendung zu starten und zu betreuen. Zudem hat er Zugriff auf die Partie-Konfiguration und kann diese bei bedarf verändern.}
	
	\fakt	{KI}
			{Spielender Benutzer, welcher von einem Computerprogramm gesteuert wird.}
			{<Rolle>}
	
	\fakt	{Kunde}
			{SoPra-Tutor}
			{<Rolle>}
	
	\fakt	{Client [Komponente]}
			{Anwendung, mit der ein Anwendungsbenutzer aktiv an einem Spiel teilnehmen kann oder ein Spiel beobachten kann.}
			{<Rolle>}
	
	\fakt	{KI-Client [Komponente]}
			{Anwendung, die als KI an einem Spiel teilnimmt.}
			{<Rolle>}
	
	\fakt	{Server [Komponente]}
			{Anwendung, auf der die Spiellogik zur Verfügung gestellt wird und mit der sich Clients über das Netzwerk verbinden.}
			{<Rolle>}
			
	\fakt	{Quidditchteam-Editor [Komponente]}
			{Anwendung, mit der ein Quidditchteam-Konfiguration erstellt und bearbeitet werden kann.}
			{<Rolle>}
	
	\fakt 	{Team}
			{<Beschreibung>}
			{<Rolle>}
	
	\fakt 	{Schiedsrichter}
			{<Beschreibung>}
			{<Rolle>}
	
	
	\subsection{Funktionale Anforderungen (allgemein)}
	
	\fanf	{Quidditch-Spielfeld}
			{Das Quidditch-Spielfeld ist eine Ovale Form, welche in ein Raster von 17x13 quadratischen Feldern eingepasst ist. Auf diesem Feld finden alle Spielhandlungen statt, welche während dem Spiel getätigt werden können. Auf dem Spielfeld gibt es noch ein Mittelkreis und an den jeweils gegenüberliegenden Enden noch Hüterzonen.}
			{Diese Anforderung geht aus den im Lastenheft zu Verfügung gestellten Spielregeln für das Spiel \textit{Fantastic Feasts} hervor.}
			{<Abhängigkeiten>}
			{++}
			{\textbf{ALLE?}}
	
	\fanf	{Mittelkreis}
			{Der Mittelkreis ist ein Bereich auf dem Quidditch-Spielfeld, welcher in der Mitte angeordnet ist und aus 3x3 quadratischen Kacheln besteht. In dem Mittelkreis befindet sich das Mittelfeld, welches die mittlere Kachel des Mittelkreises ist und durch ein \textit{M} gekennzeichnet ist.}
			{Diese Anforderung geht aus den im Lastenheft zu Verfügung gestellten Spielregeln für das Spiel \textit{Fantastic Feasts} hervor.}
			{FA1}
			{++}
			{<Akteure>}
	
	\fanf	{Mittelfeld}
			{Das Mittelfeld stellt den mittleren Punkt des Mittelkreises dar, welcher im Zentrum des Quidditch-Spielfeldes ist.}
			{Diese Anforderung geht aus den im Lastenheft zu Verfügung gestellten Spielregeln für das Spiel \textit{Fantastic Feasts} hervor.}
			{FA2}
			{++}
			{<Akteure>}
	
	\fanf	{Hüterzone}
			{Die Hüterzonen sind jeweils an den jeweils gegenüberliegenden Seiten des Quidditch-Spielfeldes. Die Hüterzonen beinhalten jeweils 3 Torringe und sind somit die Zonen in den die Teams Punkten können. Die Hüterzonen sind in einer Ovalen Form, welche 11 quadratische Felder hoch ist und 5 quadratische Felder breit ist an den jeweilgen Maximalen Punkten.}
			{Diese Anforderung geht aus den im Lastenheft zu Verfügung gestellten Spielregeln für das Spiel \textit{Fantastic Feasts} hervor. Außerdem beinhaltet diese Zone die Torringe, durch welche die Teams Punkte machen können.}
			{FA1}
			{++}
			{<Akteure>}
	
	\fanf	{Torring}
			{Die Torringe sind in der Hüterzone angebracht und dazu da, dass die Teams jeweils Punkten können. Die 3 torringe werden jeweils von einem Torhüter bewacht, welcher verhindern kann, dass das gegnerische Team ein Tor schiessen kann.}
			{Diese Anforderung geht aus den im Lastenheft zu Verfügung gestellten Spielregeln für das Spiel \textit{Fantastic Feasts} hervor.}
			{FA4}
			{++}
			{<Akteure>}
	
	\fanf	{Schussvektor}
			{<Beschreibung>}
			{Diese Anforderung geht aus den im Lastenheft zu Verfügung gestellten Spielregeln für das Spiel \textit{Fantastic Feasts} hervor.}
			{<Abhängigkeiten>}
			{<Prio>}
			{<Akteure>}
	
	\fanf	{Bälle}
			{<Beschreibung>}
			{Diese Anforderung geht aus den im Lastenheft zu Verfügung gestellten Spielregeln für das Spiel \textit{Fantastic Feasts} hervor.}
			{<Abhängigkeiten>}
			{<Prio>}
			{<Akteure>}
	
	\fanf	{Quaffel [Ball]}
			{<Beschreibung>}
			{Diese Anforderung geht aus den im Lastenheft zu Verfügung gestellten Spielregeln für das Spiel \textit{Fantastic Feasts} hervor.}
			{<Abhängigkeiten>}
			{<Prio>}
			{<Akteure>}
	
	\fanf	{Klatscher [Ball]}
			{<Beschreibung>}
			{Diese Anforderung geht aus den im Lastenheft zu Verfügung gestellten Spielregeln für das Spiel \textit{Fantastic Feasts} hervor.}
			{<Abhängigkeiten>}
			{<Prio>}
			{<Akteure>}
	
	\fanf	{Goldener Schnatz [Ball]}
			{<Beschreibung>}
			{Diese Anforderung geht aus den im Lastenheft zu Verfügung gestellten Spielregeln für das Spiel \textit{Fantastic Feasts} hervor.}
			{<Abhängigkeiten>}
			{<Prio>}
			{<Akteure>}
	
	\fanf	{Besen}
			{<Beschreibung>}
			{Diese Anforderung geht aus den im Lastenheft zu Verfügung gestellten Spielregeln für das Spiel \textit{Fantastic Feasts} hervor.}
			{<Abhängigkeiten>}
			{<Prio>}
			{<Akteure>}
	
	\fanf	{Zauberfauch [Besen]}
			{<Beschreibung>}
			{Diese Anforderung geht aus den im Lastenheft zu Verfügung gestellten Spielregeln für das Spiel \textit{Fantastic Feasts} hervor.}
			{<Abhängigkeiten>}
			{<Prio>}
			{<Akteure>}
	
	\fanf	{Sauberwisch 11 [Besen]}
			{<Beschreibung>}
			{Diese Anforderung geht aus den im Lastenheft zu Verfügung gestellten Spielregeln für das Spiel \textit{Fantastic Feasts} hervor.}
			{<Abhängigkeiten>}
			{<Prio>}
			{<Akteure>}
	
	\fanf	{Komet 2-60 [Besen]}
			{<Beschreibung>}
			{Diese Anforderung geht aus den im Lastenheft zu Verfügung gestellten Spielregeln für das Spiel \textit{Fantastic Feasts} hervor.}
			{<Abhängigkeiten>}
			{<Prio>}
			{<Akteure>}
	
	\fanf	{Nimbus 2001  [Besen]}
			{<Beschreibung>}
			{Diese Anforderung geht aus den im Lastenheft zu Verfügung gestellten Spielregeln für das Spiel \textit{Fantastic Feasts} hervor.}
			{<Abhängigkeiten>}
			{<Prio>}
			{<Akteure>}
	
	\fanf	{Feuerblitz [Besen]}
			{<Beschreibung>}
			{Diese Anforderung geht aus den im Lastenheft zu Verfügung gestellten Spielregeln für das Spiel \textit{Fantastic Feasts} hervor.}
			{<Abhängigkeiten>}
			{<Prio>}
			{<Akteure>}
	
	\fanf	{Spieler}
			{<Beschreibung>}
			{Diese Anforderung geht aus den im Lastenheft zu Verfügung gestellten Spielregeln für das Spiel \textit{Fantastic Feasts} hervor.}
			{<Abhängigkeiten>}
			{<Prio>}
			{<Akteure>}	
	
	\fanf	{Jäger [Spielertyp]}
			{<Beschreibung>}
			{Diese Anforderung geht aus den im Lastenheft zu Verfügung gestellten Spielregeln für das Spiel \textit{Fantastic Feasts} hervor.}
			{<Abhängigkeiten>}
			{<Prio>}
			{<Akteure>}
	
	\fanf	{Treiber [Spielertyp]}
			{<Beschreibung>}
			{Diese Anforderung geht aus den im Lastenheft zu Verfügung gestellten Spielregeln für das Spiel \textit{Fantastic Feasts} hervor.}
			{<Abhängigkeiten>}
			{<Prio>}
			{<Akteure>}
	
	\fanf	{Hüter [Spielertyp]}
			{<Beschreibung>}
			{Diese Anforderung geht aus den im Lastenheft zu Verfügung gestellten Spielregeln für das Spiel \textit{Fantastic Feasts} hervor.}
			{<Abhängigkeiten>}
			{<Prio>}
			{<Akteure>}
	
	\fanf	{[Spielertyp] Sucher}
			{<Beschreibung>}
			{Diese Anforderung geht aus den im Lastenheft zu Verfügung gestellten Spielregeln für das Spiel \textit{Fantastic Feasts} hervor.}
			{<Abhängigkeiten>}
			{<Prio>}
			{<Akteure>}
	
	\fanf	{Fans]}
			{<Beschreibung>}
			{Diese Anforderung geht aus den im Lastenheft zu Verfügung gestellten Spielregeln für das Spiel \textit{Fantastic Feasts} hervor.}
			{<Abhängigkeiten>}
			{<Prio>}
			{<Akteure>}	
	
	\fanf	{Elfen [Fantyp]}
			{<Beschreibung>}
			{Diese Anforderung geht aus den im Lastenheft zu Verfügung gestellten Spielregeln für das Spiel \textit{Fantastic Feasts} hervor.}
			{<Abhängigkeiten>}
			{<Prio>}
			{<Akteure>}
	
	\fanf	{Kobolde [Fantyp]}
			{<Beschreibung>}
			{Diese Anforderung geht aus den im Lastenheft zu Verfügung gestellten Spielregeln für das Spiel \textit{Fantastic Feasts} hervor.}
			{<Abhängigkeiten>}
			{<Prio>}
			{<Akteure>}
	
	\fanf	{Trolle [Fantyp]}
			{<Beschreibung>}
			{Diese Anforderung geht aus den im Lastenheft zu Verfügung gestellten Spielregeln für das Spiel \textit{Fantastic Feasts} hervor.}
			{<Abhängigkeiten>}
			{<Prio>}
			{<Akteure>}
	
	\fanf	{Niffler [Fantyp]}
			{<Beschreibung>}
			{Diese Anforderung geht aus den im Lastenheft zu Verfügung gestellten Spielregeln für das Spiel \textit{Fantastic Feasts} hervor.}
			{<Abhängigkeiten>}
			{<Prio>}
			{<Akteure>}
	
	\fanf	{Foul}
			{<Beschreibung>}
			{Diese Anforderung geht aus den im Lastenheft zu Verfügung gestellten Spielregeln für das Spiel \textit{Fantastic Feasts} hervor.}
			{<Abhängigkeiten>}
			{<Prio>}
			{<Akteure>}
	
	\fanf	{Flacken [Foul]}
			{<Beschreibung>}
			{Diese Anforderung geht aus den im Lastenheft zu Verfügung gestellten Spielregeln für das Spiel \textit{Fantastic Feasts} hervor.}
			{<Abhängigkeiten>}
			{<Prio>}
			{<Akteure>}
	
	\fanf	{Nachtarocken [Foul]}
			{<Beschreibung>}
			{Diese Anforderung geht aus den im Lastenheft zu Verfügung gestellten Spielregeln für das Spiel \textit{Fantastic Feasts} hervor.}
			{<Abhängigkeiten>}
			{<Prio>}
			{<Akteure>}
	
	\fanf	{Stutschen [Foul]}
			{<Beschreibung>}
			{Diese Anforderung geht aus den im Lastenheft zu Verfügung gestellten Spielregeln für das Spiel \textit{Fantastic Feasts} hervor.}
			{<Abhängigkeiten>}
			{<Prio>}
			{<Akteure>}
	
	\fanf	{Keilen [Foul]}
			{<Beschreibung>}
			{Diese Anforderung geht aus den im Lastenheft zu Verfügung gestellten Spielregeln für das Spiel \textit{Fantastic Feasts} hervor.}
			{<Abhängigkeiten>}
			{<Prio>}
			{<Akteure>}
	
	\fanf	{Schnaltzeln [Foul]}
			{<Beschreibung>}
			{Diese Anforderung geht aus den im Lastenheft zu Verfügung gestellten Spielregeln für das Spiel \textit{Fantastic Feasts} hervor.}
			{<Abhängigkeiten>}
			{<Prio>}
			{<Akteure>}
	
	\fanf	{Wurf mit dem Quaffel}
			{<Beschreibung>}
			{Diese Anforderung geht aus den im Lastenheft zu Verfügung gestellten Spielregeln für das Spiel \textit{Fantastic Feasts} hervor.}
			{<Abhängigkeiten>}
			{<Prio>}
			{<Akteure>}
	
	\fanf	{Quaffel Abfangen}
			{<Beschreibung>}
			{Diese Anforderung geht aus den im Lastenheft zu Verfügung gestellten Spielregeln für das Spiel \textit{Fantastic Feasts} hervor.}
			{<Abhängigkeiten>}
			{<Prio>}
			{<Akteure>}
	
	\fanf	{Torschuss}
			{<Beschreibung>}
			{Diese Anforderung geht aus den im Lastenheft zu Verfügung gestellten Spielregeln für das Spiel \textit{Fantastic Feasts} hervor.}
			{<Abhängigkeiten>}
			{<Prio>}
			{<Akteure>}
	
	\fanf	{Klatscher kloppen}
			{<Beschreibung>}
			{Diese Anforderung geht aus den im Lastenheft zu Verfügung gestellten Spielregeln für das Spiel \textit{Fantastic Feasts} hervor.}
			{<Abhängigkeiten>}
			{<Prio>}
			{<Akteure>}
			
	\fanf	{Partie-Konfiguration}
			{<Beschreibung>}
			{<Begründung>}
			{<Abhängigkeiten>}
			{<Prio>}
			{<Akteure>}
	
	\fanf	{Quidditchteam-Konfiguration}
			{<Beschreibung>}
			{<Begründung>}
			{<Abhängigkeiten>}
			{<Prio>}
			{<Akteure>}	
			
	\fanf	{Netzwerkinterface}
			{<Beschreibung>}
			{<Begründung>}
			{<Abhängigkeiten>}
			{<Prio>}
			{<Akteure>}
	
	\fanf	{Runde}
			{<Beschreibung>}
			{Diese Anforderung geht aus den im Lastenheft zu Verfügung gestellten Spielregeln für das Spiel \textit{Fantastic Feasts} hervor.}
			{<Abhängigkeiten>}
			{<Prio>}
			{<Akteure>}
	
	\fanf	{Ballphase}
			{<Beschreibung>}
			{Diese Anforderung geht aus den im Lastenheft zu Verfügung gestellten Spielregeln für das Spiel \textit{Fantastic Feasts} hervor.}
			{<Abhängigkeiten>}
			{<Prio>}
			{<Akteure>}
			
	\fanf	{Spielerphase}
			{<Beschreibung>}
			{Diese Anforderung geht aus den im Lastenheft zu Verfügung gestellten Spielregeln für das Spiel \textit{Fantastic Feasts} hervor.}
			{<Abhängigkeiten>}
			{<Prio>}
			{<Akteure>}
			
	\fanf	{Fanphase}
			{<Beschreibung>}
			{Diese Anforderung geht aus den im Lastenheft zu Verfügung gestellten Spielregeln für das Spiel \textit{Fantastic Feasts} hervor.}
			{<Abhängigkeiten>}
			{<Prio>}
			{<Akteure>}
			
	\fanf	{Zufallsgenerator}
			{<Beschreibung>}
			{<Begründung>}
			{<Abhängigkeiten>}
			{<Prio>}
			{<Akteure>}
			
	\fanf	{Spielende}
			{<Beschreibung>}
			{<Begründung>}
			{<Abhängigkeiten>}
			{<Prio>}
			{<Akteure>}
			
	\fanf	{Überlängenbehnadlung}
			{<Beschreibung>}
			{<Begründung>}
			{<Abhängigkeiten>}
			{<Prio>}
			{<Akteure>}
			
	\fanf	{Spiellogik}
			{<Beschreibung>}
			{<Begründung>}
			{<Abhängigkeiten>}
			{<Prio>}
			{<Akteure>}
	
	\fanf	{Log-Datei}
			{<Beschreibung>}
			{<Begründung>}
			{<Abhängigkeiten>}
			{<Prio>}
			{<Akteure>}
	
	\fanf	{<Titel>}
			{<Beschreibung>}
			{<Begründung>}
			{<Abhängigkeiten>}
			{<Prio>}
			{<Akteure>}
	
	\fanf	{<Titel>}
			{<Beschreibung>}
			{<Begründung>}
			{<Abhängigkeiten>}
			{<Prio>}
			{<Akteure>}
	
	\fanf	{<Titel>}
			{<Beschreibung>}
			{<Begründung>}
			{<Abhängigkeiten>}
			{<Prio>}
			{<Akteure>}
	
	\fanf	{<Titel>}
			{<Beschreibung>}
			{<Begründung>}
			{<Abhängigkeiten>}
			{<Prio>}
			{<Akteure>}
	
	\fanf	{<Titel>}
			{<Beschreibung>}
			{<Begründung>}
			{<Abhängigkeiten>}
			{<Prio>}
			{<Akteure>}
			
	
	\subsection{Funktionale Anforderungen (Server)}
	
	\subsection{Funktionale Anforderungen (Client)}
	
	\fanf	{Hauptmenü [Client-View]}
			{<Beschreibung>}
			{<Begründung>}
			{<Abhängigkeiten>}
			{<Prio>}
			{<Akteure>}
	
	\fanf	{Connect to Game [Client-View]}
			{<Beschreibung>}
			{<Begründung>}
			{<Abhängigkeiten>}
			{<Prio>}
			{<Akteure>}
	
	\fanf	{End of Game [Client-View]}
			{<Beschreibung>}
			{<Begründung>}
			{<Abhängigkeiten>}
			{<Prio>}
			{<Akteure>}
			
	\fanf	{Import Team Config [Client-View]}
			{<Beschreibung>}
			{<Begründung>}
			{<Abhängigkeiten>}
			{<Prio>}
			{<Akteure>}
			
	\fanf	{Game Play [Client-View]}
			{<Beschreibung>}
			{<Begründung>}
			{<Abhängigkeiten>}
			{<Prio>}
			{<Akteure>}	
			
	\fanf	{Hilfe [Client-View]}
			{<Beschreibung>}
			{<Begründung>}
			{<Abhängigkeiten>}
			{<Prio>}
			{<Akteure>}
			
	\fanf	{Beobachter [Client-View]}
			{<Beschreibung>}
			{<Begründung>}
			{<Abhängigkeiten>}
			{<Prio>}
			{<Akteure>}
	
	\fanf	{Game Rendering Engine}
			{<Beschreibung>}
			{<Begründung>}
			{<Abhängigkeiten>}
			{<Prio>}
			{<Akteure>}
	
	\fanf	{Input Handler [Client]}
			{<Beschreibung>}
			{<Begründung>}
			{<Abhängigkeiten>}
			{<Prio>}
			{<Akteure>}
			
	\fanf	{Input Validierung [Client]}
			{<Beschreibung>}
			{<Begründung>}
			{<Abhängigkeiten>}
			{<Prio>}
			{<Akteure>}
	
	\fanf	{Hotkey [Client]}
			{<Beschreibung>}
			{<Begründung>}
			{<Abhängigkeiten>}
			{--}
			{<Akteure>}
			
	\fanf	{Pausieren [Client]}
			{<Beschreibung>}
			{<Begründung>}
			{<Abhängigkeiten>}
			{--}
			{<Akteure>}
			
	
	\subsection{Funktionale Anforderungen (Quidditchteam-Editor)}
	
	\fanf	{Edit Team [Team-Editor-View]}
			{<Beschreibung>}
			{<Begründung>}
			{<Abhängigkeiten>}
			{<Prio>}
			{<Akteure>}
	
	\fanf	{Save Team [Team-Editor-View]}
			{Datei Speicher Dialog um die Team-Konfiguration an einem beliebigen Ort im Datei System ab zu legen.}
			{<Begründung>}
			{<Abhängigkeiten>}
			{<Prio>}
			{<Akteure>}
			
	
	\subsection{Nicht funktionale Anforderungen}
	
	\qanf 	{Plattformunabhängigkeit}
			{Der Spielclient soll auf mindestens einer gängigen Computerbetriebssystem-Plattform (z.B. Linux, Windows) uneingeschränkt benutzbar sein. Des weiteren soll die Server- und die KI-Komponente auf mindestens zwei gängigen Computerbetriebssystem-Plattformen (z.B. Linux, Windows) uneingeschränkt benutzbar sein.}
			{Die Plattformunabhängigkeit ist im Lastenheft gefordert.}
			{<Abhängigkeiten>}
			{++}
			{<Akteure>}
	
	\qanf 	{Version-Controlling}
			{Der Quellcodes soll mit Hilfe eines Git basierten Version-Controlling Tool verwaltet werden.}
			{<Begründung>}
			{-}
			{++}
			{-}
	
	\qanf 	{Continuous Integration}
			{In die Version-Controlling Lösung mit Hilfe einer CI automatisch jeder gepushte Commit Unit-Tests und der Statischen Codeanalyse unterzogen werden. Zudem soll eine automatisierte Code Dokumentation angestoßen werden. Bei erfolgreichem Abschließen alle Test soll zum Schluss der aktuelle Stand deployed werden.}
			{<Begründung>}
			{-}
			{<Prio>}
			{<Akteure>}
	
	\qanf 	{Statische Codeanalyse}
			{Mit Hilfe des Tools 'SonarQube' bzw. 'SonarCloud' soll der gesamt Quellcode einer statischen Analyse unterzogen werden. Dabei darf die technische Codequalität von diesen Tool nicht schlechter als 'B' bewertet werden.}
			{}
			{-}
			{<Prio>}
			{<Akteure>}
	
	\qanf 	{Automatisierte Unit-Tests}
			{80 Prozent des Quellcodes der einzelnen Komponenten soll durch automatische Unit-Test auf Korrektheit und Funktion geprüft werden}
			{Da alle Komponenten möglichst fehlerfrei funktionieren müssen ist es unerlässlich die einzelnen Teil der Software ständig auf ihre Funktionalität zu prüfen, die ist nur effizient möglich wenn automatisiert Test durchgeführt werden, damit Fehler frühzeitig erkannt werden.}
			{-}
			{+}
			{<Akteure>}
	
	\qanf 	{Docker Container}
			{Um die Plattformunabhängigkeit zu gewährleisten soll sowohl die Server Komponente, als auch die KI-Komponenten mit Hilfe eines Docker Container veröffentlicht werden.}
			{Docker Container bieten den Vorteil, dass die Software nicht auf jedem Zielsystem neu compiliert werden muss sondern, sobald sie auf einem System in einem Docker-Container lauffähig gemacht wurde lässt sich dieser Container in der Regel auf diversen anderen Zielsystemen ausführen.}
			{-}
			{+}
			{<Akteure>}
	
	\qanf 	{Dokumentation}
			{Alle Klassen und Methoden der Software müssen so dokumentiert werden. Dabei sollen zumindest alle Übergabeparameter und Rückgabewerte genau spezifiziert werden. Zudem ist sind komplexe Algorithmen detailliert zu dokumentieren}
			{<Begründung>}
			{-}
			{+}
			{<Akteure>}
	
	\qanf 	{Benutzerhandbuch}
			{Zu jeder Komponente des Projektes muss eine Benutzerhandbuchh existieren, in welchem alle Features unmissverständlich erklärt sind, sodass ein neuer Benutzer auf Basis des Benutzerhandbuches die Software bedienen kann.}
			{<Begründung>}
			{-}
			{+-}
			{<Akteure>}
	
	\qanf 	{Anwendungssprache}
			{Das User-Interface der Anwendungen soll in deutscher Sprache gestaltet werden.}
			{Die Anwendungssprache ist im Lastenheft vorgegeben.}
			{-}
			{<Prio>}
			{<Akteure>}
	
	\qanf 	{Programmiersprache}
			{Implementierungssprache}
			{<Begründung>}
			{-}
			{<Prio>}
			{<Akteure>}
	
	\qanf 	{Dokumentationssprache}
			{<Beschreibung>}
			{<Begründung>}
			{-}
			{<Prio>}
			{<Akteure>}
	
	\qanf 	{Format für Konfigurationsdateien}
			{Alle Konfigurationsdateien müssen dem \textit{JSON} Standard genügen. Des Weiteren sind alle vom Komitee festgelegten weiteren Standards einzuhalten.}
			{<Begründung>}
			{-}
			{<Prio>}
			{<Akteure>}
	
	\qanf 	{Netzwerkkommunikation}
			{Die Netzwerkkommunikation zwischen Client und Server soll über sogenannte \textit{Web-Socket-Sessions} realisiert werden, sodass Client und Server ortsunabhängig von einander betrieben werden können.}
			{Die Netzwerkkommunikation muss gewissen Standards genügen, damit Client- und Serveranwendungen von unterschiedlichen Entwicklerteams mit einander kompatibel sind und Client und Server ortsunabhängig von einander betrieben werden können.}
			{-}
			{<Prio>}
			{<Akteure>}
	
	\qanf 	{Log-Dateien}
			{Log-Dateien anlegen um unter anderem den Spielverlauf zu Speichern und eventuelle Fehlfunktionen der Software fest zu halten.}
			{Log-Dateien können unter anderen die Wartung der Software erleichtern und für zusätzliche Features, wie eine Statistik über den Spielverlauf verwendet werden.}
			{-}
			{-}
			{<Akteure>}
	
	\qanf 	{Funktionalität}
			{Die Anwendungen müssen alle im Lastenheft als minimal Anforderungen aufgeführten Anforderungen erfüllen.}
			{<Begründung>}
			{-}
			{<Prio>}
			{<Akteure>}
	
	\qanf 	{Zuverlässigkeit}
			{Nur in maximal einer von 50 Sessions, darf sich eine Anwendung komplett aufhängen.}
			{Durch zu häufiges Abstürzen der Software ist das Benutzererlebnis massiv beeinträchtigt.}
			{-}
			{+}
			{<Akteure>}
	
	\qanf 	{Robustheit}
			{Die Anwendungen dürfen nicht aufgrund einer Falschen Benutzereingabe abstürzen, sondern müssen den Benutzer auf seinen Fehler hinweisen.}
			{Um das Benutzererlebnis nicht zu beeinträchtigen und keine Sicherheitslücken zu verursachen ist es notwendig, dass die Funktion der Software nicht durch fehlerhafte Benutzereingaben beeinträchtigt wird.}
			{-}
			{++}
			{<Akteure>}
	
	\qanf 	{Benutzbarkeit}
			{Dem Endnutzer muss es möglich sein alle Komponenten des Projektes nur auf Basis des Mitgelieferten Benutzerhandbuches und den Hilfeseiten die Software ohne Einschränkungen bedienen zu können.}
			{Wenn es für die Endnutzer der Software zu kompliziert ist die Software zu Benutzen, dann ist das Benutzererlebnis erheblich gestört und die Software wird nicht Benutzt werden, da die Endbenutzer unzufrieden sind.}
			{-}
			{+}
			{<Akteure>}
			
	\qanf 	{Wartbarkeit}
			{Die Software muss so aufgebaut sein, dass einzelne Teilstücke bei Bedarf ohne Umbauten der übrigen Software ersetzbar sind.}
			{Im Falle einer Fehlfunktion in einem Teilstück der Software muss diese einfach austauschbar sein um den Fehler schnellst möglich beheben zu können. Zudem sollte das Hinzufügen weiterer Features möglich sein um das Produkt stetig weiter entwickeln zu können.}
			{-}
			{-+}
			{<Akteure>}
	
	\qanf 	{Effizienz}
			{<Beschreibung>}
			{<Begründung>}
			{-}
			{-}
			{<Akteure>}
	
	\qanf 	{Kurze Ladezeiten}
			{Systembedingte Ladezeiten der Software dürfen auf einem aktuellen Mittelklasse Computer fünf Sekunden pro geladenem Teil überschreiten.}
			{Bei längeren Ladezeiten ist das Benutzererlebnis massiv beeinträchtigt.}
			{-}
			{-+}
			{-}
			
	\qanf 	{<Titel>}
			{<Beschreibung>}
			{<Begründung>}
			{-}
			{<Prio>}
			{<Akteure>}
			
	\qanf 	{<Titel>}
			{<Beschreibung>}
			{<Begründung>}
			{-}
			{<Prio>}
			{<Akteure>}
	
	
\end{document}
