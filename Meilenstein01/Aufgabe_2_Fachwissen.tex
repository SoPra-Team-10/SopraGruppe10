\begriff{Nutzer}
{Ein Mensch, der einen Rechner bedient und entweder den Client zum Spielen des Spiels oder zur Beobachtung einer Partie benutzt, oder den Team-Editor bedient. Jeder Nutzer hat einen Nutzernamen, mittels dem er von anderen Nutzern erkannt werden kann.}
{-}
{Spieler, Gast}
{Zur Beschreibung des Programmverlaufs}
{-}
{Der Nutzer \glqq{}JägerMaister69\grqq{} ist der Partie beigetreten.}

\begriff{Spieler}
{Ein Nutzer, der in dem Computerspiel \glqq{}Fantastic Feasts\grqq{} eine Partie gegen einen anderen Nutzer bestreitet. Er meldet sich beim Server als Spieler an und führt die Partie mittels Aktionen. Pro Partie kann es höchstens zwei Spieler geben.}
{Nutzer}
{-}
{Zur Beschreibung des Programmverlaufs}
{-}
{Der Spieler \glqq{}KäptnCola\grqq{} hat die Partie verlassen.}

\begriff{Gast}
{Ein Nutzer, der eine laufende Partie beobachtet. Er kann den derzeitigen Stand der Partie und des Spielfeldes sehen, sie aber nicht beeinflussen.}
{Nutzer}
{-}
{Zur Beschreibung des Programmverlaufs}
{-}
{Der Nutzer \glqq{}TequilaLove\grqq{} ist der Partie als Gast beigetreten.}

\begriff{Client}
{Das Computerprogramm, das mit einer grafischen Oberfläche ausgestattet ist und einem Nutzer erlaubt, eine Verbindung mit einem Server herzustellen und damit zu kommunizieren. Der Client erlaubt es einem Nutzer, einer Partie als Spieler mit einer ausgewählten Team-Konfiguration oder als Gast beizutreten. Dies wird durch ein Hauptmenü organisiert. In der Partie dient die grafische Oberfläche dazu, das Spielgeschehen zu visualisieren und die Steuerung der Spielfiguren zu ermöglichen.}
{-}
{-}
{Zum Spielen des Spiels \glqq{}Fantastic Feasts\grqq{}}
{Der Begriff bezieht sich nicht auf den Menschen, der das Programm bedient. Der Client ist eine eigenständige Komponente.}
{-}

\begriff{Server}
{Die zentrale Komponente, in der die Spiellogik implementiert ist. Mit dem Server können sich Clients verbinden, um eine Partie zu spielen oder zu beobachten. Die Kommunikation erfolgt über Nachrichten im JSON-Format. Es handelt sich dabei um eine Kommandozeilenanwendung. Ein Administrator startet den Server auf einem Port eines Rechners mit einer ausgewählten Partie-Konfiguration.}
{-}
{-}
{Ist für die Kommunikation von Clients, für das Verwalten des Spielgeschehens, Ressourcenverwaltung und die Spiellogik verantwortlich}
{Der Server wird mit mittels Kommandozeile gestartet.}
{-}

\begriff{Konfigurator}
{Ermöglicht einem Nutzer mit einer grafischen Oberfläche, Team- und Partie-Konfigurationen zu erstellen, zu editieren, zu speichern und zu laden. Die Einstellungen werden danach als JSON-Datei gespeichert.}
{-}
{-}
{Zur Erstellung von nutzereigenen Teams}
{Der Konfigurator hat eine grafische Oberfläche.}
{-}

\begriff{KI-Client}
{Der KI-Client meldet sich beim Server wie ein normaler Client an und simuliert mit einer KI einen menschlichen Spieler. Er hat keine grafische Oberfläche. }
{-}
{-}
{Zum Spielen gegen einen Computergegner}
{Meldet sich mit dem Nutzernamen \glqq{}KI\grqq{} an.}
{-}

\begriff{KI}
{Definiert die Regeln, nach denen der KI-Client auf die durch den Server vermittelten Geschehen im Spiel reagiert. Es soll möglich sein, Schwierigkeitsgrade auszuwählen.}
{-}
{-}
{Zum Spielen gegen einen Computergegner}
{Die KI ist die Logik, nach der der Computer das Spiel spielt und kein Programm.}
{-}

\begriff{Spielfeld}
{Ein grafisch darstellbares Raster, auf dem sich alle Spielobjekte bewegen.}
{-}
{-}
{Dient als virtuelles Spielbrett mit klar definierten Abgrenzungen}
{Wird nicht Spielumgebung genannt um Verwechslung mit dem Client zu vermeiden.}
{Die Spielfiguren werden auf dem Spielfeld platziert.}

\begriff{Zelle}
{Die kleinste Einheit des Spielfeldes, also ein Quadrat davon.}
{-}
{Zentrumszelle, Torring, Hüterzonenzelle}
{Mögliche Standorte der Spielobjekte}
{Wird nicht Feld genannt, um Verwechselungen mit dem Spielfeld zu vermeiden.}
{Ein Spieler wählt die Zelle, auf die er seine Spielfigur bewegen möchte.}

\begriff{Zentrum}
{Der 3x3 Zellen große Abschnitt in der Mitte des Spielfeldes.}
{-}
{-}
{Summe aller Zentrumszellen}
{Bezieht sich auf das Mittelfeld im Lastenheft. Die Bälle befinden sich zu Spielbeginn im Zentrum.}
{-}

\begriff{Hüterzone}
{Die Bereiche am linken und rechten Rand des Spielfeldes, in denen sich die Torringe befinden.}
{-}
{-}
{Summe aller Hüterzonenzellen und Torringe}
{Nur eine gegnerische Spielfigur darf sich gleichzeitig in der eigenen Hüterzone befinden.}
{-}

\begriff{Torring}
{Die Zellen, in die beide Teams den Quaffel bewegen wollen. Es wird zwischen eigenen und gegnerischen Torringen unterschieden.}
{Zelle}
{Eigener Torring, Gegnerischer Torring}
{Hauptquelle von Punkten}
{Die Torringe befinden sich innerhalb der Hüterzone.}
{-}

\begriff{Zentrumszelle}
{Eine Zelle im Zentrum des Spielfeldes (siehe Zentrum).}
{Zelle}
{-}
{Startpunkt für Quaffel und Klatscher}
{-}
{-}

\begriff{Hüterzonenzelle}
{Eine Zelle in einer Hüterzone.}
{Zelle}
{-}
{Limitierendes Element für das Abliefern des Quaffel}
{Betritt ein Jäger eine Hüterzonenezelle, wird geprüft, ob sich bereits ein weiterer Jäger desselben Teams auf einer anderen Hüterzonenzelle befindet.}
{-}

\begriff{Spielobjekt}
{Jedes Objekt, das sich auf dem Spielfeld befindet und darauf bewegt werden kann.}
{-}
{Ball, Spielfigur}
{-}
{Nicht zu verwechseln mit Spielfigur.}
{Die Spielobjekte werden vom Client visualisiert.}

\begriff{Ball}
{Ein Spielobjekt, das nicht direkt, nur indirekt von einem Spieler beeinflusst werden kann. Bälle erfüllen bestimmte Funktionen und dienen als zentrale Elemente, um die herum die Spieler ihre Aktionen richten.}
{Spielobjekt}
{Quaffel, Klatscher, Schnatz}
{Festpunkte zur Steuerung des Spielverlaufs}
{-}
{Der Quaffel ist ein Ball.}

\begriff{Spielfigur}
{Ein Spielobjekt, das von einem Spieler direkt gesteuert wird. Jede Spielfigur hat einen Namen, einen Besenrang, ein Geschlecht, ein Team und eine Rolle. Man unterscheidet außerdem zwischen eigenen und gegnerischen Spielfiguren.}
{Spielobjekt}
{Hüter, Sucher, Jäger, Treiber}
{Mitglieder eines Teams}
{Spieler im Lastenheft}
{Zu Spielbeginn platzieren die Spieler ihre Spielfiguren auf dem Spielfeld.}

\begriff{Quaffel}
{Passives Spielobjekt, mit dem Jäger und Hüter interagieren können und von ihnen nach Möglichkeit in einen gegnerischen Torring befördert werden soll.}
{Ball}
{-}
{Zentrales Spielobjekt}
{Der Quaffel bietet den Spielern die Möglichkeit, Punkte zu sammeln.}
{-}

\begriff{Klatscher}
{Ball, der sich von selbst auf Spielfiguren zubewegt, die keine Treiber sind und diese betäuben kann. Kann von Treibern geschlagen und dadurch beeinflusst werden.}
{Ball}
{-}
{Zusätzliches taktisches Spielelement}
{-}
{-}

\begriff{Schnatz}
{Beschreibung: Ball, der sich selbstständig bewegt und von den Suchern gejagt wird. Wird er von einem Sucher gefangen, bekommt dessen Team 30 Punkte und die Partie ist zu Ende.}
{Ball}
{-}
{Definiert Spielende}
{Der Schnatz erscheint erst im Laufe der Partie und ist nicht dierkt von Anfang an vorhanden.}
{-}

\begriff{Partie}
{Ein einzelnes Spiel. Beginnt beim Platzieren der Figuren und endet mit dem Bestimmen des Gewinners.}
{-}
{-}
{Beschreibung des Spielablaufs}
{-}
{Spieler \glqq{}VodkaVodka98\grqq{} spielt eine Partie gegen Spieler \glqq{}LongEiländ\grqq{}}

\begriff{Hüter}
{Spielfigur, deren Aufgabe es ist, den Quaffel von den eigenen Torringen fernzuhalten.}
{Spielfigur}
{Eigener Hüter, Gegnerischer Hüter}
{Letzte Verteidigungslinie}
{Jedes Team hat genau einen Hüter.}
{Nach einem erfolgreichen Torschuss bekommt der gegnerische Hüter den Quaffel.}

\begriff{Sucher}
{Spielfigur, die durch Einfangen des Schnatzes eine Partie beenden kann.}
{Spielfigur}
{Eigener Sucher, Gegnerischer Sucher}
{Beendet die Partie}
{Jedes Team hat genau einen Sucher.}
{Der Sucher \glqq{}Darth Vader\grqq{} fängt den Schnatz und beendet damit die Partie.}

\begriff{Jäger}
{Spielfigur, die den Quaffel in ein einen gegnerischen Torring befördern soll.}
{Spielfigur}
{Eigener Jäger, Gegnerischer Jäger}
{Holt Punkte für das eigene Team}
{Jedes Team hat genau 3 Jäger.}
{Der Jäger \glqq{}Han Solo\grqq{} erzielt ein Tor.}

\begriff{Treiber}
{Spielfigur, mit der der Spieler eigene Spielfiguren vor Klatschern schützt und gegnerische damit abschießen kann.}
{Spielfigur}
{Eigener Treiber, Gegnerischer Treiber}
{Interagiert mit Klatschern}
{Jedes Team hat genau 2 Treiber.}
{Der Treiber \glqq{}Boba Fett\grqq{} schlägt den Klatscher auf die Zelle 3:1.}

\begriff{Geschlecht}
{Ein Parameter, der bei der Team-Konfiguration für jede Spielfigur entweder auf \glqq{}männlich\grqq{} oder \glqq{}weiblich\grqq{} gesetzt werden muss.}
{-}
{-}
{Team-Editierung}
{-}
{-}

\begriff{Team}
{Die Menge aller Spielfiguren auf dem Spielfeld, die von einem einzigen Spieler kontrolliert wird. Ein Team hat einen Namen, ein Motto, ein Logo, eine Hauptfarbe und eine Ersatzfarbe. Ein Nutzer hat die Möglichkeit, seine Team-Konfiguration mit dem Client zu ändern. Die Team-Konfigurationen können mit dem Konfigurator verändert und erstellt werden.}
{eigenes Team, gegnerisches Team}
{-}
{Beschreibung einer Partie}
{Ein Team besteht aus 7 Spielfiguren: Ein Sucher, ein Hüter, drei Jäger und zwei Treiber.}
{Ein Nutzer wählt sein zuvor im Team-Editor konfiguriertes Team im Client aus.}

\begriff{Punkte}
{Der Spieler mit mehr Punkten am Ende der Partie gewinnt. Ein Team kann Punkte erzielen, indem es den Quaffel durch einen gegnerischen Torring wirft oder den Schnatz fängt.}
{-}
{-}
{Bestimmung des Gewinners}
{Ein erfolgreicher Torschuss ist 10 Punkte wert, das Fangen des Schnatzes 30.}
{-}

\begriff{Besetzen}
{Eine Spielfigur besetzt die Zelle, auf der sie sich befindet.}
{-}
{-}
{Beschreibung des Spielgeschehens}
{Zwei Spielfiguren können nicht ein und die selbe Zelle besetzen.}
{Die Spielfigur \glqq{}Chewbacca\grqq{} besetzt die Zelle 5:3.}

\begriff{Besenrang}
{Jede Spielfigur hat einen Besenrang von 1 bis 5, der die Wahrscheinlichkeit bestimmt, dass sie nach dem Ziehen erneut um ein Feld ziehen darf. Der Besenrang 1 ist der beste.}
{-}
{-}
{Unterscheidet Qualität der Spielfiguren.}
{Ersetzt die verschiedenen \glqq{}Besen\grqq{} aus dem Lastenheft mit einer Skala von 1 bis 5 zur besseren Übersicht.}
{Die Spielfigur \glqq{}Yoda\grqq{} hat den Besenrang 1.}

\begriff{Aktion}
{Jede durch einen Spieler hervorgerufene Änderung der Spielsituation. Ein Spieler führt Aktionen mittels des Clients durch.}
{-}
{Ziehen, Schießen, Schlagen, Einmischung, Übernahme, Befördern}
{Weiterführung der Partie}
{-}
{Spieler steuern das Spielgeschehen durch Aktionen.}

\begriff{Ziehen}
{Die Bewegung einer Spielfigur von einer Zelle auf eine andere durch direkten Befehl des Spielers.}
{Aktion}
{-}
{Beschreibung des Spielverlaufs}
{Bezieht sich nicht auf erzwungene Bewegungen einer Spielfigur.}
{Die Spielfigur \glqq{}Obiwan Kenobi\grqq{} zieht von Zelle 8:7 auf Zelle 9:7.}

\begriff{Befördern}
{Bewegen des Quaffel mittels einer Spielfigur.}
{-}
{Aktion}
{Bewegen des Quaffel, allgemeiner Begriff}
{Nicht zu verwechseln mit Schießen. Eine Spielfigur kann den Quaffel aktiv befördern (also eine Aktion der Spielfigur), wenn die Spielfigur den Quaffel hält und mit ihm auf eine andere Zelle zieht. Der Quaffel kann aber auch durch ein anderes Ereignis auf eine Zelle befördert werden.}
{Ein Hüter zieht, während er den Quaffel hält, auf eine andere Zelle und befördert ihn damit aktiv (Aktion des Hüters). Wenn ein Treiber auf Zelle zieht, auf der der Quaffel liegt, wird der Quaffel indirekt auf eine andere freie Zelle befördert.}

\begriff{Schießen}
{Die Bewegung des Quaffel durch einen Hüter oder Jäger auf eine andere, entfernte Zelle ohne Bewegung der Spielfigur.}
{Aktion}
{-}
{Bewegung des Quaffel um mehrere Felder}
{\glqq{}Werfen\grqq{}  im Lastenheft. Analog zum Schussvektor benannt.}
{Der Jäger \glqq{}Mace Windu\grqq{} schießt den Quaffel auf Zelle 10:4.}

\begriff{Schlagen}
{Die erzwungene Bewegung eines Klatschers durch einen Treiber.}
{Aktion}
{-}
{Interaktion mit Klatschern}
{\glqq{}Kloppen\grqq{}  im Lastenheft}
{Der Treiber \glqq{}R2-D2\grqq{} schlägt einen Klatscher auf Zelle 5:10.}

\begriff{Einmischung}
{Hilfsfähigkeiten, die nicht von Spielobjekten ausgehen. Werden von einem Spieler gesteuert. Bei jeder Benutzung besteht eine Chance, dass die verwendete Einmischung bis zum Ende der Partie für den jeweiligen Spieler vom Schiedsrichter deaktiviert werden.}
{Aktion}
{Teleportation, Fernangriff, Impuls, Schnatzjagd}
{Zusätzliche taktische Element}
{Ersetzt die \glqq{}Fans\grqq{}  aus dem Lastenheft.}
{Der Schiedsrichter erkennt eine Einmischung mit einer in der Partie-Konfiguration festgelegten Wahrscheinlichkeit.}

\begriff{Teleportation}
{Einmischung, die eine Spielfigur auf eine zufällige Zelle bewegt.}
{Einmischung}
{-}
{-}
{Ersetzt \glqq{}Elfen\grqq{}  aus Lastenheft. Die teleportierte Spielfigur kann aus einem beliebigen Team sein.}
{Der Sucher \glqq{}Lando Calrissian\grqq{} wird auf Zelle 6:6 teleportiert.}

\begriff{Fernangriff}
{Trifft eine gegnerische Spielfigur. Das Ziel verliert gegebenenfalls den Quaffel und wird auf eine zufällige benachbarte, freie Zelle bewegt.}
{Einmischung}
{-}
{-}
{Statt \glqq{}Kobolde\grqq{}  im Lastenheft}
{Der Treiber \glqq{}Jango Fett\grqq{} wird von einem Fernangriff auf Zelle 5:6 gestoßen.}

\begriff{Impuls}
{Wenn der Quaffel von einer Spielfigur gehalten wird, wird er von dieser verloren.}
{Einmischung}
{-}
{-}
{Statt \glqq{}Trolle\grqq{}  im Lastenheft}
{Der Hüter \glqq{}C-3PO\grqq{} verliert wegen eines Impulses den Quaffel.}

\begriff{Schnatzstoß}
{Bewegt den Schnatz in eine zufällige Richtung um eine Zelle.}
{Einmischung}
{-}
{-}
{\glqq{}Schnatzschnappen\grqq{}  im Lastenheft}
{Ein Schnatzstoß treibt den Schnatz auf Zelle 4:12.}

\begriff{Entfernung}
{Eine Entfernung zwischen zwei Zellen ist, wie oft eine Spielfigur mindestens ziehen muss, um von der Einen auf die Andere zu  gelangen.}
{-}
{-}
{Spielfeldgeometrie}
{-}
{Die Entfernung zwischen dem Jäger und dem Torring beträgt drei Zellen.}

\begriff{Schussvektor}
{Pfeil vom Mittelpunkt einer Zelle zum Mittelpunkt einer anderen.}
{-}
{Torschussvektor}
{Spielfeldgeometrie}
{-}
{Der Client visualisiert den Schussvektor für einen Schuss vom jeweiligen Jäger oder Hüter auf die vom Spieler ausgewählte Zelle.}

\begriff{Torschussvektor}
{Schussvektor zu einem Schuss, der möglicherweise in einem Torschuss resultiert. (Ein Schussvektor, der die linke oder rechte Seite eines Torrings schneidet.)}
{Schussvektor}
{-}
{Punkte sammeln}
{Auch wenn ein Schussvektor durch einen Torring verläuft (und damit ein Torschussvektor ist), kann der Schuss den Torring verfehlen.}
{Ein Schussvektor, der von einer Zentrumszelle zu einer anderen zeigt, ist kein Torschussvektor. Ein Schussvektor, der von der Zelle links von einem Torring auf die Zelle rechts davon zeigt, ist ein Torschussvektor.}

\begriff{Torschuss}
{Ein Jäger schießt den Quaffel in einen Torring und holt damit Punkte für sei Team.}
{-}
{-}
{Punkte sammeln}
{Nur erfolgreiche Schüsse auf das Tor werden als Torschüsse bezeichnet. Nach einem Torschuss bekommt der Hüter des verteidigenden Team den Quaffel.}
{Der Jäger \glqq{}Darth Sidious\grqq{} führt einen erfolgreichen Torschuss aus und erzielt 10 Punkte für sein Team.}

\begriff{Runde}
{Der Zeitraum, in dem jedes Spielobjekt auf dem Spielfeld einmal in einer Rundenphase gesteuert wird und jede verfügbare Einmischung einmal ausgeführt werden kann.}
{-}
{-}
{Zeiteinteilung}
{Nicht die Rundenphasephase einer Spielfigur}
{Die Überlängenbehandlung tritt nach einer in der Partie-Konfiguration festgelegten Anzahl an Runden in Kraft.}

\begriff{Rundenphase}
{Ein Zeitraum, in dem eine einzelne Spielfigur entweder von einem Spieler oder automatisch vom KI-Client gesteuert werden kann. Eine Runde besteht aus mehreren Rundenphasen.}
{-}
{-}
{Zeiteinteilung}
{Die Rundenphase einer Spielfigur beinhaltet das Ziehen der Spielfigur (wenn gewünscht) und gegebenenfalls eine weitere Aktion.}
{Eine Spielfigur zieht um eine Zelle und schießt danach den Quaffel, der in ihrem Besitz war.}

\begriff{Endphase}
{Letzter Teil einer Runde. Der Spieler kann darin Einmischungen vornehmen.}
{-}
{-}
{Zeiteinteilung}
{-}
{Die KI setzt während ihrer Endphase Teleportation ein, um den gegnerischen Jäger von den eigenen Torringen fernzuhalten.}

\begriff{Verlieren}
{Der Quaffel wird auf eine zufällige freie Zelle bewegt, die an die Zelle angrenzt, auf der sich die Spielfigur, die bis jetzt in Ballbesitz war, befindet.}
{-}
{-}
{Spielablauf}
{\glqq{}Vertändeln\grqq{}  im Lastenheft}
{Der Jäger \glqq{}Jar Jar\grqq{} verliert den Quaffel, der sich dadurch um eine Zelle nach links bewegt.}

\begriff{Halten}
{Ein Jäger oder Hüter kann den Quaffel halten. Ist das der Fall, bewegt sich der Quaffel auf die Zelle, auf die die Spielfigur zieht.}
{-}
{-}
{Beschreibung des Spielgeschehens}
{-}
{-}

\begriff{Übernahme}
{Ein Jäger neben einer gegnerischen Spielfigur, die den Quaffel hält, kann diesen mit einer bestimmten Wahrscheinlichkeit übernehmen und hält ihn anschließend selbst.}
{Aktion}
{-}
{Aggressives Spielmanöver}
{-}
{Der Jäger \glqq{}Darth Vader\grqq{} übernimmt den Quaffel vom Hüter \glqq{}Anakin Skywalker\grqq{}.}

\begriff{Betäubt}
{Eine betäubte Spielfigur kann in ihrer nächsten Rundenphase keine Aktion durchführen.}
{-}
{-}
{Wirkung der Klatscher}
{\glqq{}Ausgeknockt\grqq{}  im Lastenheft}
{-}

\begriff{Foul}
{Handlung, wegen der eine Spielfigur vorübergehend vom Spielfeld entfernt werden kann.}
{-}
{Torring Blockieren, Stürmen, Großoffensive, Rammen, Schnatz Blockieren}
{Taktische Elemente}
{Fouls sind Aktionen, die von der Spielmechanik erlaubt sind, jedoch trotzdem bestraft werden können.}
{Der Hüter \glqq{}Qui-Gon Jinn\grqq{} wird wegen eines Fouls durch den Schiedsrichter vom Spielfeld entfernt.}

\begriff{Torring Blockieren}
{Eine eigene Spielfigur besetzt einen eigenen Torring, was einen Torschuss unmöglich macht.}
{Foul}
{-}
{Taktik}
{\glqq{}Flackern\grqq{}  im Pflichtenheft}
{Die KI weist einen ihrer Jäger an, den Torring zu blockieren, um einen gegnerischen Torschuss zu verhindern.}

\begriff{Stürmen}
{Ein Jäger, der den Quaffel hält, zieht auf einen gegnerischen Torring, was das Abliefern garantiert.}
{Foul}
{-}
{Taktik}
{\glqq{}Nachtarocken\grqq{}  im Lastenheft}
{Der Jäger \glqq{}Han Solo\grqq{} stürmt den mittleren gegnerischen Torring und erzielt dadurch ein Tor.}

\begriff{Großoffensive}
{Ein eigener Jäger betritt eine gegnerische Hüterzonenzelle während ein anderer eigener Jäger sich ebenfalls in der gegnerischen Hüterzone befindet.}
{Foul}
{-}
{Taktik}
{\glqq{}Stutschen\grqq{}  im Lastenheft}
{Der Jäger \glqq{}Lando Calrissia\grqq{} schließt sich dem Jäger \glqq{}Chewbacca\grqq{} in einer Großoffensive an.}

\begriff{Rammen}
{Eine eigene Spielfigur zieht auf eine Zelle, die von einer gegnerischen Spielfigur besetzt wird. Dadurch wird die gegnerische Spielfigur auf eine benachbarte Zelle bewegt und verliert ggf. den Quaffel}
{Foul}
{-}
{Taktik}
{\glqq{}Keilen\grqq{}  im Lastenheft}
{\glqq{}Boba Fett\grqq{} rammt \glqq{}Jar Jar\grqq{}.}

\begriff{Schnatz blockieren}
{Eine Spielfigur, die kein Sucher ist, besetzt die Zelle, auf der sich der Schnatz befindet.}
{Foul}
{-}
{Taktik}
{\glqq{}Schnatzeln\grqq{}  im Lastenheft}
{\glqq{}Darth Maul\grqq{} blockiert den Schnatz.}

\begriff{Schiedsrichter}
{Entfernt mit einer bestimmten Wahrscheinlichkeit eine Spielfigur, die ein Foul ausführt, vom Spielfeld bis ein Torschuss erfolgt. Ebenfalls ahndet er mit einer bestimmten Wahrscheinlichkeit eine Einmischung  und deaktiviert diese dadurch permanent für den Rest der Partie (für das Team, das die Einmischung ausgeführt hatte).}
{-}
{-}
{Taktik}
{\glqq{}Schiedsrichter\grqq{}  im Lastenheft}
{\glqq{}Sheev Palpatine\grqq{} wurde vom Schiedsrichter vom Spielfeld entfernt.}

\begriff{Disqualifikation}
{Tritt ein wenn drei oder mehr Spielfiguren eines Spielers in der selben Runde durch den Schiedsrichter aus dem Spiel entfernt wurden. Führt zur Niederlage des Spielers.}
{-}
{-}
{Erhöhtes Risiko}
{-}
{Der Spieler \glqq{}CubaLibre\grqq{} gewinnt die Partie durch Disqualifikation des Gegners.}
