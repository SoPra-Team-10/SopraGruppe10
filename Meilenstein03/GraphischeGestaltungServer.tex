\subsubsection{ServerInit}
    \lstset{language=bash} 
    \begin{lstlisting}[frame=single]
    $ server -p1230 standard_partie.json
    ... pending
    \end{lstlisting}
    Das Beispiel stellt einen Aufruf des Servers mit entsprechenden Argumenten dar. Mit dem Parameter wird eine Port-Nummer angegeben. Das zweite Argument gibt den Pfad zu einer gültigen Partie-Konfigurationsdatei an.

    \subsubsection{ServerRunning}
    \lstset{language=bash} 
    \begin{lstlisting}[frame=single]
    Fantastic Feasts server is running on Port 1230 ...
    \end{lstlisting}
    Im Falle einer erfolgreichen Initialisierung wird dem Systemadministrator diese Nachricht mitgeteilt. Es können hier auch weitere Informationen über den initialisierten Server ausgegeben werden.

    \subsubsection{ServerError}
    \lstset{language=bash} 
    \begin{lstlisting}[frame=single]
    Error: Initialization failed
    \end{lstlisting}
    Im Falle eines Fehlers bei der Initialisierung wird der Systemadministrator mit dieser Fehlermeldung benachrichtigt.


