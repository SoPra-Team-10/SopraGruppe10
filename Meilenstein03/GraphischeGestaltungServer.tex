\subsection{Server}
	\subsubsection{ffServerInit}
	\lstset{language=bash} 
	\begin{lstlisting}[frame=single]
  	$ ffserver 1230 standard_partie.json
  	... pending
	\end{lstlisting}
	Das Beispiel für einen Aufruf des Servers mit entsprechenden Parametern dar.
	
	\subsubsection{ffServerRunning}
	\lstset{language=bash} 
	\begin{lstlisting}[frame=single]
  	Fantastic Feasts server is running on Port 1230 ...
	\end{lstlisting}
	Im Falle einer erfolgreichen Initialisierung wird dies dem Systemadministrator über diese Nachricht mitgeteilt. In dieser Nachricht können weitere Informationen über den initialisierten Server verpackt werden.
	
	\subsubsection{ffServerError}
	\lstset{language=bash} 
	\begin{lstlisting}[frame=single]
  	Error: Initialization failed
	\end{lstlisting}
	Im Falle eines Fehlers bei der Initialisierung wird der Systemadministrator mit dieser Fehlermeldung benachrichtigt.


