Diese Komponente simuliert einen Menschlichen Gegner, der sich wie ein normaler Client bei einem Spielserver anmeldet und dann autonom seine Spielentscheidungen trifft.

\subsubsection{Schnittstellenarten}
Es besteht kein Grund, für diese Komponente eine grafische Oberfläche bereitzustellen, da die Anwendung zur Laufzeit keine Eingabe von einem menschlichen Benutzer erwartet und eine Partie mittels des Clients verfolgt werden kann. \\
Für eine Kommandozeilenanwendung ist es einfacher, Plattformunabhängigkeit sicherzustellen. Außerdem wird es damit problemlos möglich, den KI-Client aus einem anderen Programm zu starten. Beispielsweise kann dem Client eine Funktion hinzugefügt werden, gegen die KI zu spielen, ohne dass der Benutzer den KI-Client extern starten muss.\\

\subsubsection{Dialoge}
Im Folgenden werden die Anforderungen an den KI-Client einem CLI-Dialog zugeordnet.

\begin{figure}[H]
    \centering
    \begin{tabular}{| l l l p{4cm} |}
    \hline
    \textbf{Name} & \textbf{Typ} & \multicolumn{2}{l|}{\textbf{Abgedeckte Anwendungsfälle}} \\\hline
    Init & CLI Befehl mit Params & FA73 - FA75 & Schwierigkeit, Server- und Team-Konfiguration einstellen.\\\hline
	InitFailure & Response & FA73 - FA75 & Feedback.\\\hline
    WaitingForGame & Response & FA55 & Netzwerkschnittstelle, allgemeine Kommunikation.\\\hline
    Playing & Response & allgemein & Feedback.\\\hline
    AttemptingReconnect & Response & FA55 & Netzwerkschnittstelle, allgemeine Kommunikation.\\\hline
    ConnectionLost & Response & FA55 & Netzwerkschnittstelle, allgemeine Kommunikation.\\\hline
    \end{tabular}
\end{figure}

\subsubsection{Dialogstruktur}  
\textbf{Init:} Der KI-Client wird über die Kommandozeile gestartet. Serverkonfiguration, Team-Konfiguration, die maximale Anzahl an Reconnect-Versuchen und der Schwierigkeitsgrad werden beim Start der Anwendung mittels Kommandozeilenparametern gehandhabt. Der Server, mit dem sich der KI-Client verbinden soll wird als Argument übergeben. Die Team-Konfiguration und der Schwierigkeitsgrad können mittels Optionen verändert werden und nehmen ansonsten einen Standardwert an.\\
\textbf{InitFailure:} Wird eine ungültige Option angegeben, ist der Server nicht erreichbar oder wurde eine ungültige Team-Konfigurationsdatei geladen, so erscheint eine Fehlermeldung mit entsprechenden Hinweisen.\\
\textbf{WaitingForGame:} War die Initialisierung erfolgreich und der KI-Client konnte sich mit dem Server verbinden, so erscheint eine Nachricht, die darauf hinweist, dass noch auf den Beginn der Partie gewartet wird.\\
\textbf{Playing:} Während einer laufenden Partie wird ein Hinweis angezeigt.\\
\textbf{AttemptingReconnect:} Bei einem Verbindungsabbruch zeigt der KI-Client eine Meldung an und versucht automatisch, die Verbindung wiederherzustellen.\\
\textbf{ConnectionLost:} Konnte sich der KI-Client nicht innerhalb der angegebenen Anzahl von Versuchen erneut mit dem Server verbinden, so erscheint eine Fehlermeldung und das Programm beendet sich.\\
Der KI-Client beendet sich außerdem nach Abschluss einer Partie durch ein reguläres Spielende. 

\subsubsection{Zulässige Optionen:}
\begin{figure}[H]
    \centering
    \begin{tabular}{|p{2cm}|p{12cm} |}
        \hline
        Flag & Erklärung \\\hline
        -s & Legt den Schwierigkeitsgrad fest.\\
        & Akzeptiert eine ganze Zahl zwischen 0 und 2, wobei 0 für einfach,\\
        & 1 für mittelschwer und 2 für schwer steht.\\
        & Bei einer ungültigen Eingabe wird eine Fehlermeldung ausgegeben.\\\hline
        -t & Legt die Team-Konfiguration fest.\\
        & Akzeptiert einen String als Pfad zu einer JSON-Datei.\\
        & Existiert der angegebene Pfad nicht oder ist die Datei keine gültige Konfigurationsdatei,\\
        & wird eine entsprechende Fehlermeldung ausgegeben.\\\hline
		-r & Reconnects: Spezifiziert, wie oft der KI-Client versucht, die Verbindung nach einem Verbindungsabbruch 
		wiederherzustellen. Standardwert ist 5.\\\hline
        -{}-help & Zeigt eine Liste möglicher Optionen an.\\\hline
    \end{tabular}
\end{figure}
