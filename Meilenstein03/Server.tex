\subsection{Server}
Diese Komponente stellt den Server dar. Er hostet Spiele und enthält die Spiellogik. Zu Beginn wird er einmalig vom Systemadministrator gestartet und steht anschließend den Spielern von \textit{Fantastic Feasts} zur Verfügung.

\subsubsection{Schnittstellenarten}
Als Benutzerschnittstelle wird ein CLI verwendet. \textbf{Begründung:} Der Nutzer kommt mit dieser Komponente über keine Benutzerschnittstelle in Berührung. Deswegen spielen Look and Feel keine Rolle. Zusätzlich wird der Server nur einmal mit wenigen Parametern gestartet, benötigt zur Laufzeit dann keine weiteren Eingaben und muss auch keinerlei graphische Ausgabe zur Verfügung stellen, weswegen das CLI ausreicht.

\subsubsection{Dialoge}
Im Folgenden wird die Anforderung an den Server einem CLI-Dialog zugeordnet.

\begin{figure}[H]
    \centering
    \begin{tabular}{| l l l l |}
    \hline
    \textbf{Name} & \textbf{Typ} & \multicolumn{2}{l|}{\textbf{Abgedeckte Anwendungsfälle}} \\\hline
    ServerInit & CLI Befehl mit Params & FA74 & Serverkonfiguration einstellen \\\hline
    ServerRunning & Response & FA74 & Serverkonfiguration Feedback.\\\hline
    ServerError & Response & FA74 & Serverkonfiguration Feedback.\\\hline
    
    \end{tabular}
\end{figure}

\subsubsection{Dialogstruktur}  
Die Dialogstruktur des Servers lässt sich wie folgt beschreiben: \textbf{ServerInit:} Der Server lässt sich in der Konsole mit dem Namen der Server-Anwendung, dem Namen einer gültigen Partie-Konfiguration und einem Parameter aufrufen. Der Parameter ist die Port-Nummer, über die der Server erreichbar ist. Darauf gibt es zwei mögliches Antworten. \textbf{ServerRunning:} War die Initialisierung erfolgreich antwortet der Server mit einer entsprechenden Nachricht. \textbf{ServerEerror:} Im Falle eines Fehlers bei der Initialisierung wird mit einer Fehlernachricht geantwortet.
\subsubsection{Zulässige Optionen}
\begin{figure}[H]
    \centering
    \begin{tabular}{|p{2cm}|p{12cm} |}
        \hline
        Flag & Erklärung \\\hline
        -p & Legt die Portnummer fest.\\
        \hline
        -{}-help & Zeigt eine Liste möglicher Optionen an.\\\hline
    \end{tabular}
\end{figure}

