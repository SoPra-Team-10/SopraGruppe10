\includepdf[scale = 0.8, pagecommand={\section{Datenmodell}\subsection{Gesamtübersicht}Im folgenden werden die Domänen des Spiels modelliert. Die Diagrammstruktur orientiert sich an einem UML-2 Klassendiagramm. Das erste Diagramm gibt einen Überblick über die Domäne, die folgenden Diagramme bilden die spezifischen Aspekte der Domäne detailierter ab.}]{domaenenmodell.pdf}
\includepdf[scale = 0.8, pagecommand={\subsection{Einmischung}Gibt den Fantyp an.}]{fan.pdf}
\includepdf[scale = 0.8, pagecommand={\subsection{Foul}Es gibt verschiedene Fouls, welche in dem Domänenmodell alle aufgeführt werden und möglich sind.}]{foul.pdf}
\includepdf[scale = 0.8, pagecommand={\subsection{Besen}Alle in dem Domänenmodell verschiedenen Besen sind möglich zum auswählen.}]{besen.pdf}
\includepdf[scale = 0.8, pagecommand={\subsection{Schuss}Ablauf eines Schusses im Domänenmodell.}]{wurf.pdf}
\includepdf[scale = 0.8, pagecommand={\subsection{Zelle}Die verschiedenen Zelltypen, die in dem Spiel vorkommen.}]{zelle.pdf}
\includepdf[scale = 0.8, pagecommand={\subsection{Ball}Die verschiedenen Ballarten des Spiels.}]{ball.pdf}
\includepdf[scale = 0.8, pagecommand={\subsection{Interaktion der Spielerfiguren mit den Bällen}Die möglichen Interaktionen, die Spielfiguren jeweils mit den Bällen machen können.}]{ballSpieler.pdf}
