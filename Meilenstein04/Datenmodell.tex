\includepdf[scale = 0.8, pagecommand={\section{Datenmodell}\subsection{Gesamtübersicht}Im folgenden werden die Domänen des Spiels modelliert. Die Diagrammstruktur orientiert sich an einem UML-2 Klassendiagramm. Das erste Diagramm gibt einen Überblick über die Domäne, die folgenden Diagramme bilden die spezifischen Aspekte der Domäne detailierter ab.}]{domaenenmodell.pdf}
\includepdf[scale = 0.8, pagecommand={\subsection{Einmischung}}]{fan.pdf}
\includepdf[scale = 0.8, pagecommand={\subsection{Foul}}]{foul.pdf}
\includepdf[scale = 0.8, pagecommand={\subsection{Besen}}]{besen.pdf}
\includepdf[scale = 0.8, pagecommand={\subsection{Schuss}}]{wurf.pdf}
\includepdf[scale = 0.8, pagecommand={\subsection{Zelle}}]{zelle.pdf}
\includepdf[scale = 0.8, pagecommand={\subsection{Ball}}]{ball.pdf}
\includepdf[scale = 0.8, pagecommand={\subsection{Interaktion der Spielerfiguren mit den Bällen}}]{ballSpieler.pdf}
