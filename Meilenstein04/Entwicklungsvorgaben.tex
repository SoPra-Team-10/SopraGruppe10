Alle Teile der Anwendung werden in C++17 geschrieben. Als Compiler wird dafür entweder der GNU-C++-Compiler (GCC) oder Visual C++ (MSVC) verwendet, als Buildsystem CMake.
Zur statischen Codeanalyse wird Clang-Tidy verwendet, Unit Tests werden mit Google-Test bzw. Google-Mock implementiert.
Um undefiniertes Verhalten zu vermeiden wird das Programm während der Ausführung mit Adress Sanitizer überprüft.

Als Vorgehensmodell wird während der Entwicklung der Softwareteile Scrum verwendet, die Dauer eines Sprints beträgt 2 Wochen. Für jede Komponente wird ein Ansprechpartner definiert, dieser übernimmt die Koordination der Entwicklung dieser Komponente.
Das Entwicklungsteam besteht aus sechs Entwicklern, zusätzlich agiert der Tutor als Scrummaster. Der Quellcode wird mithilfe von Git versioniert, für jede Komponente existiert ein eigenes Repository, die Entwicklung der Features erfolgt auf dedizierten Feature-Branches, so dass der master-Branch immer in einem funktionalen Zustand bleibt.

Alle Teile des Quellcodes werden mithilfe von Doxygen-Kommentaren dokumentiert. Im wesentlichen soll sich während des Entwicklungsprozess an den Linux-Kernel Styleguide \footnote{https://www.kernel.org/doc/html/v4.20/process/coding-style.html} gehalten werden, zwei wesentliche Abweichungen von diesem Styleguide ist eine Einrückungstiefe von 4 Zeichen, sowie die Platzierung der öffnenden Klammer bei Funktionsaufrufen \footnote{Siehe Absatz 3 des Styleguides} in der selben Zeile.

