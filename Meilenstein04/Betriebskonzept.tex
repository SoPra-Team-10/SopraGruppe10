\subsection{Allgemein}
Um die Funktionalität der jeweilgen Anwendung zu nutzen, muss der verwendete Computer über eine Internetanbindung verfügen, die eine durchschnittliche Datenrate von $1 \frac{\text{mbit}}{\text{s}}$ mit einer Latenz von unter $200\text{ms}$ zur Verfügung stellt. Des weiteren sollte der verwendete Computer über eine moderne CPU und mindestens $2\text{GB}$ RAM verfügen.

Da die Anwendung keine alte Anwendung ersetzt, ist ein Ablösungskonzept nicht von nöten. Auch das Einführungskonzept gestaltet sich einfach, da die Anwendung primär von Privatpersonen genutzt werden soll. Diese können sich bei passendem Zielsystem eine kompilierte Fassung der Anwendung herunterladen oder die Anwendung selber kompilieren.

Die angestrebte Lebensdauer der Anwendung beträgt fünf Jahre. Für die Anwendung wird jedoch keine Haftung übernommen, insbesondere auch nicht für die angestrebte Lebensdauer.

Für die Verwendung des jeweiligen Programms fallen keine Kosten an. Das in diesem Dokument aufgeführte Konzept wird als final angesehen, es sind keine weiteren Erweiterungen geplant. Auch Instandhaltungen sind nach Abnahme des Projekts nicht geplant.

\subsection{Client}
Der Client soll auf einem aktuellen Desktopbetriebssystem mit graphischer Oberfläche eingesetzt werden. Das heißt im speziellen, aber nicht ausschließlich, auf einer aktuellen Linux-Distribution.

Die Anwendung hängt von der SFML-Bibliothek ab, große Veränderungen an dieser Bibliothek können zu einer Verkürzung der Lebensdauer führen.

\subsection{Server}
Die Serveranwendung soll auf einem aktuellen Betriebssystem mit Unterstützung für Docker eingesetzt werden. Das heißt im speziellen, aber nicht ausschließlich, auf einer aktuellen Linux-Distribution, auf macOS und auf Windows 10.

Durch eine Beschreibung der Konfigurationsoptionen der Applikation, also der Kommandozeilenargumente sowie der Konfigurationsdateien, in Form einer Dokumentation ist es dem Administrator möglich, sich über die Benutzung der Anwendung zu informieren. Dadurch sind gesonderte Schulungen nicht vonnöten.

Die Serveranwendung läuft in einem Docker-Container, weshalb ihre Funktionalität von Docker abhängig ist.

\subsection{Team- und Partiekonfigurator}
Der Team- und Partiekonfigurator soll auf einem aktuellen Desktopbetriebssystem mit graphischer Oberfläche eingesetzt werden. Das heißt im speziellen, aber nicht ausschließlich, auf einer aktuellen Linux-Distribution.

Die Anwendung hängt von der SFML-Bibliothek ab, große Veränderungen an dieser Bibliothek können zu einer Verkürzung der Lebensdauer führen.

\subsection{KI-Client}
Die Serveranwendung soll auf einem aktuellen Betriebssystem mit Unterstützung für Docker eingesetzt werden. Das heißt im speziellen, aber nicht ausschließlich, auf einer aktuellen Linux-Distribution, auf macOS und auf Windows 10.

Der KI-Client läuft in einem Docker-Container, weshalb seine Funktionalität von Docker abhängig ist.
